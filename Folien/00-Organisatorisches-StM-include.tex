%%%%%%%%%%%%%%%%%%%%%%%%%%%%%%%%%%%%%%%%%%%%%%%%%%%%
%%%             Metadata                         %%%
%%%%%%%%%%%%%%%%%%%%%%%%%%%%%%%%%%%%%%%%%%%%%%%%%%%%      

\title{Grundkurs Linguistik}

\subtitle{Organisatorisches}


\institute{Institut für deutsche Sprache und Linguistik}

%%%%%%%%%%%%%%%%%%%%%%%%%      
%\date{ }
%\publishers{\textbf{6. linguistischer Methodenworkshop \\ Humboldt-Universität zu Berlin}}

%\hyphenation{nobreak}


%%%%%%%%%%%%%%%%%%%%%%%%%%%%%%%%%%%%%%%%%%%%%%%%%%%%
%%%             Preamble's End                   %%%
%%%%%%%%%%%%%%%%%%%%%%%%%%%%%%%%%%%%%%%%%%%%%%%%%%%%      


%%%%%%%%%%%%%%%%%%%%%%%%%      
%\huberlintitlepage

\iftoggle{toc}{
 \frame{
 \begin{multicols}{2}
 	\frametitle{Inhaltsverzeichnis}\tableofcontents
 	%[pausesections]
 \end{multicols}
 	}
}
%%%%%%%%%%%%%%%%%%%%%%%%%%%%%%%%%%%%%%%%%%%%%%%%%%%%%%%%%%
%%%%%%%%%%%%%%%%%%%%%%%%%%%%%%%%%%%%%%%%%%%%%%%%%%%%%%%%%

\section{Organisatorisches}

\subsection{Kontakt}
%\frame{
%\begin{multicols}{2}
%\frametitle{~}
%	\tableofcontents[currentsection]
%\end{multicols}
%}
%
%%%%%%%%%%%%%%%%%%%%%%%%%%%%%%%%%%%%%%%%%%%%%%%%%%%%

\begin{frame}{Kontakt}

\begin{itemize}
	\item \textbf{Büro:} Dorotheenstraße 24, Raum: 3.345
	\item \textbf{Telefon:} (030)2093-9631
	\item \textbf{Webseite:} \url{https://hpsg.hu-berlin.de/~stefan/}
	\item \textbf{E-Mail:} \href{mailto:St.Mueller@hu-berlin.de}{St.Mueller@hu-berlin.de}
	\item[]
	\item \textbf{Sprechstunde}: Mo. 14:00--15:00h (bitte Anmeldung über Sekretariat)
\end{itemize}	

\end{frame}

%%%%%%%%%%%%%%%%%%%%%%%%%%%%%%%%%%%%%%%%%%%%%%%%%%%%%%%%%%
%%%%%%%%%%%%%%%%%%%%%%%%%%%%%%%%%%%%%%%%%%%%%%%%%%%%%%%%%%%
%
\subsection{Sekretariat}
%% \frame{
%% \begin{multicols}{2}
%% \frametitle{~}
%% 	\tableofcontents[currentsection]
%% \end{multicols}
%% }
%%%%%%%%%%%%%%%%%%%%%%%%%%%%%%%%%%%%%%%%%%%%%%%%%%%%%%%%%%%%%%

\begin{frame}{Sekretariat}
	
\begin{itemize}
	\item[] \textbf{Anina Klein}	
	\item \textbf{Büro:} Dorotheenstraße 24, Raum: 3.306
	\item \textbf{Telefon:} (030)2093-9639
	\item \textbf{E-Mail:} \href{mailto:Anina.Klein@cms.hu-berlin.de}{Anina.Klein@cms.hu-berlin.de}
\end{itemize}	

\end{frame}


\subsection{AustauschstudentInnen}

\frame{
\frametitle{AustauschstudentInnen}

Es gibt einen speziellen Kurs für Studierende aus dem Ausland, die nicht regulär in unserem BA studieren.

There is a special course for exchange students.



}

%%%%%%%%%%%%%%%%%%%%%%%%%%%%%%%%%%%%%%%%%%%%%%%%%%%%%%%%%%%%%%
%%%%%%%%%%%%%%%%%%%%%%%%%%%%%%%%%%%%%%%%%%%%%%%%%%%%%%%%%%%%%%
%
\subsection{Moodle}	
%\frame{
%\begin{multicols}{2}
%\frametitle{~}
%	\tableofcontents[currentsection]
%\end{multicols}
%}
%%%%%%%%%%%%%%%%%%%%%%%%%%%%%%%%%%%%%%%%%%%%%%%%%%%%%%%%%%%%

\begin{frame}{Moodle}

\begin{itemize}
	\item Folien und Materialien sind alle auf Moodle.
	\item[]
	\item wichtige Hinweise (Ausfälle, etc \dots) immer über Moodle
	\item[]
	\item \textbf{Moodleseite des Kurses:} \url{https://moodle.hu-berlin.de/course/view.php?id=84039}\\
	\textbf{Moodleschlüssel:} %Mueller_GK_18/19
\end{itemize}		

\end{frame}

%%%%%%%%%%%%%%%%%%%%%%%%%%%%%%%%%%%%%%%%%%%%%%%%%%%%%%%%%%%%%
%%%%%%%%%%%%%%%%%%%%%%%%%%%%%%%%%%%%%%%%%%%%%%%%%%%%%%%%%%%%%
%
\subsection{Tutorien}
%\frame{
%\begin{multicols}{2}
%\frametitle{~}
%	\tableofcontents[currentsection]
%\end{multicols}
%}
%%%%%%%%%%%%%%%%%%%%%%%%%%%%%%%%%%%%%%%%%%%%%%%%%%%%%%%%%%%%%

\begin{frame}{Tutorien}

	\begin{itemize}
		\item \textbf{Online-Tutorium Linguistik} \ras\\
                Fragen mit automatischer Korrektur (über Moodle)!\\
%		\url{http://moodle.hu-berlin.de/course/view.php?id=38846}\\
%		Moodleschlüssel: tutonline
		\item \textbf{Präsenztutorium}
		
		\begin{itemize}
			\item Termine siehe Moodle %Fr. 12--14 wöch. | SO 22, 0.01 | Mareike Lisker
			\item Die Tutorien fangen erst in der zweiten Woche an!
		\end{itemize}
		
	\end{itemize}
	
\end{frame}


%%%%%%%%%%%%%%%%%%%%%%%%%%%%%%%%%%%%%%%%%%%%%%%%%%%%%%%%%%%%
%%%%%%%%%%%%%%%%%%%%%%%%%%%%%%%%%%%%%%%%%%%%%%%%%%%%%%%%%%%
%
\subsection{Zu erbringende Leistungen}
%\frame{
%\begin{multicols}{2}
%\frametitle{~}
%	\tableofcontents[currentsection]
%\end{multicols}
%}
%%%%%%%%%%%%%%%%%%%%%%%%%%%%%%%%%%%%%%%%%%%%%%%%%%%%%%%%%%%

\begin{frame}{Zu erbringende Leistungen}

	\begin{itemize}
	\item Regelmäßige und \textbf{aktive!} Teilnahme (45 h)
        \item Vor- und Nachbereitung 105 h (17 * 6 h 11 min)
	\item Abgabe von 9 der 11 Übungsaufgaben ist Voraussetzung für die MAP-Zulassung
        \item Eine Teilaufgabe muss man erklären.

        \item eine Kurzklausur, die bestanden werden muss


% Vormachen freiwillig.

% Kurzklausur muss bestanden werden.

% Zusatzpunkte 

% Antonio: Mündliche Beteiligung 10 Punkte 

              %Jede/r muss mindestens zweimal eine Lösung vorstellen. Auswahl per Zufall.

	\item Modulabschlussprüfung \ras GK Linguistik + Ü Deutsche Grammatik

              Achtung: Klausurergebnis taucht bei allen auf dem Zeugnis auf

        \item Klausurtermin: 17.02.2019 12:00--14:00
        \item Teilnahmescheine Klausur beilegen, Unterschrift bei Klausurkorrektur
              
	\end{itemize}
	
\end{frame}


%%%%%%%%%%%%%%%%%%%%%%%%%%%%%%%%%%%%%%%%%%%%%%%%%%%%%%%%%%%%%
%%%%%%%%%%%%%%%%%%%%%%%%%%%%%%%%%%%%%%%%%%%%%%%%%%%%%%%%%%%%%
%
\subsection{Literatur}
%\frame{
%\begin{multicols}{2}
%\frametitle{~}
%	\tableofcontents[currentsection]
%\end{multicols}
%}
%%%%%%%%%%%%%%%%%%%%%%%%%%%%%%%%%%%%%%%%%%%%%%%%%%%%%%%%%%%%%

\begin{frame}{Literatur}

\begin{itemize}
	\item Für jede Sitzung wird die Literatur im Semesterplan vorausgesetzt.\\
              (s.\ Handout bzw.\ Semesterplan in Moodle) 
	\item Die Lektüre für jede Sitzung wird als PDF über Moodle bereitgestellt.
\bigskip

	\item Dieser Kurs basiert hauptsächlich auf \citew{Schaefer2018a}, \citew{Luedeling2009a}, \citew{Meibauer&Co07a} und \citew{Abramowski2016a}.
\end{itemize}		

\end{frame}

\frame{
\frametitle{Beschwerden, Verbesserungsvorschläge}


      \begin{itemize}
      \item mündlich
      \item per Mail oder 
      \item anonym über das Web:\\
            \url{http://hpsg.hu-berlin.de/~stefan/Lehre/}
      \end{itemize}

Bitte unbedingt Mail-Regeln beachten!\\
\url{http://hpsg.hu-berlin.de/~stefan/Lehre/mailregeln.html}
\begin{itemize}
\item HU-Mail-Adresse verwenden
\item Vor- und Nachname richtig angeben (nicht Maier, Kim sondern Kim Meier)\\
  Also: ``Kim Meier <Kim.Meier@student.hu-berlin.de>''% sollte Ihr Absender sein.
\item Einfacher für mich $\to$ schnellere Antwort für Sie  
\end{itemize}


}
