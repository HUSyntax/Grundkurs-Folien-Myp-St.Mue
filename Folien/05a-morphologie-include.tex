%%%%%%%%%%%%%%%%%%%%%%%%%%%%%%%%%%%%%%%%%%%%%%%%
%% Compile the master file!
%% 		Include: Antonio Machicao y Priemer
%% 		Course: GK Linguistik
%%%%%%%%%%%%%%%%%%%%%%%%%%%%%%%%%%%%%%%%%%%%%%%%


%%%%%%%%%%%%%%%%%%%%%%%%%%%%%%%%%%%%%%%%%%%%%%%%%%%%
%%%             Metadata                         
%%%%%%%%%%%%%%%%%%%%%%%%%%%%%%%%%%%%%%%%%%%%%%%%%%%%      

\title{Grundkurs Linguistik}

\subtitle{Morphologie I: Einführung \& Begriffe}

\author[A. Machicao y Priemer, S. Müller, A. Lüdeling]{
	{\small Antonio Machicao y Priemer}
	\\
	{\footnotesize \url{http://www.linguistik.hu-berlin.de/staff/amyp}}
	%	\\
	%	{\small\href{mailto:mapriema@hu-berlin.de}{mapriema@hu-berlin.de}}
}

\institute{Institut für deutsche Sprache und Linguistik}

% bitte lassen, sonst kann man nicht sehen, von wann die PDF-Datei ist.
%\date{ }

%\publishers{\textbf{6. linguistischer Methodenworkshop \\ Humboldt-Universität zu Berlin}}

%\hyphenation{nobreak}


%%%%%%%%%%%%%%%%%%%%%%%%%%%%%%%%%%%%%%%%%%%%%%%%%%%%
%%%             Preamble's End                  
%%%%%%%%%%%%%%%%%%%%%%%%%%%%%%%%%%%%%%%%%%%%%%%%%%%%      


%%%%%%%%%%%%%%%%%%%%%%%%%
\huberlintitlepage[22pt]
\iftoggle{toc}{
	\frame{
		\begin{multicols}{2}
			\frametitle{Inhaltsverzeichnis}
			\tableofcontents
			%[pausesections]
		\end{multicols}
	}
}

%%%%%%%%%%%%%%%%%%%%%%%%%%%%%%%%%%
%%%%%%%%%%%%%%%%%%%%%%%%%%%%%%%%%%
%%%%%LITERATURE:

%% Allgemein
\nocite{Glueck&Roedel16a}
\nocite{Schierholz&Co18}
\nocite{Luedeling2009a}
\nocite{Meibauer&Co07a} 
\nocite{Repp&Co15a} 

%%% Sprache & Sprachwissenschaft
%\nocite{Fries16c} %Adäquatheit
%\nocite{Fries16a} %Grammatikalität
%\nocite{Fries&MyP16c} %GG
%\nocite{Fries&MyP16b} %Akzeptabilität
%\nocite{Fries&MyP16d} %Kompetenz vs. Performanz

%%% Phonetik & Phonologie
%\nocite{Altmann&Co07a}
%\nocite{DudenAussprache00a}
%\nocite{Hall00a} 
%\nocite{Kohler99a}
%\nocite{Krech&Co09a}
%\nocite{Pompino95a}
%\nocite{Ramers08a}
%\nocite{Ramers&Vater92a}
%\nocite{Rues&Co07a}
%\nocite{WieseR96a}
%\nocite{WieseR11a}

%%% Graphematik
%\nocite{Altmann&Co07a}
%\nocite{Duerscheid04a}
%\nocite{Eisenberg00a}
%\nocite{Fuhrhop08a}
%\nocite{Fuhrhop09a}
%\nocite{Fuhrhop&Co13a}

%% Morphologie
\nocite{Bauer00a} %Word
\nocite{Eisenberg00a}
\nocite{Fleischer00a} %Wortbildungsprozesse
\nocite{Fleischer&Barz12a} %Einführung Morphologie
\nocite{Haspelmath2002a}
\nocite{Plungian00a} %Morphologie im Sprachsystem
\nocite{Salmon00a} %Term Morphology
\nocite{Wurzel00a} %Gegenstand Morphologie
\nocite{Wurzel00b} %Wort



%%%%%%%%%%%%%%%%%%%%%%%%%%%%%%%%%%%
%%%%%%%%%%%%%%%%%%%%%%%%%%%%%%%%%%%
\section{Morphologie I}
%%%%%%%%%%%%%%%%%%%%%%%%%%%%%%%%%%%

\begin{frame}
\frametitle{Begleitlektüre}

\begin{itemize}
	\item \citet[35--40]{Abramowski2016} 
	\item Lüdeling
\end{itemize}

\end{frame}


%%%%%%%%%%%%%%%%%%%%%%%%%%%%%%%%%%
%%%%%%%%%%%%%%%%%%%%%%%%%%%%%%%%%%
\subsection{Einführung}
\iftoggle{toc}{
\frame{
\begin{multicols}{2}
\frametitle{~}
	\tableofcontents[currentsection]
\end{multicols}
}
}


%%%%%%%%%%%%%%%%%%%%%%%%%%%%%%%%%%
\begin{frame}
\frametitle{Einführung}

\begin{itemize}
	\item Morphologie: \textbf{Formenlehre} in der Biologie \citep[vgl.][]{Salmon00a, Wurzel00a}\\
	(griech. \emph{morphe}: \gq{Form, Gestalt}; \emph{logos} \gq{Sinn, Lehre}

	\item Goethe (1796): Bezeichnung der \textbf{Lehre von Form und Struktur lebender Organismen}.
	
	\item August Schleicher (1859): Übernahme in die Sprachwissenschaft zur Bezeichnung von \textbf{Wörtern als Untersuchungsgegenstand}

\end{itemize}

\pause 

\begin{block}{Morphologie}
	Linguistische Disziplin, die sich mit der \textbf{Struktur} und dem \textbf{Aufbau} von \textbf{Wörtern} und mit \textbf{Theorien} von komplexen Wörtern (Produktivität, Schnittstellen zu Phonologie, Syntax, Semantik) befasst.
\end{block}

\settowidth\jamwidth{(Luedeling, 2009)} 
\ea Brunnenkressesüppchens \jambox{\citep{Luedeling2009a}}

\pause 

{[[[[Brunnen $+$ kresse] $+$ süpp] $+$ -chen] $+$ s]}
\z 

\end{frame}


%%%%%%%%%%%%%%%%%%%%%%%%%%%%%%%%%%
\begin{frame}
\frametitle{Unterteilung der Morphologie}

\begin{itemize}
	\item Morphologie unterteilt sich in:
	
	\begin{itemize}
		\item \textbf{Wort}bildung: Ableitung und Zusammensetzung lexikalischer Wörter (Lemmata)
		
		\ea {[[[Brunnen $+$ kresse] $+$ süpp] $+$ -chen]}		
		\z 
		
		\item \textbf{Wortformen}bildung (\textbf{Flexion}): grammatische Wortformveränderungen
		
		\ea {[Brunnenkressesüppchen] $+$ s}
		\z 
	\end{itemize}
\end{itemize}

\begin{figure}	
\centering
\scalebox{0.6}{
\begin{forest} %% how to scale forest?
%sm edges,
	[\textbf{Morphologie}
		[\alertblue{Flexion}
			[\alertred{Konjugation}] 
			[\alertred{Deklination}]]
		[\alertblue{Wortbildung}
			[\alertred{Komposition}
				[Determinativ]
				[Kopulativ]
				[\dots]]
			[\alertred{Derivation}
				[Suffigierung]
				[Präfigierung]
				[\dots]]
			[\alertred{Konversion}
				[morphologisch]
				[syntaktisch]]
			[\alertred{Rückbildung}]
			[\dots]]]										
\end{forest}}
\end{figure}


\end{frame}


%%%%%%%%%%%%%%%%%%%%%%%%%%%%%%%%%%
%%%%%%%%%%%%%%%%%%%%%%%%%%%%%%%%%%
\subsection{Wortbegriff}
\iftoggle{toc}{
\frame{
\begin{multicols}{2}
\frametitle{~}
	\tableofcontents[currentsection]
\end{multicols}
}
}

%%%%%%%%%%%%%%%%%%%%%%%%%%%%%%%%%%
\begin{frame}
\frametitle{Wortbegriff}

\begin{block}{Wort}
\textbf{Intuitiv} vorgegebener und \textbf{umgangssprachlich} verwendeter
Begriff für \textbf{sprachliche Grundeinheiten}. Seine Definition ist
\textbf{uneinheitlich} und \textbf{kontrovers}. \citep[vgl.][]{Bussmann2002a, Glueck&Roedel16a}
\end{block}

\pause 

\begin{itemize}
	\item Wörter werden auf verschiedenen Ebenen unterschiedlich definiert.
	
	\begin{itemize}
		\item phonetisch-phonologisches Wort
		\item orthographisch-graphemisches Wort
		\item morphologisches Wort
		\item flektivisches Wort
		\item leixikalisch-semantisches Wort
		\item syntaktisches Wort
	\end{itemize}

\item Je nach Ebene gibt es eine \textbf{unterschiedliche Menge} von \gqq{Wörtern}.
\end{itemize}

\end{frame}


%%%%%%%%%%%%%%%%%%%%%%%%%%%%%%%%%%
%%%%%%%%%%%%%%%%%%%%%%%%%%%%%%%%%%
\subsubsection{Phonetisch-phonologisches Wort}
%\frame{
%\frametitle{~}
%	\tableofcontents[currentsection]
%}


%%%%%%%%%%%%%%%%%%%%%%%%%%%%%%%%%%
\begin{frame}
\frametitle{Phonetisch-phonologisches Wort}

\begin{itemize}
	\item kleinsten durch \textbf{Wortakzent} und \textbf{Grenzsignale} (Pause, Knacklaut) theoretisch isolierbare Lautsegmente
	\item Es stimmt nicht immer mit dem orthographisch-graphemischen Wort überein.

\pause 
	
	\item Viele \textbf{phonologische Prozesse} haben das phonologische Wort als Domäne:
	
	\begin{itemize}
		\item Die \textbf{Silbifizierung} erfolgt nur innerhalb des phonologischen Wortes.
		
		\ea kindlich \vs kindisch \ab{-lich} ist ein phonolog. Wort, aber \ab{-isch} nicht.
		\z 

\pause 

		\item \textbf{Assimilationsprozesse} sind nur innerhalb des phonolog. Wortes obligatorisch.
		
		\ea ungern \vs Bearbeitung \ab{un-} ist ein phonolog. Wort
		\z 

\pause 
		
		\item \textbf{Vokalharmonie} (\zB im Türkischen) erfolgt nur innerhalb eines phonolog. Wortes.
	\end{itemize}
\end{itemize}

\end{frame}


%%%%%%%%%%%%%%%%%%%%%%%%%%%%%%%%%%
%%%%%%%%%%%%%%%%%%%%%%%%%%%%%%%%%%
\subsubsection{Orthographisch-graphemisches Wort}
%\frame{
%\frametitle{~}
%	\tableofcontents[currentsection]
%}


%%%%%%%%%%%%%%%%%%%%%%%%%%%%%%%%%%
\begin{frame}
\frametitle{Orthographisch-graphemisches Wort}

\begin{itemize}
	\item Buchstabensequenz zwischen zwei \textbf{Leerzeichen} (Spatien) oder zwischen einem \textbf{Leerzeichen} und einem \textbf{Satzzeichen}

\pause 

	\item Es enthält selbst \textbf{kein Leerzeichen}.
	
	\ea Hör auf! \vs Aufhören!
	\ex New York \vs Berlin
	\z 

	\item Definition ist \textbf{sprachspezifisch}:
	
	\ea Sommerschule \vs summer school
	\ex Morphologieeinführungsbuch \vs introductory morphology book
	\z 
	
\pause 

	\item Seit der letzten großen Rechtschreibreform gibt es im Deutschen \textbf{weniger orth.-
graph. Wörter} (obwohl der \textbf{Wortstatus} dieser Buchstabensequenzen sich nicht verändert hat!)
	
		\ea \ab{radfahren} \ras \ab{Rad fahren} 
		\z 
		
\end{itemize}

\end{frame}


%%%%%%%%%%%%%%%%%%%%%%%%%%%%%%%%%%
\begin{frame}
\frametitle{Orthographisch-graphemisches Wort}

\begin{itemize}
	\item Definition gilt nur für Sprachen mit \textbf{alphabetischem Schriftsystem}.

	\ea 
%\begin{CJK*}{UTF8}{gbsn} % ist jetzt gefixt (SuSE 11.1)
近年来,``应用语言学''作为语言学的一个分支,在国内外都得到了较大的发展,但对于``什么是应用语言学'',
``应用语言学包括哪些研究领域'' 等最基本的问题,学者们却始终没有一个统一的看法。对于一门发展中的、涉及内容广泛的学科而言这是正常的,但长期下去,又会对学科的发展产生不利影响。
%\end{CJK*}
	\z 

\pause 

\begin{itemize}
	\item Chinesische Wörter können aus einem oder mehreren Symbolen bestehen.
	
	\item Texte werden von oben nach unten geschrieben.
	
	\item Auf Computern von links nach rechts.
	
	\item Es gibt \textbf{keine Leerzeichen} zwischen Wörtern.
\end{itemize}

\pause

	\item Es gibt \textbf{Sprachen ohne Schriftsystem}, \dash ohne orth.-graph. Wörter.
\end{itemize}

\end{frame}


%%%%%%%%%%%%%%%%%%%%%%%%%%%%%%%%%%
%%%%%%%%%%%%%%%%%%%%%%%%%%%%%%%%%%
\subsubsection{Morphologisches Wort}
%\frame{
%\frametitle{~}
%	\tableofcontents[currentsection]
%}


%%%%%%%%%%%%%%%%%%%%%%%%%%%%%%%%%%
\begin{frame}
\frametitle{Morphologisches Wort}

\begin{itemize}
	\item \textbf{strukturell stabile} (und \textbf{nicht trennbare}) Grundeinheit eines grammatischen \textbf{Paradigmas} (auch \textbf{lexikalisches Wort} oder \textbf{Lexem} genannt)

	\ea \ab{schreiben}: schreibe, schreibst, schrieb, geschrieben, \ldots 
	\z 

\pause 
	
	\item Sie können \textbf{morphologisch einfach} oder \textbf{komplex} sein.
	
	\ea Tisch, Tischbein, Hals-Nasen-Ohren-Arzt
	\z 

\pause 
	
	\item Komplexe morph. Wörter sind durch spezifische \textbf{Regeln der Wortbildung} zu beschreiben.
	
	\ea Tischbein $=$ Tisch $+$ Bein (Komposition)
	\z 

\pause 

	\item Nichttrennbarkeitskriterium ist problematisch für Partikelverben:
	
	\ea umfahren, mitkommen, anrufen
	\z 
		
	\item im Lexikon kodifiziert (Basiseinheit des Lexikons)
\end{itemize}

\end{frame}


%%%%%%%%%%%%%%%%%%%%%%%%%%%%%%%%%%
\begin{frame}
\frametitle{Morphologisches \vs flektivisches Wort}

\begin{itemize}
	\item Das \textbf{morphologische Wort} sollte von dem \textbf{flektivischen Wort} (Wortform) unterschieden werden.
	
	\item Das \textbf{morphologische Wort} ist die Grundeinheit eines Paradigmas.
\end{itemize}

	\begin{block}{Paradigma}
		alle vorkommenden Wortformen eines Lexems
	\end{block}
	
%	\item 
	Die \textbf{flektivischen Wörter} sind die \textbf{verschiedenen Realisierungen} eines \textbf{morphologischen Wortes}. Sie sind \textbf{hinsichtlich grammatischer Kategorien} wie
	Tempus, Person, Numerus, Kasus, \dots\ spezifiziert.
	
	\ea flektivische Wörter von \emph{Bank} (\gq{Geldinstitut}): Bank, Banken
	\ex flektivische Wörter von \emph{Bank} (\gq{Sitzgelegenheit}): Bank, Bänke, Bänken
	\ex flektivische Wörter von \emph{kaufen}: kaufe, kaufte, (gekauft), kaufest, \dots   
	\z

\end{frame}


%%%%%%%%%%%%%%%%%%%%%%%%%%%%%%%%%%
\begin{frame}
\frametitle{Morphologisches \vs flektivisches Wort}

\begin{itemize}
	\item Die morphologischen Wörter (\textbf{Lexeme}) sind die lexikalischen Einheiten der Sprache.
	
	\item Um auf Lexeme zu referieren verwendet man häufig eine \textbf{Zitierform}.
\end{itemize}
	
	\begin{block}{Zitierform (auch Lemma)}
		\textbf{konventionell festgelegte} Form eines Paradigmas, die stellvertretend für das gesamte Paradigma steht.
		
		Im Deutschen bei \textbf{Nomina} \ras Nominativ Singular
		
		Im Deutschen bei \textbf{Verben} \ras Infinitiv
		
	\end{block} 

\begin{itemize}
	\item Zitierform von Verben ist eine \textbf{komplexe Wortform}:
	
	\ea
	\gll lach- {} -en\\
	{\footnotesize morph. Wort (Imperativform)} $+$ {\footnotesize gebundenes Morphem}\\
	
	\z 
\end{itemize}
\end{frame}


%%%%%%%%%%%%%%%%%%%%%%%%%%%%%%%%%
%%%%%%%%%%%%%%%%%%%%%%%%%%%%%%%%%
\subsubsection{Syntaktisches Wort}
%\frame{
%\frametitle{~}
%	\tableofcontents[currentsection]
%}


%%%%%%%%%%%%%%%%%%%%%%%%%%%%%%%%%%
\begin{frame}
\frametitle{Syntaktisches Wort}

\begin{itemize}
	\item die kleinste verschiebbare und ersetzbare Einheit eines Satzes
(Problem: Artikel, manche Präpositionen)
	
	  \eal
          \ex[]{
            Wir bauten Häuser.
            }
	    \ex[]{
              Häuser bauten wir.
            }
	  \ex[*]{
            Ein bauten wir Haus.
          }
          \zl

	\item Auch definiert als die kleinste Einheit, die alleine als Satz
vorkommen kann.

\eal
\ex Heißt es \gqq{ein} oder \gqq{eine} Hund?
\ex \gqq{Ein}
\zl
		 
\end{itemize}


\end{frame}


%%%%%%%%%%%%%%%%%%%%%%%%%%%%%%%%%%
%%%%%%%%%%%%%%%%%%%%%%%%%%%%%%%%%%
\subsubsection{Wort: lexikalisch-semantisch}
%\frame{
%\frametitle{~}
%	\tableofcontents[currentsection]
%}


%%%%%%%%%%%%%%%%%%%%%%%%%%%%%%%%%%
\begin{frame}
\frametitle{Wort: lexikalisch-semantisch}

\begin{itemize}
	\item \textbf{die kleinste Einheit},
	
	\begin{itemize}
		\item[]
		\item der eine Bedeutung zugeordnet werden kann (\emph{Tisch}) oder
		\item[]		
		\item die eine syntaktische/pragmatische Funktion hat (\emph{der}, \emph{ja})
		\item[]
		\item Problem: \emph{der US-amerikanische Präsident}
	\end{itemize}
\end{itemize}


\end{frame}


%%%%%%%%%%%%%%%%%%%%%%%%%%%%%%%%%%
%%%%%%%%%%%%%%%%%%%%%%%%%%%%%%%%%%
\subsubsection{Wort: Hauptkriterien}
%\frame{
%\frametitle{~}
%	\tableofcontents[currentsection]
%}


%%%%%%%%%%%%%%%%%%%%%%%%%%%%%%%%%%
\begin{frame}
\frametitle{Wort: Hauptkriterien}

\begin{itemize}
	\item akustische und semantische Identität,
	\item morphologische Stabilität und
	\item syntaktische Mobilität
	\item[]
	\item Jede unterschiedliche Wortdefinition liefert eine unterschiedliche Menge von \gqq{Wörtern}, mit denen in den verschiedenen Teilgebieten der Linguistik gearbeitet wird.
	
	\begin{itemize}
		\item Morphologie \ras \gqq{morphologische und flektivische Wörter}
	\end{itemize}
\end{itemize}


\end{frame}


%%%%%%%%%%%%%%%%%%%%%%%%%%%%%%%%%%
%%%%%%%%%%%%%%%%%%%%%%%%%%%%%%%%%%
\subsection{Morphologische Grundbegriffe}
\iftoggle{toc}{
\frame{
\begin{multicols}{2}
\frametitle{~}
	\tableofcontents[currentsection]
\end{multicols}
}
}
%%%%%%%%%%%%%%%%%%%%%%%%%%%%%%%%%%
%%%%%%%%%%%%%%%%%%%%%%%%%%%%%%%%%%
\subsubsection{Morph, Morphem, Allomorph}
%\frame{
%\frametitle{Morphologische Grundbegriffe}
%	\tableofcontents[currentsection]
%}


%%%%%%%%%%%%%%%%%%%%%%%%%%%%%%%%%%
\begin{frame}
\frametitle{Morph, Morphem, Allomorph}

\begin{itemize}
	\item \textbf{Morphem}:

	\begin{itemize}
		\item[]
		\item Strukturalistische Definition: \\
		kleinste bedeutungstragende Einheit
		\item[]
		\item Wurzel 1984: \\
		Ein Morphem ist die kleinste, in ihren \textbf{verschiedenen Vorkommen} als formal \textbf{einheitlich identifizierbare Folge von Segmenten}, der (wenigstens) eine als einheitlich identifizierbare \textbf{außerphonologische Eigenschaft} zugeordnet ist.
	\end{itemize}
\end{itemize}


\end{frame}



%%%%%%%%%%%%%%%%%%%%%%%%%%%%%%%%%%
\begin{frame}
\frametitle{Morph, Morphem, Allomorph}

\begin{itemize}
	\item \textbf{Morphem}:
	
	\begin{itemize}
		\item[]
		\item Außerphonologische Eigenschaften: grammatische (\zB Kasus, Numerus) und/ oder lexikalische Bedeutung
		
		  \eal
                  \ex	Tisches = Tisch + 
                  es = Bed. \gq{TISCH} + Bed./Kat. \gq{GEN.SG}
		  \ex	Haustüren = Haus + tür + en = Bed. \gq{HAUS} + Bed. \gq{TÜR} + Bed./Kat. \gq{PL} 
		  \ex	(sie) essen = ess + en = Bed. \gq{ESS} +
                  Bed./Kat. \gq{3.P.PL}
                  \zl
			 
	\end{itemize}
\end{itemize}


\end{frame}



%%%%%%%%%%%%%%%%%%%%%%%%%%%%%%%%%%
\begin{frame}
\frametitle{Morph, Morphem, Allomorph}

\begin{itemize}
	\item \textbf{Morphem vs. Morph vs. Allomorph:}
	
	\begin{itemize}
		\item[]
		\item Verschiedene Vorkommen: Unterschiedliche Formen (\textbf{Morphe}) können dieselbe Funktion/Bedeutung tragen.
		
		  \ea
                  Tür + en, Kind + er, Schal + s
           \z
		
		\item \textbf{Allomorphe} \ras Varianten eines Morphems, die dieselbe Bedeutung/Kategorie tragen
		
		\begin{itemize}
			\item[]
			\item \{\emph{-en}, \emph{-er}, \emph{-s}\} tragen eine einzelne Bedeutung \gq{PLURAL}; sie sind unterschiedliche \textbf{Morphe} und alle \textbf{Allomorphe} zu einem \textbf{Morphem} (abstrakte Einheit).
		\end{itemize}
	\end{itemize}
\end{itemize}


\end{frame}



%%%%%%%%%%%%%%%%%%%%%%%%%%%%%%%%%%
\begin{frame}
\frametitle{Morph, Morphem, Allomorph}

\begin{itemize}
	\item \textbf{phonologisch bedingte Allomorphie:}
	
	\begin{itemize}
		\item[]
		\item Ein Morphem kann verschiedene Allomorphe aufgrund phonologischer Regularitäten haben:
		
		\begin{itemize}
			\item[]
			\item Allomorphe \textipa{[land]} und \textipa{[lant]} \\
			durch Auslautverhärtung in Landes vs. Land
			\item[]
			\item Allomorphe \textipa{[n]} und \textipa{[@n]} für Infinitiv: \\
			durch Schwaeinsetzung: segeln vs. formen, turnen
		\end{itemize}
	\end{itemize}
\end{itemize}


\end{frame}



%%%%%%%%%%%%%%%%%%%%%%%%%%%%%%%%%%
\begin{frame}
\frametitle{Morph, Morphem, Allomorph}

\begin{itemize}
	\item \textbf{morphologisch bedingte Allomorphie:}
	
	\begin{itemize}
		\item Allomorphe \textipa{[haUs]} und \textipa{[hOIs]} in Haus vs. Häuschen, häuslich
		\item Regel: Neutra mit \emph{-er}-Plural und umlautfähigem Stammvokal erhalten immer einen Umlaut (Fässer, Bücher, Hörner).
	\end{itemize}
	
	\item \textbf{lexikalisch bedingte Allomorphie:}
	
	\begin{itemize}
		\item Allomorphe \textipa{[kUs]} und \textipa{[kYs]} in Kuss vs. Küsse (auch: Küsschen) im Lexikon festgelegt: Maskulina mit der Pluralendung \emph{-e} erhalten manchmal einen Umlaut und manchmal nicht (Tage)
	\end{itemize}
	
	\item[]
	\item Häufig verwendet man den Begriff \textbf{morphologisch bedingte Allomorphie} auch für die \textbf{lexikalisch bedingte Allomorphie}.
\end{itemize}


\end{frame}



%%%%%%%%%%%%%%%%%%%%%%%%%%%%%%%%%%
\begin{frame}
\frametitle{Morph, Morphem, Allomorph}

\begin{itemize}
	\item Morpheme (sowie Phoneme) findet man mithilfe von \textbf{Minimalpaaren}:
\end{itemize}

\begin{table}
\begin{tabular}{l | l}
lach + t  & träum + t \\
lach + st & träum + st \\
lach + en & träum + en \\
lach + te & träum + te \\
\end{tabular}
\end{table}

\end{frame}


%%%%%%%%%%%%%%%%%%%%%%%%%%%%%%%%%%
%%%%%%%%%%%%%%%%%%%%%%%%%%%%%%%%%%
\subsubsection{Morphemklassifikation}
%\frame{
%\frametitle{~}
%	\tableofcontents[currentsection]
%}


%%%%%%%%%%%%%%%%%%%%%%%%%%%%%%%%%%
\begin{frame}
\frametitle{Morphemklassifikation}

\begin{itemize}
	\item \textbf{Morpheme lassen sich hinsichtlich verschiedener Kriterien klassifizieren:}
	
	\begin{itemize}
	 \item[]
	 \item Verhältnis Form und Bedeutung
	 \item[]
	 \item Art der Bedeutung
	 \item[]
	 \item Distribution und Selbstständigkeit
	\end{itemize}
\end{itemize}


\end{frame}



%%%%%%%%%%%%%%%%%%%%%%%%%%%%%%%%%%
\begin{frame}
\frametitle{Form \& Bedeutung}

\begin{itemize}
	\item \textbf{Wodurch unterscheiden sich die unterstrichenen Morpheme?}
		
	  \only<1->{
            \ea
            Helga ist die schön\underline{st}e.
          \z
          }
	
	\only<2>{eine Form - eine Bedeutung: \\
		Form: \emph{-st} \\
		gramm. Funktion: Superlativ\\
		\textbf{= strukturalistischer Idealfall}}
	
	\only<1->{\ea
          Karl \underline{gab} Ilse die Hauptrolle.
          \z}
	
	\only<3>{eine Form - Komplex mehrerer Bedeutungen \\
		Form: \emph{gab} \\
		Bedeutung: \gq{GEBEN} + \gq{3.P.SG.PRÄT.IND.AKTIV} \\
		\textbf{= Portmanteau-Morphem} \\
		Die Verschmelzung zweier Morpheme wird manchmal auch Portmanteau-Morphem genannt: \ab{zum, am, im}}
	
	\only<1->{\ea Paul hat Ilse wirklich \underline{ge}lieb\underline{t}!
        \z}	
		
	\only<4>{zwei Formen - eine Bedeutung (gramm. Funktion) \\
		Form: \emph{ge-} + \emph{-t} \\
		gramm. Funktion: \gq{Partizip II} \\
		\textbf{= diskontinuierliches Morphem}}
	
\end{itemize}


\end{frame}


%%%%%%%%%%%%%%%%%%%%%%%%%%%%%%%%%%
\begin{frame}
\frametitle{Bedeutungsart}

\begin{itemize}
	\item \textbf{Wodurch unterscheiden sich die unterstrichenen Morpheme?}
	
		\only<1->{\ea Paul \underline{geht} mit Lisa ins \underline{Kino}.\z}
				
		\only<2>{Morpheme bezeichnen Außersprachliches (Objekte, Sachverhalte). \\
		Inhalt ist Gegenstand semantischer/lexikologischer Analyse. \\
		Ihre Klasse ist erweiterbar. \\
		\textbf{= lexikalische Morpheme (offene Klasse)}}		
		
		\only<1->{\ea
                  Karl spiel\underline{t} in der Küche den Held\underline{en}, \underline{dass} es
                  einen graust.
                \z}
		
		\only<3>{Morpheme kodieren grammatische Information, dienen der Realisierung grammatischer Beziehungen im Satz \\
		\textbf{= grammatische Morpheme (geschlossene Klasse)} \\
		Umstritten: Wortbildungsmorpheme wie \emph{-lich}, \emph{-heit}; sog. Funktionswörter wie Präpositionen, Konjunktionen, etc.}
		
	
\end{itemize}


\end{frame}


%%%%%%%%%%%%%%%%%%%%%%%%%%%%%%%%%%
\begin{frame}
\frametitle{Distribution/Selbstständigkeit}

\begin{itemize}
	\item \textbf{Wodurch unterscheiden sich die unterstrichenen Morpheme?}

	  \only<1->{\ea
            \underline{Und} Paul sieht \underline{rot}, \underline{weil} Lisa \underline{sehr}
            \underline{schnell} \underline{mit} Peter verschwand.
          \z}
		
		\only<2>{Morpheme kommen frei vor; können sowohl lexikalische als auch grammatische Bedeutung haben \\
		\textbf{= freie Morpheme}}
		
		\only<1->{\ea
                  Sprachwissen\underline{schaft} kann auch sehr
                  \underline{un}übersicht\underline{lich} sein.}
                \z
		
		\only<3>{Morpheme sind an andere Morpheme gebunden; treten nicht selbstständig auf (sie sind nicht \gqq{wortfähig}) \\
		\textbf{= gebundene Morpheme} \\
		Umstritten: die Einordnung bestimmter lexikalischer Morpheme, wie \emph{geb-}, \emph{weiger-}, wenn sie nicht frei vorkommen (meist dient die Wortform des Imperativs als Kriterium).}

	
\end{itemize}


\end{frame}


%%%%%%%%%%%%%%%%%%%%%%%%%%%%%%%%%%
\begin{frame}
\frametitle{Distribution/Selbstständigkeit}

\begin{itemize}
	\item Sonderfall des gebundenen Morphems: \textbf{Unikales Morph(em)} (\emph{cranberry morph})
	
	\begin{itemize}
		\item[]
		  \ea
                  \underline{Brom}beere, \underline{Him}beere, \underline{Schorn}stein,
                  ver\underline{geu}den, Tausend\underline{sassa}
                  \z
		\item[]
		\item nur in einer einzigen Kombination
		\item[]
		\item keine produktiven Morpheme
		\item[]
		\item Bedeutung synchron nicht mehr erschließbar
		\item[]
		\item Bedeutung auf distinktive Funktion beschränkt
	\end{itemize}
\end{itemize}


\end{frame}


%%%%%%%%%%%%%%%%%%%%%%%%%%%%%%%%%%
%%%%%%%%%%%%%%%%%%%%%%%%%%%%%%%%%%
\subsubsection{Wurzel, Stamm, Basis, Simplex}
%\frame{
%\frametitle{~}
%	\tableofcontents[currentsection]
%}


%%%%%%%%%%%%%%%%%%%%%%%%%%%%%%%%%%
\begin{frame}
\frametitle{Wurzel, Stamm, Basis, Simplex}

\begin{itemize}
	\item \textbf{Wurzel:} (Wurzelmorphem, Basismorphem)
	
	\begin{itemize}
		\item[]
		\item Unterste, atomare Basis komplexer Wörter
		\item[]
		\item hinsichtlich \textbf{Wortbildung und Flexion} nicht mehr zerlegbar
		\item[]
		\item oft, aber nicht immer frei
		
		\begin{itemize}
			\item[]			
			\item Wurzel \emph{ehr}: \emph{Ehr-e, Ehr-gefühl, ehr-bar}
			\item Wurzel \emph{ess}: \emph{ess-en, ess-bar}
		\end{itemize}
	\end{itemize}
\end{itemize}


\end{frame}



%%%%%%%%%%%%%%%%%%%%%%%%%%%%%%%%%
\begin{frame}
\frametitle{Wurzel, Stamm, Basis, Simplex}

\begin{itemize}
	\item \textbf{Stamm:}
	
	\begin{itemize}
		\item[]
		\item Ausgangsform der \textbf{Flexion}
		\item[]
		\item kann Wurzel oder komplexe morphologische Einheit sein
		
		\begin{itemize}
			\item[]
			\item \ab{sag} + \emph{-st}
			\item \abu{be-lächel} + \emph{-st}
		\end{itemize}
	\end{itemize}
\end{itemize}

\end{frame}



%%%%%%%%%%%%%%%%%%%%%%%%%%%%%%%%%%
\begin{frame}
\frametitle{Wurzel, Stamm, Basis, Simplex}

\begin{itemize}
	\item \textbf{Basis:} (Pl. Basen)
	
	\begin{itemize}
		\item etwas, woran etwas affigiert werden kann
		\item Ausgangsformen der \textbf{Derivation}
		\item kann selber auch komplex sein
		
		\begin{itemize}
			\item (Basis) \emph{Les} + (Suffix) \emph{ung}
			\item (Präfix) \emph{un} + (Basis) \emph{freundlich}
			\item (Basis) \emph{freund} + (Suffix) \emph{lich}
		\end{itemize}
	\end{itemize}
	\item[]
	\item \textbf{Derivat:} Resultat der \textbf{Derivation}
	
	\begin{itemize}
		\item[]
		\begin{itemize}
			\item Lesung
			\item unfreundlich
			\item freundlich
		\end{itemize}
	\end{itemize}
\end{itemize}


\end{frame}



%%%%%%%%%%%%%%%%%%%%%%%%%%%%%%%%%%
\begin{frame}
\frametitle{Wurzel, Stamm, Basis, Simplex}

\begin{itemize}
	\item \textbf{Simplex}: (Pl. Simplizia)
	
	\begin{itemize}
		\item[]
		\item nicht zusammengesetztes oder abgeleitetes Lexem
		\item kann als Basis für Neubildungen dienen.
		
		\begin{itemize}
			\item[]
			\item geben
			\item in angeben, vergeblich, Zugabe
		\end{itemize}
		
	\end{itemize}
		
		\item[]
		\item Wenn Derivationsaffixe oder Stämme/Wurzeln nicht mehr aktiv (auch nicht mehr produktiv) sind, nimmt man die Form als Simplex wahr.
		
	\begin{itemize}
		\item[]
		\begin{itemize}
			\item Ursache, Mädchen, freilich
		\end{itemize}
	\end{itemize}
\end{itemize}


\end{frame}


%%%%%%%%%%%%%%%%%%%%%%%%%%%%%%%%%%
%%%%%%%%%%%%%%%%%%%%%%%%%%%%%%%%%%
\subsubsection{Affix \& Konfix}
%\frame{
%\frametitle{~}
%	\tableofcontents[currentsection]
%}


%%%%%%%%%%%%%%%%%%%%%%%%%%%%%%%%%%
\begin{frame}
\frametitle{Affix \& Konfix}

\begin{itemize}
	\item \textbf{Affixe}
	
	\begin{itemize}
		\item \textbf{nicht frei vorkommende} \emph{Wort}bildungs- oder \emph{Wortform}bildungselemente
		\item Nach ihrer \textbf{Stellung zum Stamm/Basis} unterscheidet man:
		
		\begin{itemize}
			\item Präfix: \\
			\underline{un}-schön, \underline{ver}-teilen
			\item Suffix: \\
			teil-\underline{bar}, Bäck-\underline{er}
			\item Zirkumfix: \\
			\underline{ge}-sag-\underline{t}, \underline{Ge}-red-\underline{e}
			\item Infix: \\
			Chrau (Vietnam): v\u{o}h \gq{wissen} \ras v\underline{an}\u{o}h \gq{weise} \\
			Tagalog (Philippinen): sulat \gq{schreiben} \ras su\underline{mu}lat \gq{schrieb}
		\end{itemize}
	\end{itemize}
\end{itemize}


\end{frame}


%%%%%%%%%%%%%%%%%%%%%%%%%%%%%%%%%%
\begin{frame}
\frametitle{Affix \& Konfix}

\begin{itemize}
	\item \textbf{Affixe}
	
	\begin{itemize}
		\item[]
		\item Nach ihrer morphologischen Funktion unterscheidet man:
		
		\begin{itemize}
			\item[]
			\item Derivationsaffixe (\emph{Wort}bildungsaffixe): \\
			\emph{-ig, -lich, -keit; ver-, be-, ent-, un-, \dots}
			\item[]
			\item Flexionsaffixe (\emph{Wortformen}bildungsaffixe): \\
			\emph{-st} (kommst), \emph{-(e)n} (gehen, Betten), \emph{-er} (Kinder, kleiner), \dots
		\end{itemize}
	\end{itemize}
\end{itemize}


\end{frame}



%%%%%%%%%%%%%%%%%%%%%%%%%%%%%%%%%%
\begin{frame}
\frametitle{Affix \& Konfix}

\begin{itemize}
	\item \textbf{Konfixe}
	
	\begin{itemize}
		\item nicht frei vorkommende Elemente (ähnlich wie Affixe)
		\item Sie lassen sich zu einem selbständigen Wort kombinieren (wie normale Wurzeln/ Stämme)
		
		\begin{itemize}
			\item \underline{Bio}-loge
			\item Soft-ie
		\end{itemize}
		
		\item[]
		\item stärker lexikalische Grundbedeutung als Affixe, können jedoch als Präfixe oder Suffixe fungieren
		
		\begin{itemize}
			\item kino-phil
			\item Phil-anthrop
			\item Soft-getränk
		\end{itemize}
	\end{itemize}
\end{itemize}


\end{frame}


