%%%%%%%%%%%%%%%%%%%%%%%%%%%%%%%%%%%%%%%%%%%%%%%%
%% Compile the master file!
%% 		Include: Antonio Machicao y Priemer
%% 		Course: GK Linguistik
%%%%%%%%%%%%%%%%%%%%%%%%%%%%%%%%%%%%%%%%%%%%%%%%


%%%%%%%%%%%%%%%%%%%%%%%%%%%%%%%%%%%%%%%%%%%%%%%%%%%%
%%%             Metadata                         
%%%%%%%%%%%%%%%%%%%%%%%%%%%%%%%%%%%%%%%%%%%%%%%%%%%%      

\title{Grundkurs Linguistik}

\subtitle{Morphologie II: Wortbildung \& Komposition}

\author[A. Machicao y Priemer]{
	{\small Antonio Machicao y Priemer}
	\\
	{\footnotesize \url{http://www.linguistik.hu-berlin.de/staff/amyp}}
	%	\\
	%	{\small\href{mailto:mapriema@hu-berlin.de}{mapriema@hu-berlin.de}}
}

\institute{Institut für deutsche Sprache und Linguistik}

% bitte lassen, sonst kann man nicht sehen, von wann die PDF-Datei ist.
%\date{ }

%\publishers{\textbf{6. linguistischer Methodenworkshop \\ Humboldt-Universität zu Berlin}}

%\hyphenation{nobreak}


%%%%%%%%%%%%%%%%%%%%%%%%%%%%%%%%%%%%%%%%%%%%%%%%%%%%
%%%             Preamble's End                  
%%%%%%%%%%%%%%%%%%%%%%%%%%%%%%%%%%%%%%%%%%%%%%%%%%%%      


%%%%%%%%%%%%%%%%%%%%%%%%%
\huberlintitlepage[22pt]
\iftoggle{toc}{
	\frame{
		\begin{multicols}{2}
			\frametitle{Inhaltsverzeichnis}
			\tableofcontents
			%[pausesections]
		\end{multicols}
	}
}


%%%%%%%%%%%%%%%%%%%%%%%%%%%%%%%%%%
%%%%%%%%%%%%%%%%%%%%%%%%%%%%%%%%%%
%%%%%LITERATURE:

%% Allgemein
\nocite{Glueck&Roedel16a}
\nocite{Schierholz&Co18}
\nocite{Luedeling2009a}
\nocite{Meibauer&Co07a} 
\nocite{Repp&Co15a} 

%%% Sprache & Sprachwissenschaft
%\nocite{Fries16c} %Adäquatheit
%\nocite{Fries16a} %Grammatikalität
%\nocite{Fries&MyP16c} %GG
%\nocite{Fries&MyP16b} %Akzeptabilität
%\nocite{Fries&MyP16d} %Kompetenz vs. Performanz

%%% Phonetik & Phonologie
%\nocite{Altmann&Co07a}
%\nocite{DudenAussprache00a}
%\nocite{Hall00a} 
%\nocite{Kohler99a}
%\nocite{Krech&Co09a}
%\nocite{Pompino95a}
%\nocite{Ramers08a}
%\nocite{Ramers&Vater92a}
%\nocite{Rues&Co07a}
%\nocite{WieseR96a}
%\nocite{WieseR11a}

%%% Graphematik
%\nocite{Altmann&Co07a}
%\nocite{Duerscheid04a}
%\nocite{Eisenberg00a}
%\nocite{Fuhrhop08a}
%\nocite{Fuhrhop09a}
%\nocite{Fuhrhop&Co13a}

%% Morphologie
%\nocite{Bauer00a} %Word
\nocite{Eisenberg00a}
\nocite{Fleischer00a} %Wortbildungsprozesse
\nocite{Fleischer&Barz12a} %Einführung Morphologie
\nocite{Fuhrhop96a} %Fugenelemente
\nocite{Fuhrhop00a} %Fugenelemente
\nocite{Fries&MyP16j} %Kopf
\nocite{Grewendorf&Co91a} %Betonung bei Komposita
\nocite{Haspelmath2002a}
\nocite{MyP18b} %Kopf
\nocite{Olsen86a} %Morphologie des Deutschen
\nocite{Olsen14a} %Coordinative Structures
%\nocite{Plungian00a} %Morphologie im Sprachsystem
%\nocite{Salmon00a} %Term Morphology
\nocite{Wegener03b}
\nocite{Wurzel00a} %Gegenstand Morphologie
\nocite{Wurzel00b} %Wort


%%%%%%%%%%%%%%%%%%%%%%%%%%%%%%%%%%%
%%%%%%%%%%%%%%%%%%%%%%%%%%%%%%%%%%%
\section{Morphologie II}
%%%%%%%%%%%%%%%%%%%%%%%%%%%%%%%%%%%

\begin{frame}
\frametitle{Begleitlektüre}

\begin{itemize}
	\item AM S.~41--45
\end{itemize}

\end{frame}

%%%%%%%%%%%%%%%%%%%%%%%%%%%%%%%%%%
%%%%%%%%%%%%%%%%%%%%%%%%%%%%%%%%%%
\subsection{Einführung}
\iftoggle{sectoc}{
	\frame{
		\begin{multicols}{2}
			\frametitle{~}
			\tableofcontents[currentsection]
		\end{multicols}
	}
}


%%%%%%%%%%%%%%%%%%%%%%%%%%%%%%%%%%
\begin{frame}
\frametitle{Einführung}

\begin{itemize}
	\item Wortschatz des Deutschen: \\
300\,000– 500\,000 Wörter und Phraseologismen (fachliche und regionale Wortschätze, veraltete und neue Wörter)
	\item durchschnittlicher \textbf{aktiver Wortschatz}:
10\,000–20\,000 Wörter
	\item Jährlich werden ung.\ 1\,000 neue Wörter in den Duden aufgenommen \\
	(5\,000 im Jahr 2018), davon:
	
	\begin{itemize}
		\item 83\% Wortbildungen (\zB \emph{facebooken}, \emph{gegenchecken}, \emph{entfreunden}, \emph{Späti})
		\item 12\% neue Bedeutung alter Wörter (\zB \emph{runterwürgen}, \emph{verpeilen})
		\item 5\% Entlehnungen (\zB \emph{tindern}, \emph{liken})
		\item außerdem: neue Redewendungen (\zB \emph{Es ist alles im grünen Bereich})
	\end{itemize}
\end{itemize}


\end{frame}


%%%%%%%%%%%%%%%%%%%%%%%%%%%%%%%%%%
\begin{frame}
\frametitle{Einführung (zur Erinnerung)}

Die Morphologie unterteilt man in:
	
	\begin{itemize}
		\item \textbf{Wortbildung:} Ableitung und Zusammensetzung lexikalischer Wörter:
		
		\eal 
			\ex {[anforder(n)] $+$ [-ung] $=$ \alertred{Anforderung}}
			\ex {[les(e)] $+$ [kreis] $=$ \alertred{Lesekreis}}
		\zl
		
		\item \textbf{Wortformenbildung} (Flexion): Bildung von Wortformen in einem Paradigma: 
		
		\begin{itemize}
			\item Deklination der Nomina: 
			
			\ea (der) Lesekreis, (den) Lesekreis, (dem) Lesekreis(e), (des) Lesekreises, \ldots
			\z 
			
			\item Konjugation der Verben: 
			
			\ea fordere, forderst, fordert, fordern, \ldots 
			\z 
		\end{itemize}	
	\end{itemize}	

\end{frame}


%%%%%%%%%%%%%%%%%%%%%%%%%%%%%%%%%%%
\begin{frame}
\frametitle{Einführung (zur Erinnerung)}

\begin{itemize}
	\item \textbf{Wortbildung}
	
	\begin{itemize}
		\item neue lexikalische Wörter
		\item neue lexikalische Bedeutung (Begriffe)
		\item Ausgangswörter können einfache (Simplizia) oder komplexe Lexeme sein.
		\item Änderung der Wortart ist möglich (vgl.\ (\ref{ex:5bWortart1})) aber nicht zwingend (vgl.\ (\ref{ex:5bWortart2})). 
		
		\ea\label{ex:5bWortart1} {[\MyPdown{V}bearbeit] $+$ -ung $=$ [\MyPdown{N}Bearbeitung]}
		\ex\label{ex:5bWortart2} {be- $+$ [\MyPdown{V}arbeit(-en)] $=$ [\MyPdown{V}bearbeit(-en)]}
		\z
	\end{itemize}

	\item \textbf{Flexion}
	
	\begin{itemize}
		\item Flexionsmorpheme sind rechtsperipher: sie werden erst nach der Wortbildung mit dem Stamm verbunden
		\item Flexionsmorpheme enthalten nicht zwingend einen Vokal (nur Schwa):
		
		\ea Wortbildungsaffixe: -ung, -in, -bar, ent-, \ldots 
		\ex Flexionsaffixe: -en, -t, -est, -st, -n, ge- -t, \ldots 
		\z
		
	\end{itemize}
\end{itemize}

\end{frame}


%%%%%%%%%%%%%%%%%%%%%%%%%%%%%%%%%%
%%%%%%%%%%%%%%%%%%%%%%%%%%%%%%%%%%
\subsection{Struktur komplexer Wörter}
\iftoggle{sectoc}{
	\frame{
		\begin{multicols}{2}
			\frametitle{~}
			\tableofcontents[currentsection]
		\end{multicols}
	}
}


%%%%%%%%%%%%%%%%%%%%%%%%%%%%%%%%%%
\begin{frame}
\frametitle{Struktur komplexer Wörter}

\begin{itemize}
		\item Die meisten Wortbildungsprozesse sind \textbf{konkatenativ} (vgl.\ (\ref{ex:5bKonk1}) \vs (\ref{ex:5bKonk2})).
	
	\settowidth\jamwidth{[ nicht konkatenativ]} 
	\ea\label{ex:5bKonk1} {[\MyPdown{N}Tisch] $+$ [\MyPdown{N}bein] \ras [\MyPdown{N}Tischbein]} 
	\jambox{[konkatenativ]}
	
	\ex\label{ex:5bKonk2} {[\MyPdown{V}schlaf] \ras [\MyPdown{N}Schlaf]}
	\jambox{[nicht konkatenativ]}
	\z 
	
\end{itemize}

\begin{block}{Konkatenation (auch Verkettung)}
	Prozess und Ergebnis einer systematischen linearen \textbf{Aneinanderreihung} linguistischer Kategorien	\citep[vgl.][]{Fries16i, Fuhrhop17a} 
\end{block}	


\end{frame}


%%%%%%%%%%%%%%%%%%%%%%%%%%%%%%%%%%
\begin{frame}
\frametitle{Struktur komplexer Wörter}

\begin{minipage}{.65\textwidth}

	\begin{itemize}
		\item Die Wortstruktur spiegelt \textbf{Bildungsprozess} wider.
		\item Sie steuert die \textbf{Interpretation}.
		\item In den meisten Theorien ist die komplexe Struktur \textbf{binär}.
		
		\begin{itemize}
			\item \textbf{Maximal} zwei Elemente ($=$ \textbf{Konstituenten} von engl. \emph{constituent} \gq{Bestandteil}) verbinden sich zu einem komplexen Element.
			\item Zwei Elemente gehören enger zusammen.
		\end{itemize}
		\item Aufbau ist \textbf{hierarchisch} (vgl.\ phonologische Struktur)
	\end{itemize}

\end{minipage}
%
\hfill%
%
\begin{minipage}{.34\textwidth}
\begin{figure}	
\centering
\scalebox{0.7}{
\begin{forest} 
	[Haustürschlüssel
		[Haustür
			[Haus] 
			[Tür]]
		[Schlüssel]]									
\end{forest}}

\vspace{.75cm}

\scalebox{.7}{
\begin{forest}
	[Zugverbindung
		[Zug]
		[Verbindung
			[verbind
				[ver-]
				[bind]]
			[-ung]]]
\end{forest}}
\end{figure}
\end{minipage}

\end{frame}


%%%%%%%%%%%%%%%%%%%%%%%%%%%%%%%%%%
\begin{frame}
\frametitle{Struktur komplexer Wörter}

\begin{minipage}{0.65\textwidth}
\begin{itemize}
	\item Morphologische Einheiten (Stämme und Affixe) sind \textbf{kategoriell ausgezeichnet}, \dash es wird markiert, ob es sich bspw. 
	
	\begin{itemize}
		\item um Nomenstämme (\alertred{N}), Verbstämme (\alertred{V}), \ldots 
		\item[] bzw.
		\item um Nomenaffixe (\alertblue{N\MyPup{af}}), Verbaffixe (\alertblue{V\MyPup{af}}), \ldots\  handelt
	\end{itemize}
\end{itemize}
\end{minipage}
%
\hfill%
%
\begin{minipage}{.34\textwidth}

\begin{figure}	
\centering
\scalebox{0.7}{
\begin{forest} 
sm edges,
	[\alertred{N}
		[\alertred{N}
			[\alertred{N}
				[Haus]]
			[\alertred{N} 
				[Tür]]]
		[\alertred{N}
			[Schlüssel]]]								
\end{forest}}

\vspace{.5cm}

\centering
\scalebox{.7}{
\begin{forest}
sm edges,
	[N
		[N		
			[Zug]]
		[N
			[V				
				[\alertblue{V\MyPup{af}}					
					[ver-]]
				[V
					[bind]]]
			[\alertblue{N\MyPup{af}}			
				[ung]]]]
\end{forest}}
\end{figure}
\end{minipage}
\end{frame}


%%%%%%%%%%%%%%%%%%%%%%%%%%%%%%%%%%
\begin{frame}
\frametitle{Struktur komplexer Wörter}

\begin{minipage}{.66\textwidth}

\begin{itemize}
	
	\item Die einzelnen Elemente haben \textbf{Beschränkungen}, mit welchen anderen Elementen sie sich verbinden können.

\pause 
	
	\item Ein \textbf{Präfix} der Kategorie $X$\MyPup{af} kann sich nur mit Elementen der Kategorie $X$ verbinden.

	\ea[*]{
		[\MyPdown{V\MyPup{af}}ver-] $+$ [\MyPdown{N}bindung] $=$ \alertred{\ding{56}} verbindung
	}
	\ex[]{
		[\MyPdown{V\MyPup{af}}ver-] $+$ [\MyPdown{V}bind] $=$ \alertred{\ding{52}} verbind 
	}
	\z 	

%\pause 
	
	\item Ein \textbf{Suffix} der Kategorie $X$\MyPup{af} bestimmt, mit welchen Elementen (\zB $X$, $Y$, $Z$) es sich verbinden kann.
	
	\ea {\alertred{\ding{52}} [\MyPdown{V}bind]$+$[\MyPdown{N\MyPup{af}}-ung], \alertred{\ding{56}} [\MyPdown{N}Lampe]$+$[\MyPdown{N\MyPup{af}}-ung]}
	\z 
\end{itemize}

\end{minipage}
%
\hfill%
%
\begin{minipage}{.32\textwidth}

\begin{figure}	
\centering
\scalebox{.7}{
\begin{forest}
sm edges,
	[\alertred{*}N
		[N
			[Zug]]
		[\alertred{*}N
			[V\MyPup{af}
				[ver-]]
			[N
				[V
					[bind]]
				[N\MyPup{af}
					[-ung]]]]]
\end{forest}}
\caption{ungrammatische Struktur}
\end{figure}

\end{minipage}
\end{frame}


%%%%%%%%%%%%%%%%%%%%%%%%%%%%%%%%%%
\begin{frame}
\frametitle{Struktur komplexer Wörter: Kopf}


\begin{itemize}
	\item Das \textbf{rechte Element} in (\ref{ex:5bKonk3}) und (\ref{ex:5bKonk4}) bestimmt die \textbf{Kategorie} des Wortbildungsprodukts %,MyP18b}
	
\ea 
	\ea\label{ex:5bKonk3} {[\MyPdown{V}brauch] $+$ [\MyPdown{\alertred{A\MyPup{af}}}-bar] $=$ [\alertred{\MyPdown{A}}brauchbar]}
	\ex\label{ex:5bKonk4} {[\MyPdown{A}brauchbar] $+$ [\MyPdown{\alertred{N\MyPup{af}}}-keit] $=$ [\alertred{\MyPdown{N}}Brauchbarkeit]}
	\z 
\z 

\end{itemize}

\pause 
	
\begin{block}{Kopf}
	Element in einem \textbf{konkatenativen morphosyntaktischen Prozess}, das die \textbf{wichtigsten Eigenschaften} des Resultats bestimmt. 

\end{block}	

\begin{block}{Kopfprinzip}
	\textbf{jedes} komplexe Wort, das durch \textbf{Komposition} oder \textbf{Derivation} entstanden ist, hat einen morphologischen Kopf
\end{block}	

\hfill (vgl.\ \citealp[76]{Olsen86a}; \citealp[]{MyP18b})
	
\end{frame}


%%%%%%%%%%%%%%%%%%%%%%%%%%%%%%%%%%
\begin{frame}
\frametitle{Struktur komplexer Wörter: Kopf}

\begin{minipage}{.7\textwidth}


\begin{itemize}
	\item In der \textbf{Wortbildung} legt der Kopf die \textbf{morphosyntaktischen} Eigenschaften des komplexen Wortes fest (\zB Genus, Wortart, Flexionsart, \ldots ). 
	
	\item Auch einige der \textbf{semantischen} Eigenschaften des Resultats werden vom Kopf bestimmt.
	
	\item \textbf{Projektion:} Vom Kopf werden die Merkmale auf den sog. Mutterknoten übertragen.  
	
	\item Problematische Fälle:
		
		\ea verholz(-en), befreund(-en), beruhig(-en), Wasserablauf
		\z
		
%	\item[] \textbf{ÜB.2}
\end{itemize}


\begin{block}{Righthand Head Rule}
	die \textbf{am weitesten rechts stehende Konstituente} in einem \textbf{Wortbildungsprozess} ist der Kopf.
\end{block}	

\end{minipage}
%
\hfill%
%
\begin{minipage}{.29\textwidth}

\begin{figure}	
	\centering
	\scalebox{.6}{
	\begin{forest}
		[N\\
		{[}\alertred{neut}{]}, name=N3K
		[N\\
		{[}\alertblue{mask}{]}, name=N2K
		[N\\	
		{[}fem{]}, name=N1
		[Kartoffel\\
		]
		]
		[N\\
		{[}\alertblue{mask}{]}, name=N2
		[salat\\
		]
		]	
		]
		[N\\
		{[}\alertred{neut}{]}, name=N3	
		[rezept\\
		]
		]
		]{
			\draw[->,dashed] (N2) to[out=east,in=east] (N2K);
			\draw[->,dashed] (N3) to[out=east,in=east] (N3K);
		}
	\end{forest}
}

\vspace{.5cm}

	\centering
	\scalebox{.6}{
\begin{forest}
	[\alertred{N}, name=N1
	[N\MyPup{af}, name=N0
	[un-]		
	]
	[\alertred{N}, name=A1
	[\alertblue{A}, name=V1 
	[V
	[brauch-]
	]
	[\alertblue{A}\MyPup{af}, name=V0
	[-bar]
	]
	]
	[\alertred{N}\MyPup{af}, name=A0
	[-keit]
	]
	]
	]{
		\draw[->,dashed] (V0) to[out=east,in=east] (V1);
		\draw[->,dashed] (A0) to[out=east,in=east] (A1);
		\draw[->,dashed] (A1) to[out=east,in=east] (N1);
	}
\end{forest}

}
\end{figure}

\end{minipage}

\end{frame}


%%%%%%%%%%%%%%%%%%%%%%%%%%%%%%%%%%
\begin{frame}
\frametitle{Struktur komplexer Wörter: Rekursion}


\begin{block}{Rekursion}
	Prozess, der es erlaubt, mittels einer \textbf{endlichen Menge von Elementen und Regeln} eine \textbf{unendliche Menge von Symbolfolgen} zu erzeugen \citep[vgl.][]{Hetland14a, Olsen15a}. 
\end{block}

\begin{itemize}
	\item Durch rekursive Regeln (vgl.\ (\ref{ex:5bRekRegel1}) \& (\ref{ex:5bRekRegel2})) kann man eine unendliche Menge von Wörtern oder Sätze generieren.

	\item \textbf{Einige Wortbildungsprozesse} können rekursiv sein.
		
	\item Durch \textbf{Nominalkomposition} kann man sehr komplexe Stämme erzeugen (\ref{ex:5bRekRegel3}).
	
\end{itemize}

\settowidth\jamwidth{[XXabstrakte rekursive Regel]} 
\ea 
	\ea\label{ex:5bRekRegel1} $X \rightarrow Y~Z$, wobei $X = Y$ oder $X = Z$ 
	\jambox{[abstrakte rekursive Regel]}
	
	\ex\label{ex:5bRekRegel2} N \ras N N 
	\jambox{[konkrete rekursive Regel]}
		
	\ex\label{ex:5bRekRegel3} Rindfleischetikettierungsüberwachungsaufgabenübertragungsgesetz
	% @EE: Link zu den Quellen:
	% https://de.wikipedia.org/wiki/Rindfleischetikettierungsüberwachungsaufgabenübertragungsgesetz
	\z 
\z 
\end{frame}


%%%%%%%%%%%%%%%%%%%%%%%%%%%%%%%%%%
\begin{frame}
\frametitle{Struktur komplexer Wörter: Rekursion}

\begin{minipage}{.74\textwidth}

\begin{itemize}
	\item Anwendung einer rekursiven Regel (\ref{ex:5bRekRegel4})
	
	\ea \label{ex:5bRekRegel4} $\alertblue{X} \rightarrow \alertblue{X}~Y$	
	\z 
	
	\item Mit einigen Affixen ist rekursive Wortbildung möglich:
	
	\ea \alertred{Ur}großmutter, \alertred{Ur}urgroßmutter, \alertred{Ur}ururgroßmutter, \ldots
	\z 
	
	\item Andere Affixe erlauben keine rekursive Wortbildung:
	\ea 
		\ea[]{Schönheit}
		\ex[*]{Schönheit\alertred{heit}}
		\z 
	\z 
	
\end{itemize}

\end{minipage}
%%
%%
\begin{minipage}[c]{.25\textwidth}

\begin{figure}
	\centering
	\scalebox{.9}{
		\begin{forest}
			[X
			[X
			[X
			[X]
			[Y]
			]
			[Y]
			]
			[Y]
			]	
		\end{forest}	
	}
\end{figure}

\end{minipage}

\end{frame}


%%%%%%%%%%%%%%%%%%%%%%%%%%%%%%%%%%
\begin{frame}
\frametitle{Struktur komplexer Wörter: Rekursion}

\begin{itemize}
	\item Mit Adjektiven als Erstglied ist \textbf{keine Rekursion} möglich.
	
	\ea *Samtigrotwein, *Weißmagerquark, *Feuchtgrünfutter
	\z
	
	\item Das kann in der Regel dadurch erfasst werden, dass die \textbf{Symbole rechts und links der Regel unterschiedlich} sind.
	
	\ea N\MyPdown{\alertred{1}} \ras A N\MyPdown{\alertred{2}}
	\z   
	
	\item Es gibt Ausnahmen:
	
	\ea Frühneuhochdeutsch, Billigrotwein
	\z 
	
\end{itemize}

\end{frame}


%%%%%%%%%%%%%%%%%%%%%%%%%%%%%%%%%%
%%%%%%%%%%%%%%%%%%%%%%%%%%%%%%%%%%
\subsection{Komposition}
\iftoggle{toc}{
\frame{
\begin{multicols}{2}
\frametitle{~}
	\tableofcontents[currentsection]
\end{multicols}
}
}

%%%%%%%%%%%%%%%%%%%%%%%%%%%%%%%%%%

\begin{frame}
\frametitle{Komposition}

\begin{itemize}
	\item \textbf{Komposition:} Kombination von \textbf{Stämmen} (Kompositionsgliedern) zu einem neuen Stamm.
	
	\item Ergebnis der Komposition: \textbf{Kompositum} (Pl. Komposita).
	
	\item Jedes Kompositionsglied kann selbst auch morphologisch komplex (\zB ein Kompositum) sein.
	
	\ea
	\glll \alertred{Haustür} $\rightarrow$ Haus $+$ Tür \\
	Haustürschlüssel $\rightarrow$ {\alertred{[Haus$+$Tür]}} $+$ Schlüssel\\
	\textbf{Kompositum} {} \textbf{Erstglied} {} \textbf{Zweitglied} \\
	\z
		 
\end{itemize}

\end{frame}


%%%%%%%%%%%%%%%%%%%%%%%%%%%%%%%%%%
\begin{frame}
\frametitle{Kopf}

\begin{itemize}
	\item \textbf{Kopfigkeit:} rechtsperiphär

	\item Ist das rechte Kompositionsglied ein Substantivstamm so ist das ganze Kompositum ein Substantiv
	
	\ea 
		\ea wein\alertred{rot} -- Rot\alertred{wein}
		
		\ex Karten\alertred{telefon} -- Telefon\alertred{karte}
	
		\ex Fahr\alertred{rad} -- rad\alertred{fahr-}
		\z
	\z 
		 
\end{itemize}

\begin{minipage}[b]{0.32\textwidth}
	\begin{figure}
		\centering
		\scalebox{0.7}{
			\begin{forest}
				sm edges,
				[A
				[N
				[welt]]
				[A
				[fremd]]]
		\end{forest}}
%	\caption{Adjektivkompositum}
	\end{figure}
\end{minipage}
%
\hfill
%
\begin{minipage}[b]{0.32\textwidth}
\begin{figure}
\centering
\scalebox{0.7}{
\begin{forest}
sm edges,
	[N
		[A
			[klein]]
		[N
			[holz]]]
\end{forest}}
%	\caption{Nominalkompositum}
\end{figure}
\end{minipage}
%
\hfill
%
\begin{minipage}[b]{0.32\textwidth}
\begin{figure}
\centering
\scalebox{0.7}{
\begin{forest}
sm edges,
	[N
		[V
			[N
				[rad]]
			[V
				[fahr]]]
		[N
			[weg]]]
\end{forest}}

\end{figure}
\end{minipage}

\end{frame}


%%%%%%%%%%%%%%%%%%%%%%%%%%%%%%%%%%
\begin{frame}
\frametitle{Kopf}

\begin{itemize}
	\item Der Kopf gibt nicht nur \textbf{kategorielle} sondern auch andere Merkmale an die Gesamtstruktur weiter.
	\item Bei Nominalkomposita bestimmt der Kopf bspw. auch \textbf{Genus} und \textbf{Flexionsklasse}:
	
	\eal 
		\ex \alertred{der} Kartoffel\alertred{salat} -- \alertblue{die} Salat\alertblue{kartoffel}
		\ex \alertred{die} Eis\alertred{schokolade} -- \alertblue{das} Schokoladen\alertblue{eis}
		\ex die Kartoffelsalat\alertred{e} -- die Salatkartoffel\alertblue{n}

	\zl
		 
\end{itemize}
\end{frame}


%%%%%%%%%%%%%%%%%%%%%%%%%%%%%%%%%%
%%%%%%%%%%%%%%%%%%%%%%%%%%%%%%%%%%
\subsubsection{Fugenelemente}
\iftoggle{toc}{
	\frame{
		\begin{multicols}{2}
			\frametitle{~}
			\tableofcontents[currentsection]
		\end{multicols}
	}
}

%%%%%%%%%%%%%%%%%%%%%%%%%%%%%%%%%%

\begin{frame}
\frametitle{Fugenelemente}

\begin{itemize}
	\item Komposition ist nicht immer einfach die Konkatenation von Stämmen.
	
	\item In ung. 30\% der Kompositionen wird noch etwas \textbf{hinzugefügt}, manchmal wird etwas \textbf{getilgt} (vgl.\ (\ref{ex:5bSchwaTilg})).
	
	\settowidth\jamwidth{[XSchwa-Tilgung]} 
	\eal 
		\ex %\emph{-es}-Einsetzung: \\
		 {[\MyPdown{N}Land\alertred{es}hauptstadt]\ra [\MyPdown{N}Land]$+$\alertred{-es}$+$[\MyPdown{N}Hauptstadt]}
		 \jambox{[\emph{-es}-Einsetzung]}
		 
		\ex %\emph{-e}-Einsetzung: \\
		 {[\MyPdown{N}Hund\alertred{e}futter]\ra [\MyPdown{N}Hund]$+$\alertred{-e}$+$[\MyPdown{N}Futter]}
		 \jambox{[\emph{-e}-Einsetzung]}
		 
		\ex %\emph{-s}-Einsetzung: \\
		 {[\MyPdown{N}Leitung\alertred{s}wasser]\ra [\MyPdown{N}Leitung]$+$\alertred{-s}$+$[\MyPdown{N}Wasser]}
		 \jambox{[\emph{-s}-Einsetzung]}
		 
		\ex\label{ex:5bSchwaTilg} %Schwa-Tilgung: \\
		 {[\MyPdown{N}Sprach\alertred{\_}kurs]\ra [\MyPdown{N}Sprache]$-$\alertred{-e}$+$[\MyPdown{N}Kurs]}
		 \jambox{[Schwa-Tilgung]}
	\zl
		 
\pause

	\item \textbf{Fugenelemente} entwickelten sich \textbf{aus Flexionsendungen} des ersten Kompositionsglieds. Synchron haben sie jedoch \textbf{keine Flexionsfunktion mehr}.
	
	\item Die \textbf{subtraktive Fuge} (vgl.\ (\ref{ex:5bSchwaTilg}) \& (\ref{ex:5bSubstrFuge})) kann mit Flexion nicht erklärt werden.
	
	\eal \label{ex:5bSubstrFuge}
	\ex die Perl\alertblue{e} -- Perl\alertred{\_}wein, Perl\alertred{\_}zwiebel
	\ex die Kehl\alertblue{e} -- Kehl\alertred{\_}kopf, Kehl\alertred{\_}laut
	\zl
	
\end{itemize}

\end{frame}


%%%%%%%%%%%%%%%%%%%%%%%%%%%%%%%%%%
\begin{frame}
\frametitle{Fugenelemente}

\begin{itemize}
	\item \gqq{Fugenelement} impliziert, dass die hinzugefügten Elemente wie Fugen \textbf{zwischen} den beteiligten Stämmen stehen. \\
	Das ist aus zwei Gründen problematisch:

	\begin{itemize}
		\item Die \textbf{Tilgung} (eine \gqq{negative Fuge}) kann so nicht erklärt werden.
		\item Es gibt Evidenz dafür, dass die hinzugefügten Elemente \textbf{zum Erstglied gehören}:

\pause 
	
	\begin{itemize}
		\item Sie bleiben bei Koordinationsellipsen (Weglassungen) beim Erstglied:
		
		\ea Leitung\alertred{s}- und Mineralwasser (nicht: und Mineral\alertred{s}wasser)
		\z
		
		\item Wird das Erstglied getilgt, darf die Fuge nicht erhalten bleiben:
		
		\ea Kind\alertred{er}wagen und -sitz (nicht: und \alertred{-er}sitz)
		\z
		
		\item Sie werden in der Regel vom Erstglied bestimmt:
		
		\ea Kuhstall -- *Küh\alertred{e}stall \vs *Huhnstall -- Hühn\alertred{er}stall
		\z
		
		\ea aber: Rind\alertred{\_}fleisch -- Rind\alertred{s}leder -- Rind\alertred{er}braten
		\z
		
	\end{itemize}

%	\item[] \textbf{ÜB.3}

\end{itemize}
\end{itemize}
\end{frame}


%%%%%%%%%%%%%%%%%%%%%%%%%%%%%%%%%%
\begin{frame}
\frametitle{Fugenelemente}

\begin{itemize}
	\item Die Flexionsendungen, die historisch zugrunde gelegen haben könnten, sind:
	
		\settowidth\jamwidth{[Xvorangestellte Genitivattribute]} 
		\ea Herz\alertred{ens}angelegenheit, Land\alertred{es}ministerium
		\jambox{[vorangestellte Genitivattribute]}

		\ex Häus\alertred{er}front, Staat\alertred{en}gemeinschaft
		\jambox{[Plural]}
		\z 

\pause 
	
	\item Es gibt jedoch zahlreiche \textbf{Gegenbeispiele}:
	
	\settowidth\jamwidth{[keine Genitivalternation bei Fuge]} 
	\eal 
		\ex Liebling\alertred{s}getränk \jambox{[semantisch falscher Genitiv]}
		\ex Lieb\alertred{es}brief \jambox{[formal falscher Genitiv]}
		\ex Hühn\alertred{er}ei, Scheib\alertred{en}wischer \jambox{[semantisch falscher Plural]}
		\ex Freund\alertred{es}kreis, Bischof\alertred{s}konferenz \jambox{[semantisch falscher Singular]}
		\ex Ende des Jahr\alertblue{es}/Jahr\alertblue{s} \jambox{[Genitivalternation]}
		
		\vs Jahr\alertred{es}zahl/*Jahr\alertred{s}zahl \jambox{[keine Genitivalternation bei Fuge]} 
	\zl
		 
\end{itemize}

\end{frame}


%%%%%%%%%%%%%%%%%%%%%%%%%%%%%%%%%%
\begin{frame}
\frametitle{Tendenzen für das Vorkommen der Fugenelemente}

\begin{itemize}
	\item Wortart des Erstglieds, Laut-, Silben- oder Wortbildungsstruktur 
	
	\item Es ist nicht möglich Fugenelemente 100\%ig vorherzusagen \citep[vgl.][]{Fuhrhop96a}.

\pause
	
	\item \textbf{phonologische Aspekte:}
	
	\begin{itemize}
		\item \textipa{[@]} nach stimmhaftem Konsonant im Stammauslaut bei verbalen Erstgliedern:
		
		\ea Pfleg-\alertred{e}-fall, Les-\alertred{e}-ecke (aber: Lesart), Reib-\alertred{e}-kuchen
		\z
		
		\item Aufeinanderfolge zweier betonter Silben wird verhindert\\
		(Primärakzent: ' / Sekundärakzent: ,):
		
		\ea 'Licht.\alertred{\_}re.,kla.me vs. 'Lich.t\alertred{er}.,ket.te (aber: 'Licht\alertred{\_}.,schal.ter)
		\z
			
	\end{itemize}

\pause 

	\item \textbf{morphologische Aspekte:}
	
	\begin{itemize}
		\item \ab{-s} steht manchmal nach komplexen Erstgliedern

		\ea {[[Hand$+$werk]\alertred{-s}]$+$[zeug] \vs [Werk]$+$[zeug] }
		\z 
	\end{itemize}

\end{itemize}

\end{frame}


%%%%%%%%%%%%%%%%%%%%%%%%%%%%%%%%%%
\begin{frame}
\frametitle{Fugenelement \vs Stammformen}

\begin{itemize}
	\item Einige Autoren \citep[vgl.][211ff; 227ff]{Eisenberg00a} sprechen von \textbf{Kompositionsstammformen} (KS): nicht nur der Stamm eines Nomens (\zB \ab{kind}) ist im Lexikon verzeichnet, sondern auch die vorkommenden Kompositionsstammformen (vgl.\ (\ref{ex:5bKSFormen})).
	
	\item Lexikon als Speicher von Idiosynkrasien, \dash von Nicht-Derivierbarem
	
	\settowidth\jamwidth{XXXXXXXXXXXXXXXXXXXXXXXXXXXXXXXXXXX} 
	\ea\label{ex:5bKSFormen} \ab{kind}:\\
	KS\MyPdown{1}: kinder \jambox{\zB in \emph{Kind\alertred{er}wagen}} 
	
	KS\MyPdown{2}: kindes \jambox{\zB in \emph{Kind\alertred{es}entführung}} 
	
	KS\MyPdown{3}: kinds \jambox{\zB in \emph{Kind\alertred{s}kopf}}
	
	KS\MyPdown{4}: kind \jambox{\zB in \emph{Kind\alertred{\_}frau}}
	\z    
	
	\item Ähnlich werden bei der Derivation \textbf{Derivationsstammformen} angenommen:
	
	\ea hoffnung\alertred{s}los, sag\alertred{e}nhaft, wein\alertred{er}lich, Hütt\alertred{\_}chen
	\z

\end{itemize}

\end{frame}


%%%%%%%%%%%%%%%%%%%%%%%%%%%%%%%%%%
\begin{frame}
\frametitle{Fugenelement \vs Stammform}

\begin{minipage}{.78\textwidth}
	
Struktur nach der \textbf{Annahme einer Kompositionsstammform}:

\begin{itemize}	
	\item Fugenelement ist \textbf{kein Morphem}

	\item Fugenelement und Stamm bilden eine \textbf{untrennbare Einheit}.
	
	\item \ab{Kinder} ist im Lexikon als Allomorph von \ab{Kind} verzeichnet.
	
	\item Annahme: Form \ab{Kinder} ist \textbf{nicht vorhersagbar}.
\end{itemize}

\end{minipage}
%%
%%
\begin{minipage}{.21\textwidth}	
	\centering	
	\scalebox{.8}{
	\begin{forest}
		[N
		[N 
		[Kind\alertred{(-er)}]
		]
		[N
		[Wagen]
		]
		]	
	\end{forest}
	}
\end{minipage}

\begin{minipage}{.76\textwidth}

\pause 

Struktur nach der \textbf{Annahme eines Fugenelements}

\begin{itemize}	
	\item Fugenelement ist \textbf{kein Morphem}
	
	\item Fugenelement bildet mit Stamm eine \textbf{neue Konstituente}.
	
	\item \ab{Kind} ist im Lexikon verzeichnet, FE \ab{-er} wird durch eine Regel hinzugefügt.
	
	\item Annahme: Form \ab{Kinder} ist (irgendwie) \textbf{vorhersagbar}.
\end{itemize}
	
\end{minipage}
%%
%%
\begin{minipage}{.23\textwidth}	
	\centering	
	\scalebox{.8}{
	\begin{forest}
		%MyP edges,	
		[N
		[N 
		[Kind]
		[FE
		[\alertred{-er}]
		]
		]
		[N
		[Wagen]
		]
		]		
	\end{forest}
	}
\end{minipage}

\end{frame}


%%%%%%%%%%%%%%%%%%%%%%%%%%%%%%%%%%
%%%%%%%%%%%%%%%%%%%%%%%%%%%%%%%%%%
\subsubsection{Komposita: Funktionale Klassifikation}
%\iftoggle{sectoc}{
%	\frame{
%		\begin{multicols}{2}
%			\frametitle{~}
%			\tableofcontents[currentsection]
%		\end{multicols}
%	}
%}

%%%%%%%%%%%%%%%%%%%%%%%%%%%%%%%%%%

\begin{frame}
\frametitle{Komposita: Funktionale Klassifikation}

\begin{itemize}
	\item Komposita kann man nach der \textbf{semantischen Beziehung} zwischen \textbf{Erst-} und \textbf{Zweitglied} klassifizieren.
	
	\begin{block}{Determinativkompositum}
		Das Erstglied \textbf{bestimmt} das Zweitglied \textbf{näher}. 
	\end{block}
	
	\settowidth\jamwidth{XXXXXXXXXXXXXXXXXXXXXXXXXXXXXXXXXXX}

	\ea {[Wein$+$Flasche]} \jambox{$=$ $x$ ist eine Flasche, die etwas mit Wein zu tun hat.}
	
	\jambox{$\neq$ $x$ ist eine Flasche und $x$ ist Wein.}
	\z 

\pause 
	
	\begin{block}{Kopulativkompositum}
		Das Erstglied und das Zweitglied sind \textbf{gleichrangig}.\\
		Es ist nicht der Fall, dass das Erstglied  das Zweitglied \textbf{näher} \textbf{bestimmt}. 
	\end{block}
	
	\ea {[süß$+$sauer]} \jambox{$=$ $x$ ist süß und $x$ ist sauer}
	\z 
	
\end{itemize}

\end{frame}


%%%%%%%%%%%%%%%%%%%%%%%%%%%%%%%%%%
%%%%%%%%%%%%%%%%%%%%%%%%%%%%%%%%%%
\subsubsection{Determinativkomposita}
%\iftoggle{sectoc}{
%	\frame{
%		\begin{multicols}{2}
%			\frametitle{~}
%			\tableofcontents[currentsection]
%		\end{multicols}
%	}
%}

%%%%%%%%%%%%%%%%%%%%%%%%%%%%%%%%%%%

\begin{frame}
\frametitle{Determinativkomposita}

\begin{itemize}
	\item \alertblue{Erste Konstituente} (auch: Bestimmendes/Determinans) bestimmt\\
	die \alertred{zweite Konstituente} (auch: Bestimmtes/Grundwort/Determinatum) näher.

	\item Das Kompositum bezeichnet eine \textbf{Unterart} des durch die \alertred{zweite Konstituente} Bezeichneten.

	\settowidth\jamwidth{XXXXXXXXXXXXXXXXXXXXXXXXXXXXX}
	\ea 
		\ea \alertblue{Wein}$+$\alertred{flasche} \jambox{\ras \alertred{Flasche}}
		\ex \alertblue{Flasche(-n)}$+$\alertred{Wein} \jambox{\ras \alertred{Wein}}
		\z
	
	\ex 
		\ea \alertblue{Stern(-en)}$+$\alertred{himmel} \jambox{\ras \alertred{Himmel}}
		\ex \alertblue{Himmel(-s)}$+$\alertred{stern} \jambox{\ras \alertred{Stern}}
		\z
	
	\ex 	
		\ea \alertblue{Fenster}$+$\alertred{glas} \jambox{\ras \alertred{Glas}}
		\ex \alertblue{Glas}$+$\alertred{fenster} \jambox{\ras \alertred{Fenster}}
		\z 
	\z
	
	\item produktivste Art der Komposition (ung.\ 80\% der Komposita)
\end{itemize}

\end{frame}


%%%%%%%%%%%%%%%%%%%%%%%%%%%%%%%%%%%
\begin{frame}
\frametitle{Determinativkomposita}


Bedeutungsbeziehung ist \textbf{vielfältig} und \textbf{nicht eindeutig bestimmbar}

\begin{columns}
	
\column[t]{.5\textwidth}

\begin{itemize}

	\item \textbf{räumlich},  \textbf{zeitlich} oder \textbf{kausal}
	
	\ea Gartentor, Winterferien, Freudentränen
	\z
	
	\item \textbf{Konstitution} des Zweitglieds (bestehen aus, haben, Form/Farbe):
	
	\ea Holzkäfig, Kapuzenjacke, Grünspecht
	\z

	\item \textbf{Zweck} des Zweitglieds (dienen zu, schützen vor)

	\ea Haarband, Regenmantel
	\z	
\end{itemize}

\column[t]{.5\textwidth}
	
\begin{itemize}
	\item \textbf{Instrument}eigenschaft des Zweitglieds (funktioniert mit Hilfe von)
	
	\ea Benzinmotor, Windrad
	\z
	
	\item \textbf{Vergleich}sbeziehung
	
	\ea aalglatt, krebsrot
	\z
	
	\item \textbf{steigernde} Bedeutung
	
	\ea bitterernst, mordsgeil, bettelarm
	\z
\end{itemize}

\end{columns}
	
\end{frame}


%%%%%%%%%%%%%%%%%%%%%%%%%%%%%%%%%%
\begin{frame}
\frametitle{Determinativkomposita}
	
Es ist nicht immer klar, wie genau die Bedeutungsbeziehung zwischen Erst- und Zweitglied ist. Sie ist \textbf{unabhängig} von \textbf{grammatischen Faktoren} und hängt häufig vom \textbf{Weltwissen}, \textbf{Kontext}, usw. ab:
	
	\ea Fischmann
		\ea Mann, der Fisch verkauft
		\ex Mann, der wie Fisch aussieht
		\ex Mann, der nach Fisch riecht
		\ex Mann, der viel über Fische redet
		\ex \ldots 
		\z 

\pause 
		
	\ex Auf einem Werbeschild:
	
	Hühner-Kebap 2,50\\
	Kinder-Kebap 1,10 \jambox{[Weltwissen, Kontext, etc.]}
	\z
	
	%\begin{figure}
	%\centering
	%\includegraphics[scale=.55]{material/05Morph-Kebap}
	%\end{figure}
	
\end{frame}


%%%%%%%%%%%%%%%%%%%%%%%%%%%%%%%%%%
%%%%%%%%%%%%%%%%%%%%%%%%%%%%%%%%%%
\subsubsection{Rektionskomposita}
%\iftoggle{sectoc}{
%	\frame{
%		\begin{multicols}{2}
%			\frametitle{~}
%			\tableofcontents[currentsection]
%		\end{multicols}
%	}
%}


%%%%%%%%%%%%%%%%%%%%%%%%%%%%%%%%%%
\begin{frame}
\frametitle{Rektionskomposita}

\begin{itemize}
	\item wichtige \textbf{Untergruppe} der Determinativkomposita
	
	\ea Bus\alertred{fahrer} \ras \alertred{Fahrer}, der einen Bus fährt
	\z 
\end{itemize}

\pause 

\begin{block}{Rektion}
	Art der \textbf{syntagmatischen Beziehung} zwischen zwei Einheiten, bei der die eine \textbf{grammatische Eigenschaften} der anderen bestimmt. I.\,d.\,R. redet man von Rektion bei Verben, aber andere Wortarten regieren auch Argumente.  Verben bestimmen bspw.\ den Kasus ihrer Argumente (vgl.\ (\ref{ex:5bKasus2}) \vs (\ref{ex:5bKasus3})) \citep[vgl.][]{McIntyre13a, MyP16e}.

\end{block}

\ea 
	\ea\label{ex:5bKasus2} {Jakob \textbf{unterstützt} [\MyPdown{\alertred{\textsc{akk}}}den Verein].}
	\ex\label{ex:5bKasus3} {Jakob \textbf{hilft} [\MyPdown{\alertred{\textsc{dat}}}dem Verein].}
	\z
\z  

\end{frame}


%%%%%%%%%%%%%%%%%%%%%%%%%%%%%%%%%%
\begin{frame}
\frametitle{Rektionskomposita}

\begin{itemize}
	\item Bedeutungsbeziehung zwischen Erst- und Zweitglied ist durch die \textbf{Argumentstruktur des Zweitglieds} bestimmt (vgl.\ (\ref{ex:5bFolien})), sie ist nicht so ambig wie bei anderen Determinativkomposita (vgl.\ (\ref{ex:5bFischMann})).
	
	\ea 
		\ea\label{ex:5bFischMann} Fisch\alertred{mann} \ras Mann, der \textbf{irgendwas} mit Fisch zu tun hat
		\ex\label{ex:5bFolien} Folien\alertred{bearbeitung} \ras Bearbeitung der Folien
		\z 	
	\z 
\end{itemize}

\begin{itemize}
	\item Häufig findet man Rektionskomposita bei \textbf{deverbalen} Nomina
	

	\ea 
		\ea tag(-en) \ras Tag$+$\alertred{-ung}
		\ex bearbeit(-en) \ras Bearbeit$+$\alertred{-ung}
		\ex fahr(-en) \ras Fahr$+$\alertred{-er}		
		\z 
	\z 
	
\end{itemize}


\end{frame}


%%%%%%%%%%%%%%%%%%%%%%%%%%%%%%%%%%
\begin{frame}
\frametitle{Rektionskomposita}

\begin{itemize}
	
	\item Verb bestimmt mit wie vielen und mit welchen Argumenten es im Satz erscheint (vgl.\ (\ref{ex:5bBsp1}) \& (\ref{ex:5bBsp2})) (s.\ Rektion, Valenz, Subkategorisierungsrahmen)
	
	\item \textbf{Erstglied} in einem deverbalen Rektionskompositum realisiert ein \textbf{Argument} des der zweiten Konstituente zugrunde liegenden Verbs.

	\settowidth\jamwidth{[2 ArgumenteX]} 
	\ea \label{ex:5bBsp1} 
		\ea {[\MyPdown{\alertred{\textsc{subj}}}Die \alertred{Linguisten}] \alertblue{tagen}.}
		\jambox{[1 Argument]}
		\ex die \alertblue{Tagung} der Linguisten 
		\ex \alertred{Linguisten}\alertblue{tagung}
		\z 

	\ex \label{ex:5bBsp2} 
		\ea {Die Linguisten \alertblue{bearbeiten} [\MyPdown{\alertred{\textsc{obj}}}die \alertred{Folien}].}
		\jambox{[2 Argumente]}
		\ex die \alertblue{Bearbeitung} der Folien
		\ex \alertred{Folien}\alertblue{bearbeitung}
		\z 
	\z
	
	\ea	 
		\ea Auto\alertblue{fahrer} (jemand fährt Auto)
		\ex Wetter\alertblue{beobachter} (jemand beobachtet das Wetter)
		\ex Rotkehlchen\alertblue{gesang} (das Rotkehlchen singt)
		\z 
	\z
		 
\end{itemize}

\end{frame}


%%%%%%%%%%%%%%%%%%%%%%%%%%%%%%%%%%
\begin{frame}
\frametitle{Rektionskomposita}

\begin{itemize}
	\item Es gibt auch Rektionskomposita, bei denen die zweite Konstituente \textbf{nicht deverbal} ist, \zB Nomen oder Adjektiv.
	
	\settowidth\jamwidth{[2 ArgumenteXXXXXXXXXXXXXXX]} 
	\ea 
		\ea {[\MyPdown{\alertred{N}}\alertred{Angst} [\MyPdown{PP}vor der \alertblue{Prüfung}]]}
		\jambox{\ras \alertblue{Prüfungs}\alertred{angst}}

		\ex Sehnsucht nach dem Tod
		\ex Todessehnsucht
		 
		\ex dem Staat treu
		\ex staatstreu
		
		\ex sicher vor Fälschung
		\ex fälschungssicher
		
		\ex frei von Blei
		\ex bleifrei
		\z 
	\z

%% @EE: Alle nach dem Muster von "Prüfungsangst"

%Üb4
		 
\end{itemize}

\end{frame}


%%%%%%%%%%%%%%%%%%%%%%%%%%%%%%%%%%
%%%%%%%%%%%%%%%%%%%%%%%%%%%%%%%%%%
\subsubsection{Possessivkomposita}
%\iftoggle{sectoc}{
%	\frame{
%		\begin{multicols}{2}
%			\frametitle{~}
%			\tableofcontents[currentsection]
%		\end{multicols}
%	}
%}


%%%%%%%%%%%%%%%%%%%%%%%%%%%%%%%%%%
\begin{frame}
\frametitle{Possessivkomposita}

\begin{itemize}
	\item Unterart der \textbf{Determinativkomposita}: Die erste Konstituente bestimmt die zweite näher, das Kompositum bezieht sich aber auf \textbf{eine dritte Entität}, sie sind \textbf{exozentrisch} (\ref{ex:5bexo}) -- im Vgl.\ zu Rektionskomposita und anderen Determinativkomposita, die \textbf{endozentrisch} sind (\ref{ex:5bendo}).
	
	\settowidth\jamwidth{XXXXXXXXXXXXXXXXXXXXXXXXXXXXXXX} 
	\ea\label{ex:5bexo}
		\ea Rot\alertred{kehlchen} \jambox{\ras \alertblue{Vogel}, mit rotem Kehlchen, \textbf{kein Kehlchen}}

	
		\ex Rot\alertred{käppchen} \jambox{\ras \alertblue{Person} mit roter Kappe, \textbf{kein Käppchen}}
	
		\ex Lang\alertred{finger} \jambox{\ras \alertblue{Person} mit langen Fingern, \textbf{kein Finger}}
		\z

	
	\ex \label{ex:5bendo}
		\ea Prüfungs\alertred{angst} \jambox{\ras \alertred{Angst}}
		\ex Linguisten\alertred{tagung} \jambox{\ras \alertred{Tagung}}
		\ex Wein\alertred{flasche} \jambox{\ras \alertred{Flasche}}
		\z 
	\z 

\pause 

	\item Der \textbf{morphosyntaktische} Kopf ist die rechte Konstituente (wie immer), aber bzgl.\ der lexikalischen Bedeutung sind Possessivkomposita \textbf{exozentrisch}\\
	\citep[vgl.][]{Fries&MyP16j}.
	
\end{itemize}

\end{frame}


%%%%%%%%%%%%%%%%%%%%%%%%%%%%%%%%%%
%%%%%%%%%%%%%%%%%%%%%%%%%%%%%%%%%%
\subsubsection{Kopulativkomposita}
%\iftoggle{sectoc}{
%	\frame{
%		\begin{multicols}{2}
%			\frametitle{~}
%			\tableofcontents[currentsection]
%		\end{multicols}
%	}
%}
%%%%%%%%%%%%%%%%%%%%%%%%%%%%%%%%%%

\begin{frame}
\frametitle{Kopulativkomposita}

\begin{itemize}
	\item Kein Determinativkompositum: \\
	Erste Konstituente \textbf{bestimmt} die zweite \textbf{nicht näher}.
	
	\ea rot-grün, Fürst-Bischof
	\z 
	
	\item Beide Konstituenten sind \textbf{gleichrangig}.
	
	\item \textbf{Koordinierende} (\dash verknüpfende) Beziehung zwischen den Kompositionsgliedern: Bedeutung des Kompositums ergibt sich \textbf{additiv}.
	
	\settowidth\jamwidth{XXXXXXXXXXXXXXXXXXXXXXXXXXXXXXX} 
	\ea 
		\ea {[süß$+$sauer]} \jambox{$=$ $x$ ist süß \textbf{und} $x$ ist sauer}
		\ex {[Spieler$+$Trainer]} \jambox{$=$ $x$ ist Spieler \textbf{und} $x$ ist Trainer}
		\z 
	\z 

	\item möglich auch aus mehr als zwei Konstituenten bestehend
		
	\ea rot-rot-grün
%	\ex süß$+$sauer, nass$+$kalt, rot$+$grün, Fürst-Bischof
%	\ex 
	\z 
	
\end{itemize}

\end{frame}


%%%%%%%%%%%%%%%%%%%%%%%%%%%%%%%%%%
\begin{frame}
\frametitle{Kopulativkomposita}

\begin{itemize}
	\item Konstituenten in Kopulativkomposita haben die \textbf{gleiche Kategorie}
	
	\ea 
		\ea nass-kalt \ras A $+$ A
		\ex Schauspieler-Regisseur \ras N $+$ N
		\z 
	\z 
	
	\item Die Reihenfolge ist im Prinzip \textbf{frei}, aber meistens \textbf{konventionalisiert}.
	
	\ea Strumpfhose \vs ?Hosenstrumpf
	\z 
	
	\item Anderes \textbf{Betonungsmuster} als Determinativkomposita: \\
	Bei Determinativkomposita wird der \textbf{Nichtkopf} betont, \\
	bei Kopulativkomposita werden \textbf{alle Konstituenten} betont.
	
	\settowidth\jamwidth{X[Determinativkompositum]} 
	\ea 
		\ea ein 'blau-'grünes 'Hemd  \jambox{[Kopulativkompositum]}
	
		\ex ein 'blaugrünes 'Hemd \jambox{[Determinativkompositum]}
		\z 
	\z
		 
%	\item 

%	\item[] \textbf{ÜB.5}
\end{itemize}


\end{frame}


%%%%%%%%%%%%%%%%%%%%%%%%%%%%%%%%%%
%%%%%%%%%%%%%%%%%%%%%%%%%%%%%%%%%%
\subsubsection{Komposition: Wortstruktur}
%\iftoggle{sectoc}{
%	\frame{
%		\begin{multicols}{2}
%			\frametitle{~}
%			\tableofcontents[currentsection]
%		\end{multicols}
%	}
%}


%%%%%%%%%%%%%%%%%%%%%%%%%%%%%%%%%%
\begin{frame}
\frametitle{Wortstruktur: Kopf}

\begin{itemize}
	\item Bei allen Kompositionsarten gilt das Prinzip der \textbf{Rechtsköpfigkeit}.
	
	\settowidth\jamwidth{X[Determinativkompositum]} 
	\ea 
		\ea die Wein\alertred{flasche} \jambox{[Determinativkompositum]} 
		
			\ea[]{die Weinflasche\alertblue{n} }
			\ex[*]{die Wein\alertblue{e}flasche\alertblue{n}}
			\z 		
		
		\ex die Schnaps\alertred{nase} 	\jambox{\hfill [Possessivkompositum]} 
			\ea[]{die Schnapsnase\alertblue{n} }
			\ex[*]{die Schnäps\alertblue{e}nase\alertblue{n}}
			\z 
	
		
		\ex der Fürst-\alertred{Bischof} \jambox{[Kopulativkompositum]} 
			\ea[]{die Fürst-Bischöf\alertblue{e}}
			\ex[*]{die Fürst\alertblue{en}-Bischöf\alertblue{e}}
			\z 		
		\z 
	\z 

	\item Bspw. bestimmt der Kopf wie das gesamte Kompositum pluralisiert wird. Der Nicht-Kopf wird nicht pluralisiert.

\end{itemize}

\end{frame}


%%%%%%%%%%%%%%%%%%%%%%%%%%%%%%%%%%
\begin{frame}
\frametitle{Wortstruktur: Verzweigung}

\begin{itemize}
	\item Die meisten Komposita sind \textbf{binär}. Kopulativkomposita können mehr als zweigliedrig sein.

\scalebox{.8}{
\begin{forest}
		[A
			[A [rot]]
			[A [rot]]
			[A [grün]]
		]
\end{forest}	
}
%	\item Einige der Kompositionsregeln (aber nicht alle) sind \textbf{rekursiv}.
	
	
	\item Komposita können symmetrisch strukturiert (\textbf{beidseitigverzweigend}) (\ref{ex:Bsp3}), \textbf{linksverzweigend} (\ref{ex:Bsp4}) oder \textbf{rechtsverzweigend} (\ref{ex:Bsp5}) sein.
	
	\ea \label{ex:Bsp3} {((Groß$+$raum)$+$(flug$+$zeug))}
	
	\ex \label{ex:Bsp4} {(((Berg$+$bau)$+$(wissenschaft$+$s)$+$studium)}
	
	\ex \label{ex:Bsp5} (Bezirk$+$s$+$(jahr$+$es$+$(haupt$+$versammlung)))
	\z

%@EE Baume malen, und runde durch eckige Klammern ersetzen
	
\end{itemize}

%
%\scalebox{.8}{
%\begin{forest}
%	[A
%	[A [rot]]
%	[A [rot]]
%	[A [grün]]
%	]
%\end{forest}	
%}

\end{frame}


%%%%%%%%%%%%%%%%%%%%%%%%%%%%%%%%%%
\begin{frame}
\frametitle{Wortstrukturregeln: Interpretation}

\begin{itemize}
	\item Komposita können \textbf{strukturell ambig} sein, vgl.\ (\ref{ex:Bsp6}) und (\ref{ex:Bsp7})
	
	\ea{[[\alertred{Bund(-es)$+$straße(-n)}]$+$bau] \vs [Bund(-es)$+$[\alertred{straße(-n)$+$bau}]]}

\pause 

	\ex\label{ex:Bsp6}  ((Frau(-en)$+$film)$+$fest) $=$ \alertblue{Fest}, das etwas mit \alertred{Frauenfilmen} (\zB Filmen von weiblichen Regisseurinnen) zu tun hat
	
	\ex\label{ex:Bsp7}  (Frau(-en)$+$(film$+$fest)) $=$ \alertred{Filmfest}, das etwas mit \alertblue{Frauen} zu tun hat (\zB Filmfest wird von Frauen organisiert)
	\z 
\end{itemize}

%@EE: Frauenfilmfest wie Bundesstraßenbau formatieren	

\begin{minipage}{.49\textwidth}

\begin{figure}
\centering
\scalebox{.6}{
\begin{forest}
sm edges,
	[N
		[N
			[N
				[Frau(en)]]
			[N
				[film]]]
		[N
			[fest]]]
\end{forest}}
\end{figure}

%\ea\label{ex:Bsp6}  ((Frau(-en)$+$film)$+$fest)\\
%$=$ \alertblue{Fest}, das etwas mit \alertred{Frauenfilmen} (\zB Filmen von weiblichen Regisseurinnen) zu tun hat
%\z

\end{minipage}%
%
\hfill ~
%
\begin{minipage}{.49\textwidth}

\begin{figure}
\centering
\scalebox{.6}{
\begin{forest}
sm edges,
	[N
		[N
			[Frau(en)]]
		[N
			[N
				[film]]
			[N
				[fest]]]]
\end{forest}}
\end{figure}

%\ea\label{ex:Bsp7}  (Frau(-en)$+$(film$+$fest))\\
%$=$ \alertred{Filmfest}, das etwas mit \alertblue{Frauen} zu tun hat (\zB Filmfest wird von Frauen organisiert)
%\z 

\end{minipage}

\pause 

Die \textbf{Betonung} ist je nach Struktur auch unterschiedlich.

\end{frame}


%%%%%%%%%%%%%%%%%%%%%%%%%%%%%%%%%%
\begin{frame}
\frametitle{Wortstrukturregeln: Betonung}

Welche Konstituente trägt die Hauptbetonung in den folgenden Wörtern?

\begin{columns}
\column{.46\textwidth}
	\ea\label{ex:5bFuss} \rotul<2->{Fuß}$+$ball$+$feld
	\ex\label{ex:5bLandes} Landes$+$\rotul<2->{haupt}$+$versammlung
	\z 

\column{.46\textwidth}
	\ea\label{ex:5bWelt} Welt$+$\rotul<2->{nicht}$+$raucher$+$tag
	\ex\label{ex:5bGross} \rotul<2->{Groß}$+$raum$+$flug$+$zeug
	\z 
\end{columns}

\pause 

\begin{itemize}
	\item Betonungs\textbf{tendenzen} bei Det.-Komposita \citep[vgl.][131ff]{Grewendorf&Co91a}:
	\begin{itemize}
		
		\item \textbf{zweigliedrige} Komposita: \textbf{Nichtkopf}
	
		\item \textbf{mehrgliedrige} Komposita: meist der \textbf{Nichtkopf} \textbf{der verzweigenden Konstituente}
				
		\item \textbf{symmetrisch} verzweigenden Komposita: \textbf{linke Konstituente} (vgl.\ (\ref{ex:5bGross}))
		
	%	\ea (('Bundes$+$es$+$straße$+$n)$+$bau) \vs (Bund$+$es$+$('straße$+$n$+$bau))
	%	\z
	%	
	%	\ea '((Großraum)$+$(flugzeug))
	%	\z
		
	\end{itemize}
\end{itemize}

\begin{minipage}[t]{.17\textwidth}
\scalebox{.7}{
\begin{forest}
[
	[s
		[s [\alertred{Fuß}]]
		[w [ball]]
	]
	[w [feld]]
]	
\end{forest}	
}
\end{minipage}
%%
\hfill ~
%%
\begin{minipage}[t]{.25\textwidth}
\scalebox{.7}{
\begin{forest}
[
	[w [Landes]]
	[s 
		[s [\alertred{haupt}]]
		[w [versammlung]]
	]
]	
\end{forest}	
}
\end{minipage}
%%
\hfill ~
%%
\begin{minipage}[t]{.21\textwidth}
\scalebox{.7}{
	\begin{forest}
	[
		[w [Welt]]
			[s 
				[s 
					[s [\alertred{nicht}]]
					[w [raucher]]
				]
			[w [tag]]
		]
	]	
	\end{forest}	
}
\end{minipage}
%%
\hfill ~
%%
\begin{minipage}[t]{.28\textwidth}
\scalebox{.7}{
\begin{forest}
[
	[s 
		[s [\alertred{Groß}]]
		[w [raum]]
	]
	[w
		[s [flug]]
		[w [zeug]]
	]
]	
\end{forest}	
}
\end{minipage}
	
\end{frame}


%%%%%%%%%%%%%%%%%%%%%%%%%%%%%%%%%%%
%%%%%%%%%%%%%%%%%%%%%%%%%%%%%%%%%%%
\subsection{Exkurs: Andere Wortbildungsarten}
\iftoggle{sectoc}{
	\frame{
		\begin{multicols}{2}
			\frametitle{~}
			\tableofcontents[currentsection]
		\end{multicols}
	}
}


%%%%%%%%%%%%%%%%%%%%%%%%%%%%%%%%%%
\begin{frame}
\frametitle{Exkurs: Andere Wortbildungsarten}

\begin{itemize}
	\item \textbf{Kontamination} (Wortverschmelzung, -kreuzung, Amalgamierung)
	
	\begin{itemize}
		\item Verschmelzung zweier Wörter, so dass Wortmaterial aus einem der Originalwörter (oder beider) gelöscht wird.
		
		\ea Infotainment, Bioghurt, mainzigartig, Eurasien
		\z
	\end{itemize}
	
	\item \textbf{Generifizierung}: Ausweitung auf Gattungsbezeichnung
	
	\ea Tempo (Taschentuch), Fit
	\z
	
	\item \textbf{Analogie}: Bildung eines neuen Wortes durch Ersetzung eines Morphems eines komplexen Wortes durch ein anderes, kontextuell passenderes
	
	\ea e-card (von e-mail), slow food (von fast food)
	\z
	
\end{itemize}

\end{frame}


%%%%%%%%%%%%%%%%%%%%%%%%%%%%%%%%%%
\begin{frame}
\frametitle{Wortbildung: Arten}

\begin{itemize}
\item \textbf{Kurzwortbildung}

\begin{itemize}
	\item phonetisch ungebunden (\textbf{Abkürzung}):
	
	\ea ARD, EU, CIA
	\z
	
	\item phonetisch gebunden (\textbf{Akronym}):
	
	\ea DAX, PIN, UFO
	\z
	
\end{itemize}

\item Weitere Kurzwörter: Wortmaterial am Anfang oder am Ende des Wortes wird getilgt

\ea Kripo, Bus, Auto, bi, öko, Schumi, Alki
\z

\item \textbf{Wortschöpfung}

\ea Vileda (wie Leder), Iglo, Haribo (Hans Riegel Bonn)
\z

\end{itemize}

\end{frame}


%%%%%%%%%%%%%%%%%%%%%%%%%%%%%%%%%%
\begin{frame}
\frametitle{Wortbildung: Arten}

\begin{itemize}
\item \textbf{Rückbildung} (Reanalyse): Umdrehen einer Wortbildungsregel

\begin{itemize}
\item im Deutschen typisch bei Verben: Ableitung komplexer Verben aus komplexen Substantiven, deren Zweitglied von einem Verb stammt.
\item Rückbildung \ras Kürzung?
\item Verben als Produkt: in finaler Satzposition, mit problematischer Verbzweitstellung, Paradigma nicht vollständig

\ea bergsteigen, schleichwerben, farbkopieren, mähdreschen
\z

\item Selten auch bei der Herleitung von Substantiven oder Adjektiven zu finden:

\ea Unsympath
\z

\end{itemize}
\end{itemize}


\end{frame}


%%%%%%%%%%%%%%%%%%%%%%%%%%%%%%%%%%
\begin{frame}
\frametitle{Wortbildung: Arten}

\begin{itemize}
\item \textbf{Fremdwortbildung:} Diese Wörter gibt es in der Ursprungssprache nicht oder nicht mit dieser Bedeutung

\ea Handy, Wellness, Beamer
\z

\begin{itemize}
\item Produktiv auch mit sog. Konfixen:

\ea Thermohose, Schokaholic
\z

\end{itemize}

\end{itemize}


\end{frame}


%%%%%%%%%%%%%%%%%%%%%%%%%%%%%%%%%%
\begin{frame}
\frametitle{Wortbildung: Arten}

\begin{itemize}
\item \textbf{Reduplikation}

\begin{itemize}
\item Komplette Dopplung:

\ea Blabla, Wauwau
\z

\item Reimdopplung:

\ea Larifari, Hokuspokus
\z

\item Ablautdopplung:

\ea Wirrwarr, Wischiwaschi, Singsang
\z

\end{itemize}

\end{itemize}

\end{frame}


%%%%%%%%%%%%%%%%%%%%%%%%%%%%%%%%%%%
\begin{frame}
\frametitle{Wortbildung: Arten}

\begin{itemize}
\item \textbf{Zusammenrückung:}

\begin{itemize}
\item Aus syntaktischen Phrasen hervorgegangen
\item Wortfolge und Flexionsmarkierungen werden beibehalten

\ea Möchtegern, infolge, wassertriefend
\z

\end{itemize}

\item \textbf{Zusammenbildung:}

\begin{itemize}
\item Dreigliedrig: weder die ersten beiden noch die letzten beiden Glieder kommen frei vor
\item Manchmal als Derivation mit einem nicht lexikalischen ersten Teil

\eal 
\ex Schriftsteller, Altsprachler
\ex Schriftsteller: \\ {[}V schriftstell-{]} + {[}-er{]} \vs {[}N Schrift-{]} + {[}N -steller{]}
\zl

\end{itemize}
\end{itemize}


\end{frame}


%%%%%%%%%%%%%%%%%%%%%%%%%%%%%%%%%%
\begin{frame}
\frametitle{Wortbildung: Arten}

\begin{itemize}
\item \textbf{Komposition}

\begin{itemize}
\item Bildung einer komplexen Form, in der zwei (oder mehr) freie Morpheme auftreten

\ea Edelmut, Baukran, Geisteswissenschaft, süßsauer
\z

\end{itemize}

\end{itemize}

\end{frame}

%%%%%%%%%%%%%%%%%%%%%%%%%%%%%%%%%%
\begin{frame}
\frametitle{Wortbildung: Arten}

\begin{itemize}
\item \textbf{Derivation}

\begin{itemize}
\item Bildung einer komplexen Form, meist mittels Derivationsaffixen, die dem Stamm vorausgehen oder ihm folgen können

\ea Ableit + ung, ver + schlaf-, Un + mensch
\z

\item Explizite / äußere Derivation: mittels abtrennbarer Affixe

\ea (Grab + ung).
\z

\item Implizite / innere Derivation: ohne klar abtrennbare Affixe

\ea trink- \vs Trank
\z

\end{itemize}
\end{itemize}

\end{frame}

%%%%%%%%%%%%%%%%%%%%%%%%%%%%%%%%%%
\begin{frame}
\frametitle{Wortbildung: Arten}

\begin{itemize}
\item \textbf{Konversion:}

\begin{itemize}
\item Umsetzung eines Stammes in eine andere Kategorie
\item ohne zusätzliches Morphem oder sonstige Veränderungen
\item Konversion \ras Derivation ? (Derivation mit einem Nullmorphem)

\eal 
\ex Nomen Dank \vs Verb dank-
\ex das Blau
\ex die Betrunkene
\zl

\end{itemize}

%\item[] \textbf{ÜB.1}
\end{itemize}


\end{frame}


%%%%%%%%%%%%%%%%%%%%%%%%%%%%%%%%%%%
%%%%%%%%%%%%%%%%%%%%%%%%%%%%%%%%%%%
%\section{X}
%%\frame{
%%\frametitle{~}
%%	\tableofcontents[currentsection]
%%}
%
%
%%%%%%%%%%%%%%%%%%%%%%%%%%%%%%%%%%%
%\begin{frame}
%\frametitle{Y}
%
%\begin{itemize}
%	\item 
%\end{itemize}
%
%
%\end{frame}


%%%%%%%%%%%%%%%%%%%%%%%%%%%%%%%%%%%
%%%%%%%%%%%%%%%%%%%%%%%%%%%%%%%%%%%
%\section{X}
%%\frame{
%%\frametitle{~}
%%	\tableofcontents[currentsection]
%%}
%
%
%%%%%%%%%%%%%%%%%%%%%%%%%%%%%%%%%%%
%\begin{frame}
%\frametitle{Y}
%
%\begin{itemize}
%	\item 
%\end{itemize}
%
%
%\end{frame}


