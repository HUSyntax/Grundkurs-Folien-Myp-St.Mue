%%%%%%%%%%%%%%%%%%%%%%%%%%%%%%%%%%%%%%%%%%%%%%%%%%%%
%%%             Metadata                         %%%
%%%%%%%%%%%%%%%%%%%%%%%%%%%%%%%%%%%%%%%%%%%%%%%%%%%%      

\title{Grundkurs Linguistik}

\subtitle{Syntax V: X-Bar-Theorie -- Lexikalische Phrasen}

\author[aMyP]{
	{\small Antonio Machicao y Priemer}
%	\\
%	{\footnotesize \url{http://www.linguistik.hu-berlin.de/staff/amyp}\\
%	\href{mailto:mapriema@hu-berlin.de}{mapriema@hu-berlin.de}}
}

\institute{Institut für deutsche Sprache und Linguistik}

%%%%%%%%%%%%%%%%%%%%%%%%%      
\date{ }
%\publishers{\textbf{6. linguistischer Methodenworkshop \\ Humboldt-Universität zu Berlin}}

%\hyphenation{nobreak}


%%%%%%%%%%%%%%%%%%%%%%%%%%%%%%%%%%%%%%%%%%%%%%%%%%%%
%%%             Preamble's End                   %%%
%%%%%%%%%%%%%%%%%%%%%%%%%%%%%%%%%%%%%%%%%%%%%%%%%%%%      


%%%%%%%%%%%%%%%%%%%%%%%%%      
\huberlintitlepage
\iftoggle{toc}{
\frame{
%\begin{multicols}{2}
	\frametitle{Inhaltsverzeichnis}\tableofcontents
	%[pausesections]
%\end{multicols}
	}
	}


%%%%%%%%%%%%%%%%%%%%%%%%%%%%%%%%%%
%%%%%%%%%%%%%%%%%%%%%%%%%%%%%%%%%%
%%%%%LITERATURE:

%\nocite{Altmann&Hofmann08a}
%\nocite{Altmann93a}
\nocite{Brandt&Co06a}
\nocite{Glueck05a} 
\nocite{Grewendorf&Co91a} 
\nocite{Luedeling2009a} 
%\nocite{Meibauer&Co07a}
\nocite{MuellerS13f} 
\nocite{MuellerS15b}
\nocite{Repp&Co15a} 
\nocite{Stechow&Sternefeld88a}
%\nocite{Woellstein10a}




%%%%%%%%%%%%%%%%%%%%%%%%%%%%%%%%%%
%%%%%%%%%%%%%%%%%%%%%%%%%%%%%%%%%%
\section{Einleitung}
%\frame{
%\frametitle{~}
%	\tableofcontents[currentsection]
%}

%%%%%%%%%%%%%%%%%%%%%%%%%%%%%%%%%%
\begin{frame}
\frametitle{Einleitung}

\begin{itemize}
	\item Organisation von natürlichen Sprachen auf:
	\begin{itemize}
		\item \textbf{lexikalischer Ebene} (Wortebene)\\
		und auf
		\item \textbf{phrasaler Ebene} (Wortgruppen, die enger zusammengehören)
	\end{itemize}
	\item[]
	\item Phrasale Organisation \ras \textbf{hierarchische} Organisation
	\item[]
	\item Alle in einem Satz auftretenden Elemente sind Phrasen!
	\item[]
	\item Alle Phrasen haben den gleichen Aufbau (X-Bar-Schema)
	\item[]
	\item Unterscheidung von
	\begin{itemize}
		\item lexikalischen Phrasen \ras NP, VP, AP, (AdvP, PP)\\
		und
		\item funktionalen Phrasen \ras DP, IP, CP
	\end{itemize}
\end{itemize}

\end{frame}


%%%%%%%%%%%%%%%%%%%%%%%%%%%%%%%%%%
%%%%%%%%%%%%%%%%%%%%%%%%%%%%%%%%%%
\section{Lexikalische Phrasen}
%\frame{
%\frametitle{~}
%	\tableofcontents[currentsection]
%}

%%%%%%%%%%%%%%%%%%%%%%%%%%%%%%%%%%
\begin{frame}
\frametitle{Lexikalische Phrasen}

\begin{itemize}
	\item Maximale Projektionen, die eine \textbf{lexikalische Kategorie als Kopf} haben.
	\item[]
	\item Lexikalische Kategorien haben eine konkrete \textbf{(lexikalische) Bedeutung}.
	\item[]
	\item Sie sind durch produktive Wortbildungsregeln erweiterbar (\textbf{offene Klasse}).
	\item[]
	\item Sie können \textbf{$\theta$-Rollen} zuweisen.
	\item[]
	\item Sie können \textbf{mehrere Argumente} selegieren.
\end{itemize}


\end{frame}


%%%%%%%%%%%%%%%%%%%%%%%%%%%%%%%%%%
%%%%%%%%%%%%%%%%%%%%%%%%%%%%%%%%%%
\subsection{Nominalphrase}
%\frame{
%\frametitle{~}
%	\tableofcontents[currentsection]
%}

%%%%%%%%%%%%%%%%%%%%%%%%%%%%%%%%%%
\begin{frame}
\frametitle{Nominalphrase}

\begin{itemize}
	\item Abk.: NP
	\item (Der Determinierer wird später analysiert.)
\end{itemize}

\begin{figure}[b]
%	\begin{minipage}[b]{0.05\textwidth}
%	\end{minipage} 
%	%
	\begin{minipage}[b]{0.18\textwidth}
	\centering
	\footnotesize{
		\begin{forest}
		sn edges,
		[NP [\MyPxbar{N} [\zerobar{N} [Behandlung]]]]
		\end{forest}
		}
		%\caption{VP}	
  	\end{minipage}  
  	%  
  	\pause            
	\begin{minipage}[b]{0.03\textwidth}
	\hfill
  	\end{minipage}
  	%         
  	\begin{minipage}[b]{0.30\textwidth}
	\centering
	\footnotesize{
		\begin{forest}
		sn edges,
		[NP [\MyPxbar{N}	[\zerobar{N} [Behandlung]]
					 	[DP [des Patienten,triangle]]
			]
		]			 
		\end{forest}
		}
		%\caption{NP}
  	\end{minipage}  
  	%              
	\begin{minipage}[b]{0.03\textwidth}
	\hfill
  	\end{minipage}
  	%  
  	\pause            
	\begin{minipage}[b]{0.41\textwidth}
	\centering
	\footnotesize{
		\begin{forest}
		sn edges,
		[NP [AP [effektive,triangle]]
			[NP 
		    [\MyPxbar{N}	[\zerobar{N} [Behandlung]]
					 	[DP [des Patienten,triangle]]
			]]
		]			 
		\end{forest}
		}
		%\caption{Adjunkt und Komplement}	
  	\end{minipage}
  	%         
\end{figure}
\end{frame}


%%%%%%%%%%%%%%%%%%%%%%%%%%%%%%%%%%
%%%%%%%%%%%%%%%%%%%%%%%%%%%%%%%%%%
\subsection{Adjektivphrase}
%\frame{
%\frametitle{~}
%	\tableofcontents[currentsection]
%}

%%%%%%%%%%%%%%%%%%%%%%%%%%%%%%%%%%
\begin{frame}
\frametitle{Adjektivphrase}

\begin{itemize}
	\item Abk.: AP
\end{itemize}

\begin{figure}[b]
%	\begin{minipage}[b]{0.05\textwidth}
%	\end{minipage} 
%	%
	\begin{minipage}[b]{0.18\textwidth}
	\centering
	\footnotesize{
		\begin{forest}
		sn edges,
		[AP [\MyPxbar{A} [\zerobar{A} [stolz]]]]
		\end{forest}
		}
		%\caption{VP}	
  	\end{minipage}  
  	%  
  	\pause            
	\begin{minipage}[b]{0.03\textwidth}
	\hfill
  	\end{minipage}
  	%         
  	\begin{minipage}[b]{0.30\textwidth}
	\centering
	\footnotesize{
		\begin{forest}
		sn edges,
		[AP [\MyPxbar{A}	[PP [auf seinen Sohn,triangle]]
						[\zerobar{A} [stolz]]
			]
		]			 
		\end{forest}
		}
		%\caption{NP}
  	\end{minipage}  
  	%              
	\begin{minipage}[b]{0.03\textwidth}
	\hfill
  	\end{minipage}
  	%  
  	\pause            
	\begin{minipage}[b]{0.41\textwidth}
	\centering
	\footnotesize{
		\begin{forest}
		sn edges,
		[AP [AdvP [sehr,triangle]]
			[AP 
		    [\MyPxbar{A}	[PP [auf seinen Sohn,triangle]]				
		    			[\zerobar{A} [stolz]]
			]]
		]			 
		\end{forest}
		}
		%\caption{Adjunkt und Komplement}	
  	\end{minipage}
  	%         
\end{figure}
\end{frame}


%%%%%%%%%%%%%%%%%%%%%%%%%%%%%%%%%%
%%%%%%%%%%%%%%%%%%%%%%%%%%%%%%%%%%
\subsection{Adverbialphrase}
%\frame{
%\frametitle{~}
%	\tableofcontents[currentsection]
%}

%%%%%%%%%%%%%%%%%%%%%%%%%%%%%%%%%%
\begin{frame}
\frametitle{Adverbialphrase}

\begin{itemize}
	\item Abk.: AdvP
\end{itemize}

\begin{figure}[b]
%	\begin{minipage}[b]{0.05\textwidth}
%	\end{minipage} 
%	%
	\begin{minipage}[b]{0.18\textwidth}
	\centering
	\footnotesize{
		\begin{forest}
		sn edges,
		[AdvP [\MyPxbar{Adv} [\zerobar{Adv} [sehr]]]]
		\end{forest}
		}
		%\caption{VP}	
  	\end{minipage}  
  	%  
  	\pause            
	\begin{minipage}[b]{0.03\textwidth}
	\hfill
  	\end{minipage}
  	%         
	\begin{minipage}[b]{0.41\textwidth}
	\centering
	\footnotesize{
		\begin{forest}
		sn edges,
		[AdvP [AdvP [sehr,triangle]]
			[AdvP 
		    [\MyPxbar{Adv}	[\zerobar{Adv} [sehr]]
			]]
		]			 
		\end{forest}
		}
		%\caption{Adjunkt und Komplement}	
  	\end{minipage}
  	%         
\end{figure}
\end{frame}


%%%%%%%%%%%%%%%%%%%%%%%%%%%%%%%%%%
%%%%%%%%%%%%%%%%%%%%%%%%%%%%%%%%%%
\subsection{Prä- \& Postpositionalphrase}
%\frame{
%\frametitle{~}
%	\tableofcontents[currentsection]
%}

%%%%%%%%%%%%%%%%%%%%%%%%%%%%%%%%%%
\begin{frame}
\frametitle{Prä- \& Postpositionalphrase}

\begin{itemize}
	\item \textbf{Präpositionalphrase} \ras PP
	\item \textbf{Postpositionalphrase} \ras PostP
\end{itemize}

\begin{figure}[b]
%	\begin{minipage}[b]{0.05\textwidth}
%	\end{minipage} 
%	%
  	\begin{minipage}[b]{0.30\textwidth}
	\centering
	\footnotesize{
		\begin{forest}
		sn edges,
		[PP [\MyPxbar{P}
				[\zerobar{P} [mit]]
				[DP [dem Finger,triangle]]
			]
		]
		\end{forest}
		}
		%\caption{NP}
  	\end{minipage}  
  	%              
  	\pause            
	\begin{minipage}[b]{0.30\textwidth}
	\centering
	\footnotesize{
		\begin{forest}
		sn edges,
		[PostP [\MyPxbar{Post} [DP [der Syntax, triangle]]
							[\zerobar{Post} [wegen]]]
		]	 
		\end{forest}
		}
		%\caption{Adjunkt und Komplement}	
  	\end{minipage}
  	%
 	\pause            
	\begin{minipage}[b]{0.30\textwidth}
	\centering
	\footnotesize{
		\begin{forest}
		sn edges,
		[PP
		[AdvP [kurz, triangle]]
		[PP [\MyPxbar{P}
				[\zerobar{P} [vor]]
				[DP [dem Haus,triangle]]
			]
		]]
		\end{forest}
		}
		%\caption{NP}
  	\end{minipage}  
  	%                       
\end{figure}
\end{frame}


%%%%%%%%%%%%%%%%%%%%%%%%%%%%%%%%%%
%%%%%%%%%%%%%%%%%%%%%%%%%%%%%%%%%%
\subsection{Verbalphrase}
%\frame{
%\frametitle{~}
%	\tableofcontents[currentsection]
%}

%%%%%%%%%%%%%%%%%%%%%%%%%%%%%%%%%%
\begin{frame}
\frametitle{Verbalphrase}

\begin{itemize}
	\item Abk.: VP
	\item \textbf{Kopf} der VP \ras rechtsperipher 
\end{itemize}

\begin{figure}[b]
%	\begin{minipage}[b]{0.05\textwidth}
%	\end{minipage} 
%	%
	\begin{minipage}[b]{0.18\textwidth}
	\centering
	\footnotesize{
		\begin{forest}
		sn edges,
		[VP [\MyPxbar{V} [\zerobar{V} [schlafen]]]]
		\end{forest}
		}
		%\caption{VP}	
  	\end{minipage}  
  	%  
  	\pause            
	\begin{minipage}[b]{0.03\textwidth}
	\hfill
  	\end{minipage}
  	%         
  	\begin{minipage}[b]{0.30\textwidth}
	\centering
	\footnotesize{
		\begin{forest}
		sn edges,
		[VP [\MyPxbar{V}	[DP [den Wagen,triangle]]
						[\zerobar{V} [kaufen]]
			]
		]			 
		\end{forest}
		}
		%\caption{NP}
  	\end{minipage}  
  	%              
	\begin{minipage}[b]{0.03\textwidth}
	\hfill
  	\end{minipage}
  	%  
  	\pause            
	\begin{minipage}[b]{0.41\textwidth}
	\centering
	\footnotesize{
		\begin{forest}
		sn edges,
		[VP [AdvP [schnell,triangle]]
			[VP 
		    [\MyPxbar{V}	[DP [den Wagen,triangle]]				
		    			[\zerobar{V} [kaufen]]
			]]
		]			 
		\end{forest}
		}
		%\caption{Adjunkt und Komplement}	
  	\end{minipage}
  	%         
\end{figure}

\end{frame}


%%%%%%%%%%%%%%%%%%%%%%%%%%%%%%%%%%
\begin{frame}
\frametitle{Verbalphrase}

\begin{itemize}
	\item \textbf{Transitive} und \textbf{ditransitive} Verbalphrase
	\item \textbf{Kopf} der VP \ras rechtsperipher 
\end{itemize}

\begin{figure}[b]
%	\begin{minipage}[b]{0.05\textwidth}
%	\end{minipage} 
%	%
	\begin{minipage}[b]{0.43\textwidth}
	\centering
	\small{
		\begin{forest}
		sn edges,
		[VP [AdvP [schnell,triangle]]
			[VP 
		    [\MyPxbar{V}	[DP [den Wagen,triangle]]				
		    			[\zerobar{V} [kaufen]]
			]]
		]			 
		\end{forest}
		}
		\caption{Transitive VP}	
  	\end{minipage} 
  	%  
%  	\pause            
%	\begin{minipage}[b]{0.03\textwidth}
%	\hfill
%  	\end{minipage}
  	%  
	\begin{minipage}[b]{0.52\textwidth}
	\centering
	\small{
		\begin{forest}
		sn edges,
		[VP [AdvP [schnell,triangle]]
			[VP 
				[DP [dem Jungen, triangle]]
				[\MyPxbar{V}	
					[DP [den Wagen,triangle]]				
		    		[\zerobar{V} [schenken]]
			]]
		]			 
		\end{forest}
		}
		\caption{Ditransitive VP}	
  	\end{minipage}
  	%         
\end{figure}

\end{frame}


%%%%%%%%%%%%%%%%%%%%%%%%%%%%%%%%%%
\begin{frame}
\frametitle{Verbalphrase}

\begin{figure}[b]
%	\begin{minipage}[b]{0.05\textwidth}
%	\end{minipage} 
%	%
	\begin{minipage}[b]{0.50\textwidth}
	\begin{itemize}
	\item Position der Elemente in Phrasen ist \textbf{strukturell} bestimmt.
%	\item[]
	\item Position \ras \textbf{Funktion} 
%	\item[]
	\item Die (Basis)Position wird von der Struktur bestimmt und ist im Subkategorisierungsrahmen kodiert:
%	\item[]
	\item[] \textbf{schenken:}\\
	\alert{DP$_{\textsc{nom,ag}}$} DP$_{\textsc{dat,ziel}}$  DP$_{\textsc{akk,th}}$ $\underline{\qquad}$ 
	\item[]
	\item Deutsch \ras SOV-Sprache (später mehr dazu!)

	\end{itemize}
  	\end{minipage}  
  	%  
%  	\pause            
%	\begin{minipage}[b]{0.03\textwidth}
%	\hfill
%  	\end{minipage}
  	%  
	\begin{minipage}[b]{0.48\textwidth}
	\centering
	\footnotesize{
		\begin{forest}
		sn edges,
		[?P [DP [Die Dame,triangle]],tikz={\node [draw,red,fit=()] {};} 
			[\MyPxbar{?} 		
		[VP [AdvP [schnell,triangle]]
			[VP [DP [dem Jungen,triangle]]
		    [\MyPxbar{V}	[DP [den Wagen,triangle]]				
		    			[\zerobar{V} [schenken]]
			]]
		]
			[\zerobar{?}]
		]]			 
		\end{forest}
		}
		%\caption{Adjunkt und Komplement}	
  	\end{minipage}
  	%         
\end{figure}

\end{frame}


%%%%%%%%%%%%%%%%%%%%%%%%%%%%%%%%%%
\begin{frame}
\frametitle{Verbalphrase}

\begin{figure}[b]
%	\begin{minipage}[b]{0.05\textwidth}
%	\end{minipage} 
%	%
	\begin{minipage}[b]{0.50\textwidth}
	\begin{itemize}
	\item Position der Elemente in Phrasen ist \textbf{strukturell} bestimmt.
%	\item[]
	\item Position \ras \textbf{Funktion} 
%	\item[]
	\item Die (Basis)Position wird von der Struktur bestimmt und ist im Subkategorisierungsrahmen kodiert:
%	\item[]
	\item[] \textbf{schenken:}\\
	DP$_{\textsc{nom,ag}}$ \alert{DP$_{\textsc{dat,ziel}}$}  DP$_{\textsc{akk,th}}$ $\underline{\qquad}$ 
	\item[]
	\item Deutsch \ras SOV-Sprache (später mehr dazu!)

	\end{itemize}
  	\end{minipage}  
  	%  
%  	\pause            
%	\begin{minipage}[b]{0.03\textwidth}
%	\hfill
%  	\end{minipage}
  	%  
	\begin{minipage}[b]{0.48\textwidth}
	\centering
	\footnotesize{
		\begin{forest}
		sn edges,
		[?P [DP [Die Dame,triangle]]
			[\MyPxbar{?} 		
		[VP [AdvP [schnell,triangle]]
			[VP [DP [dem Jungen,triangle]],tikz={\node [draw,red,fit=()] {};}
		    [\MyPxbar{V}	[DP [den Wagen,triangle]]				
		    			[\zerobar{V} [schenken]]
			]]
		]
			[\zerobar{?}]
		]]			 
		\end{forest}
		}
		%\caption{Adjunkt und Komplement}	
  	\end{minipage}
  	%         
\end{figure}

\end{frame}


%%%%%%%%%%%%%%%%%%%%%%%%%%%%%%%%%%
\begin{frame}
\frametitle{Verbalphrase}

\begin{figure}[b]
%	\begin{minipage}[b]{0.05\textwidth}
%	\end{minipage} 
%	%
	\begin{minipage}[b]{0.50\textwidth}
	\begin{itemize}
	\item Position der Elemente in Phrasen ist \textbf{strukturell} bestimmt.
%	\item[]
	\item Position \ras \textbf{Funktion} 
%	\item[]
	\item Die (Basis)Position wird von der Struktur bestimmt und ist im Subkategorisierungsrahmen kodiert:
%	\item[]
	\item[] \textbf{schenken:}\\
	DP$_{\textsc{nom,ag}}$ DP$_{\textsc{dat,ziel}}$ \alert{DP$_{\textsc{akk,th}}$} $\underline{\qquad}$ 
	\item[]
	\item Deutsch \ras SOV-Sprache (später mehr dazu!)

	\end{itemize}
  	\end{minipage}  
  	%  
%  	\pause            
%	\begin{minipage}[b]{0.03\textwidth}
%	\hfill
%  	\end{minipage}
  	%  
	\begin{minipage}[b]{0.48\textwidth}
	\centering
	\footnotesize{
		\begin{forest}
		sn edges,
		[?P [DP [Die Dame,triangle]]
			[\MyPxbar{?} 		
		[VP [AdvP [schnell,triangle]]
			[VP [DP [dem Jungen,triangle]]
		    [\MyPxbar{V}	[DP [den Wagen,triangle]],tikz={\node [draw,red,fit=()] {};}				
		    			[\zerobar{V} [schenken]]
			]]
		]
			[\zerobar{?}]
		]]			 
		\end{forest}
		}
		%\caption{Adjunkt und Komplement}	
  	\end{minipage}
  	%         
\end{figure}

\end{frame}


%%%%%%%%%%%%%%%%%%%%%%%%%%%%%%%%%%
\begin{frame}
\frametitle{Verbalphrase}

\begin{figure}[b]
%	\begin{minipage}[b]{0.05\textwidth}
%	\end{minipage} 
%	%
	\begin{minipage}[b]{0.50\textwidth}
	\begin{itemize}
	\item Position der Elemente in Phrasen ist \textbf{strukturell} bestimmt.
%	\item[]
	\item Position \ras \textbf{Funktion} 
%	\item[]
	\item Die (Basis)Position wird von der Struktur bestimmt und ist im Subkategorisierungsrahmen kodiert:
%	\item[]
	\item[] \textbf{schenken:}\\
	DP$_{\textsc{nom,ag}}$ DP$_{\textsc{dat,ziel}}$ DP$_{\textsc{akk,th}}$ \alert{$\underline{schenken}$} 
	\item[]
	\item Deutsch \ras SOV-Sprache (später mehr dazu!)

	\end{itemize}
  	\end{minipage}  
  	%  
%  	\pause            
%	\begin{minipage}[b]{0.03\textwidth}
%	\hfill
%  	\end{minipage}
  	%  
	\begin{minipage}[b]{0.48\textwidth}
	\centering
	\footnotesize{
		\begin{forest}
		sn edges,
		[?P [DP [Die Dame,triangle]]
			[\MyPxbar{?} 		
		[VP [AdvP [schnell,triangle]]
			[VP [DP [dem Jungen,triangle]]
		    [\MyPxbar{V}	[DP [den Wagen,triangle]]				
		    			[\zerobar{V} [schenken]],tikz={\node [draw,red,fit=()] {};}
			]]
		]
			[\zerobar{?}]
		]]			 
		\end{forest}
		}
		%\caption{Adjunkt und Komplement}	
  	\end{minipage}
  	%         
\end{figure}

\end{frame}


%%%%%%%%%%%%%%%%%%%%%%%%%%%%%%%%%%
%%%%%%%%%%%%%%%%%%%%%%%%%%%%%%%%%%
\section{Übung}
%\frame{
%\frametitle{~}
%	\tableofcontents[currentsection]
%}

%%%%%%%%%%%%%%%%%%%%%%%%%%%%%%%%%%
\begin{frame}
\frametitle{Übung}

\begin{itemize}
	\item Analysieren Sie die folgende Phrase nach dem X-Bar-Schema. Verwenden Sie dabei keine Abkürzungen!
\end{itemize}

\ea kurz vor Weihnachten Otto schöne Blumen schenken
\z

\pause
\begin{figure}[b]
	%
	\begin{minipage}[b]{0.41\textwidth}
	\centering
	\tiny{
		\begin{forest}
		sn edges,
		[VP [PP [AdvP [\MyPxbar{Adv} [\zerobar{Adv} [kurz]]]]
				[PP [\MyPxbar{P} 	[\zerobar{P} [vor]]
								[NP [\MyPxbar{N}[\zerobar{N}[Weihnachten]]]]]]]
			[VP [NP [\MyPxbar{N}[\zerobar{N} [Otto]]] ]
				[\MyPxbar{V} 	[NP [AP [\MyPxbar{A} [\zerobar{A} [schöne]]]]
								[NP [\MyPxbar{N} [\zerobar{N} [Blumen]]]]]
							[\zerobar{V} [schenken]]]
			]
		]			 
		\end{forest}
		}
		\caption{Vorläufige Struktur!}	
  	\end{minipage}
  	%         
\end{figure}

\end{frame}


%%%%%%%%%%%%%%%%%%%%%%%%%%%%%%%%%%
%%%%%%%%%%%%%%%%%%%%%%%%%%%%%%%%%%
\section{Schlusswort}
%\frame{
%\frametitle{~}
%	\tableofcontents[currentsection]
%}

%%%%%%%%%%%%%%%%%%%%%%%%%%%%%%%%%%%
\begin{frame}
\frametitle{Schlusswort: The Awful German Language}

\begin{scriptsize}

%\begin{quote}
There are ten parts of speech, and they are all troublesome. An average sentence, in a German newspaper, is a sublime and impressive curiosity; it occupies a quarter of a column; it contains all the ten parts of speech -- not in regular order, but mixed; it is built mainly of compound words constructed by the writer on the spot, and not to be found in any dictionary -- six or seven words compacted into one, without joint or seam -- that is, without hyphens; it treats of fourteen or fifteen different subjects, each enclosed in a parenthesis of its own, with here and there extra parentheses, which re-enclose three or four of the minor parentheses, making pens with pens; finally, all the parentheses and re-parentheses are massed together between a couple of king-parentheses, one of which is placed in the first line of the majestic sentence and the other in the middle of the last line of it -- \emph{after which comes the verb}, and you find out for the first time what the man has been talking about; and after the verb -- merely by way of ornament, as far as I can make out, -- the writer shovels in ``\emph{haben sind gewesen gehabt haben geworden sein},'' or words to that effect, and the monument is finished. I suppose that this closing hurrah is in the nature of the flourish to a man's signature -- not necessary, but pretty. German books are easy enough to read when you hold them before the looking-glass or stand on your head,-- so as to reverse the construction, -- but I think that to learn to read and understand a German newspaper is a thing which must always remain an impossibility to a foreigner.
%\end{quote}

\hfill \citep{Twain10b}
\end{scriptsize}

\end{frame}
