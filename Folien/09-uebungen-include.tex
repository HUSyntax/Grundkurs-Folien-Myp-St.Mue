%%%%%%%%%%%%%%%%%%%%%%%%%%%%%%%%%%%%%%%%%%%%%%%%
%% Compile the master file!
%% 		Slides: Antonio Machicao y Priemer
%% 		Course: GK Linguistik
%%%%%%%%%%%%%%%%%%%%%%%%%%%%%%%%%%%%%%%%%%%%%%%%


%%%%%%%%%%%%%%%%%%%%%%%%%%%%%%%%%%%%%%%%%%%%%%%%%%%%
%%%             Metadata                         
%%%%%%%%%%%%%%%%%%%%%%%%%%%%%%%%%%%%%%%%%%%%%%%%%%%%      

\title{Grundkurs Linguistik}

\subtitle{Übungen}

\author[A. Machicao y Priemer]{
	{\small Antonio Machicao y Priemer}
	\\
	{\footnotesize \url{http://www.linguistik.hu-berlin.de/staff/amyp}}\\
	%	\href{mailto:mapriema@hu-berlin.de}{mapriema@hu-berlin.de}}
}

\institute{Institut für deutsche Sprache und Linguistik}
  
%\date{ }
%\publishers{\textbf{6. linguistischer Methodenworkshop \\ Humboldt-Universität zu Berlin}}


%%%%%%%%%%%%%%%%%%%%%%%%%%%%%%%%%%%%%%%%%%%%%%%%%%%%
%%%             Preamble's End                   %%%
%%%%%%%%%%%%%%%%%%%%%%%%%%%%%%%%%%%%%%%%%%%%%%%%%%%%      


%%%%%%%%%%%%%%%%%%%%%%%%%      
\huberlintitlepage[22pt]
\iftoggle{toc}{
	\frame{
%		\begin{multicols}{2}
			\frametitle{Inhaltsverzeichnis}\tableofcontents
			%[pausesections]
%		\end{multicols}
	}
}



%%%%%%%%%%%%%%%%%%%%%%%%%%%%%%%%%%
%%%%%%%%%%%%%%%%%%%%%%%%%%%%%%%%%%
%%%%%LITERATURE:

\nocite{Altmann&Hofmann08a}
\nocite{Altmann93a}
\nocite{Brandt&Co06a}
\nocite{Glueck05a} 
\nocite{Grewendorf&Co91a} 
\nocite{Luedeling2009a} 
\nocite{Meibauer&Co07a}
\nocite{MuellerS13f} 
\nocite{MuellerS15b}
\nocite{Repp&Co15a} 
\nocite{Stechow&Sternefeld88a}
\nocite{Woellstein10a}


%%%%%%%%%%%%%%%%%%%%%%%%%%%%%%%%%%
%%%%%%%%%%%%%%%%%%%%%%%%%%%%%%%%%%
\section{Wiederholungsstunde}


%%%%%%%%%%%%%%%%%%%%%%%%%%%%%%%%%%
%%%%%%%%%%%%%%%%%%%%%%%%%%%%%%%%%%
\subsection{Anmerkungen zur Klausur}
\iftoggle{sectoc}{
	\frame{
		%\begin{multicols}{2}
		\frametitle{~}
		\tableofcontents[currentsubsection,subsubsectionstyle=hide]
		%\end{multicols}
	}
}
%%%%%%%%%%%%%%%%%%%%%%%%%%%%%%%%%%

\begin{frame}
\frametitle{Anmerkungen zur Klausur}

\begin{itemize}
	\item \textbf{Termin:} Mo. 17.02.2020, 12--14 Uhr
	\medskip 
	
	\item \textbf{Raum:} DOR 24, 1.101
	\medskip 

	\item \textbf{Punkte:} 70
	\begin{itemize}
		\item GK: 50, UE: 20
		\item Bestanden: ab 35
	\end{itemize}
	\medskip 

	\item \textbf{Zeit:} \MyPxbar{90}
	\begin{itemize}
		\item Empfehlung GK: \MyPxbar{60}
		\item Empfehlung UE: \MyPxbar{25}
		\item Empfehlung Reserve: \MyPxbar{5}
	\end{itemize}		
\end{itemize}

\end{frame}


%%%%%%%%%%%%%%%%%%%%%%%%%%%%%%%%%
\begin{frame}
\frametitle{Anmerkungen zur Klausur}

\begin{itemize}

	\item \textbf{Sitzanordnung}
	\begin{itemize}
		\item Bitte lassen Sie immer einen Platz frei zwischen Ihnen und Ihrem Nachbarn.
		\item Nehmen Sie so Platz, dass jemand direkt vor Ihnen sitzt.
	\end{itemize}
	\medskip 

	\item \textbf{Personal-} und \textbf{Studentenausweis} mitbringen.
	\medskip 

	\item \textbf{Lehrveranstaltungs-/Arbeitsnachweise} werden mit der Klausur ausgegeben.
%	\begin{itemize}
%	 	\item \dots werden mit der Klausur abgegeben, sofern die Angaben und die Unterschrift des Dozenten der UE deutsche Grammatik vorhanden ist.

%	 	\item Die Scheine werden bei der Korrektur unterschrieben.
	 	
%	 	\item Wenn Sie den Schein vergessen haben, bzw. eine Unterschrift fehlt, können Sie den Schein direkt ins Prüfungsbüro bringen.
	
%	\end{itemize}
	
\end{itemize}

\end{frame}


%%%%%%%%%%%%%%%%%%%%%%%%%%%%%%%%%%
\begin{frame}
\frametitle{Anmerkungen zur Klausur}

\begin{itemize}
	\item Sie erhalten die Klausur in einem Umschlag. Beschädigen Sie den Umschlag bitte nicht. Denken Sie an die Umwelt!

	\item Benutzen Sie \textbf{keinen Bleistift}. Antworten mit Bleistift werden nicht berücksichtigt.
	
	\item \textbf{Handys} sind auszuschalten. 

	\item \textbf{Toilettengang} einzeln

	\item Sofortige Abgabe der Klausur bei Täuschungsversuch $+$ Meldung im Prüfungsbüro

	\item \textbf{Hilfsmittel}: DaF-Wörterbuch wird bereitgestellt, andere Hilfsmittel sind nicht erlaubt!

	\item Alle \textbf{Schmierblätter} sind ebenfalls abzugeben.
	
\end{itemize}

\end{frame}


%%%%%%%%%%%%%%%%%%%%%%%%%%%%%%%%%%
%%%%%%%%%%%%%%%%%%%%%%%%%%%%%%%%%%
\subsection{Übungen: Phonetik/Phonologie}
\iftoggle{sectoc}{
	\frame{
		%\begin{multicols}{2}
		\frametitle{~}
		\tableofcontents[currentsubsection,subsubsectionstyle=hide]
		%\end{multicols}
	}
}
%%%%%%%%%%%%%%%%%%%%%%%%%%%%%%%%%%

\begin{frame}
\frametitle{Übungen: Phonetik/Phonologie}

\begin{itemize}
	\item Erläutern Sie den Unterschied zwischen Phonem, Phon und Allophon.
	\item[]	
	\item Geben sie die artikulatorischen Eigenschaften der folgenden Laute an.
	
	\eal
	\ex \textipa{[r]}
	\ex \textipa{[P]}
	\ex \textipa{[b]}
	\ex \textipa{[x]}
	\ex \textipa{[O]}
	\ex \textipa{[u:]}
	\zl

\end{itemize}

\end{frame}


%%%%%%%%%%%%%%%%%%%%%%%%%%%%%%%%%%
\begin{frame}
\frametitle{Übungen: Phonetik/Phonologie}

\begin{itemize}
	\item Geben Sie die phonologische Repräsentation und die phonetische standarddeutsche Transkription der folgenden Wörter mit Silbenstruktur und X-Skelettschicht an.
	
	\eal
	\ex Spitzenschuhe
	\ex Zwischendinger
	\ex königlich	
	\zl

	\item Benennen Sie die phonetisch/phonologischen Prozesse, die stattfinden, bei der Aussprache der folgenden Wörter:
	
	\eal
	\ex mild
	\ex ungelenkig
	\ex süchtig
	\ex Haken
	\zl


\end{itemize}

\end{frame}


%%%%%%%%%%%%%%%%%%%%%%%%%%%%%%%%%%
\begin{frame}
\frametitle{Übungen: Phonetik/Phonologie}

\begin{itemize}
	\item Sind die folgenden Segmentfolgen mögliche phonetische Wörter des Standarddeutschen?
	\ea \textipa{[\textprimstress On.tIpl]}	
	\ex \textipa{[Ne:."nt@g]}
	\z 
	
\end{itemize}

\end{frame}


%%%%%%%%%%%%%%%%%%%%%%%%%%%%%%%%%%
%%%%%%%%%%%%%%%%%%%%%%%%%%%%%%%%%%
\subsection{Übungen: Graphematik}
\iftoggle{sectoc}{
	\frame{
		%\begin{multicols}{2}
		\frametitle{~}
		\tableofcontents[currentsubsection,subsubsectionstyle=hide]
		%\end{multicols}
	}
}
%%%%%%%%%%%%%%%%%%%%%%%%%%%%%%%%%%

\begin{frame}
\frametitle{Übungen: Graphematik}

\begin{itemize}
	\item Geben Sie Beispiele für die Anwendung der folgenden graphematischen Prinzipien an:
	
	\eal 
	\ex Prinzip der Morphemkonstanz
	\ex Homonymieprinzip
	\ex Silbisches Prinzip
	\zl
	
	\item Geben Sie die rein phonographische Schreibung des folgenden Wortes an:
	\ea 
		\ea sprachbegabt
		\ex Sträuchersee
		\z 
	\z
	
	
\end{itemize}

\end{frame}


%%%%%%%%%%%%%%%%%%%%%%%%%%%%%%%%%%
%%%%%%%%%%%%%%%%%%%%%%%%%%%%%%%%%%
\subsection{Übungen: Morphologie}
\iftoggle{sectoc}{
	\frame{
		%\begin{multicols}{2}
		\frametitle{~}
		\tableofcontents[currentsubsection,subsubsectionstyle=hide]
		%\end{multicols}
	}
}
%%%%%%%%%%%%%%%%%%%%%%%%%%%%%%%%%%

\begin{frame}
\frametitle{Übungen: Morphologie}

\begin{itemize}
	\item Welche Wortbildungsprozesse haben hier stattgefunden?

	\ea
	\ea übersetzen
	\ex bleifrei
	\ex Lauf
	\ex Bearbeitung
	\z 
	\z 
	
	\item Geben Sie die Konstituentenstruktur der folgenden Wörter an und bestimmen sie die Wortbildungstypen an jedem Knoten des Baumes so genau wie möglich.

	\ea 
	\ea Unbeweisbarkeitsannahmen
	\ex (mit den) Blickbewegunsmessern
	\z 
	\z 

\end{itemize}

\end{frame}


%%%%%%%%%%%%%%%%%%%%%%%%%%%%%%%%%%

\begin{frame}
	\frametitle{Übungen: Morphologie}
	
\begin{itemize}	
	\item Geben Sie ein Beispiel für die folgenden Kompositionsarten an:
	\ea
		\ea Determinativkomposition
		\ex Rektionskomposition
		\ex Possessivkomposition
		\ex Kopulativkomposition
		\z 
	\z 

	\item Geben Sie je ein Beispiel für einen Stamm, für eine Wurzel und für eine Basis an.
\end{itemize}

\end{frame}


%%%%%%%%%%%%%%%%%%%%%%%%%%%%%%%%%%
%%%%%%%%%%%%%%%%%%%%%%%%%%%%%%%%%%
\subsection{Übungen: Syntax}
\iftoggle{sectoc}{
	\frame{
		%\begin{multicols}{2}
		\frametitle{~}
		\tableofcontents[currentsubsection,subsubsectionstyle=hide]
		%\end{multicols}
	}
}
%%%%%%%%%%%%%%%%%%%%%%%%%%%%%%%%%%

\begin{frame}
\frametitle{Übungen: Syntax}

\begin{itemize}
	\item Ordnen Sie die folgenden Matrixsätze und ihre Nebensätze in das topologische Feldermodell ein.
	
	\eal
	\ex Petra sieht müde aus, obwohl sie nicht viel getanzt hat.
	\ex Wenn ich im Konzert bin, höre ich der Musik zu.
	\ex Die Frau, die hier arbeitet, obwohl die Heizung ausgeschaltet ist, ist leider krank geworden.
	\ex Anke hat gemerkt, dass Maria trotz der Erkältung arbeiten gegangen ist.
	\zl
	
\end{itemize}

\end{frame}


%%%%%%%%%%%%%%%%%%%%%%%%%%%%%%%%%%
\begin{frame}
\frametitle{Übungen: Syntax}

\begin{itemize}
	
	\item Testen Sie anhand von jeweils zwei Konstituententests, ob die kursiv gesetzte Wortfolge eine Konstituente des Satzes bildet.
	\ea 
		\ea Am Ende bekam Jakob das \emph{für Luise vorbereitete} Kostüm.
		\ex Er hatte sich das überlegt, \emph{weil Jakob wieder krank war}.
		\z 
	\z 
	
	\item Analysieren Sie die folgenden Sätze nach dem X-Bar-Schema.
	\ea
	\ea Maria schlägt Peter.
%	\ex Maria schlug gestern Peter.
	\ex Weil der netter Nachbar morgens gearbeitet hat, hat er erst am Nachmittag Peter getroffen.
	\ex Über die Behandlung der zwei Patienten haben bis zum Morgengrauen die Ärzte diskutiert.
	\ex Luise hat gefragt, ob Jakob kommt.
	\z 
	\z 
	
\end{itemize}

\end{frame}


%%%%%%%%%%%%%%%%%%%%%%%%%%%%%%%%%%
%%%%%%%%%%%%%%%%%%%%%%%%%%%%%%%%%%
\subsection{Übungen: Semantik/Pragmatik}
\iftoggle{sectoc}{
	\frame{
		%\begin{multicols}{2}
		\frametitle{~}
		\tableofcontents[currentsubsection,subsubsectionstyle=hide]
		%\end{multicols}
	}
}
%%%%%%%%%%%%%%%%%%%%%%%%%%%%%%%%%%

\begin{frame}
\frametitle{Übungen: Semantik/Pragmatik}

\begin{itemize}
	\item Geben Sie die Bedeutungsrelationen (so genau wie möglich) zwischen den folgenden Wörtern an.
	\eal
	\ex satt -- hungrig
	\ex erwerben -- kaufen
	\ex Haare -- Kopf
	\ex schuldig -- nicht schuldig
	\ex mehr -- Meer
	\ex Tiger -- Katze
	\ex fruchtbar -- unfruchtbar
	\zl
	
	\item Illustrieren Sie die Begriffe Satzbedeutung, Äußerungsbedeutung und Sprecherbedeutung mithilfe des folgenden Satzes.
	
	\ea Ich glaube, du gehst jetzt!
	\z
	
\end{itemize}

\end{frame}


%%%%%%%%%%%%%%%%%%%%%%%%%%%%%%%%%%
\begin{frame}
\frametitle{Übungen: Semantik/Pragmatik}

\begin{itemize}
	\item Geben Sie die Bedeutungsrelationen zwischen den folgenden Sätzen an.
	
	\eal 
	\ex Auf dem Tisch liegt eine Rose.
	\ex Auf dem Tisch liegt eine Blume.
	\zl
	
	\eal 
	\ex Alle Vögel können fliegen.
	\ex Kein Vogel kann nicht fliegen.
	\zl
	
	\eal 
	\ex Einige Tiere haben Federn.
	\ex Alle Tiere haben Federn.
	\zl
	
	\eal 
	\ex Mario ist nicht tot.
	\ex Mario ist nicht lebendig.
	\zl
	
	\eal 
	\ex Ines mag keine Orangen.
	\ex Ines mag keine Apfelsinen.
	\zl
	
	\eal 
	\ex Ich lese ein Buch.
	\ex Ich lese eine Zeitschrift.
	\zl
	
\end{itemize}

\end{frame}


%%%%%%%%%%%%%%%%%%%%%%%%%%%%%%%%%%
\begin{frame}
\frametitle{Übungen: Semantik/Pragmatik}

\begin{itemize}
	
	\item Geben Sie eine Wahrheitswerttabelle für den folgenden aussagenlogischen Ausdruck an und bestimmen Sie, ob es sich dabei um eine tautologische, eine kontradiktorische oder eine kontingente Aussage handelt.
	\ea $((p \rightarrow q) \lor q)$
	\z 
	
	\item Markieren Sie alle deiktischen und anaphorischen Elemente in den folgenden Sätzen und spezifizieren Sie diese.
	
	\eal
	\ex Sie haben diese Tür nicht geschlossen.
	\ex Gestern war mir das Wetter echt zu kalt!
	\ex Peter wusste, dass er es sich dort gemütlich machen würde. 
%	\ex \gqq{Ich bin sehr glücklich, mich wieder für das WTA-Finale qualifiziert zu haben. Ich freue mich darauf, dort anzutreten und gegen die Besten der Welt zu spielen}, sagte die 27-Jährige, die im vergangenen Jahr nur als Ersatzspielerin mitfahren durfte.
	\zl
	
\end{itemize}

\end{frame}


%%%%%%%%%%%%%%%%%%%%%%%%%%%%%%%%%%
\begin{frame}
\frametitle{Übungen: Semantik/Pragmatik}

\begin{itemize}
	\item Bestimmen Sie die Art von Folgerung, die zwischen dem ersten und den folgenden Sätzen besteht:
	
	\ea Sogar Peter hat zwei Kinder. 
	\ea Peter hat nicht mehr als zwei Kinder.
	\ex Es gibt ein Individuum namens Peter.
	\ex Peter ist Vater.
	\ex Peter hat vier Kinder.
	\z
	\z 
	
	\item Bestimmen Sie jeweils eine semantische Implikation aus dem folgenden Sätzen:
	
	\eal
	\ex In einem Schuhkarton gibt es Platz für zwei Schuhe.
	\ex Jakob hat eine Schwedin geheiratet.
	\zl
		
\end{itemize}

\end{frame}


%%%%%%%%%%%%%%%%%%%%%%%%%%%%%%%%%%
\begin{frame}
\frametitle{Übungen: Semantik/Pragmatik}

\begin{itemize}
	
	\item Bestimmen Sie jeweils eine Präsupposition aus dem folgenden Sätzen:
	\eal
	\ex Ich freue mich darüber, dass wir die Klausur bestanden haben.
	\ex Maria ist auch schwanger.
	\ex Alle Banken wurden mit unseren Steuergeldern gerettet.
	\ex Sie mögen immer noch nicht Syntax.
	\zl

	\item Geben Sie an, ob eine Maxime (scheinbar) verletzt oder befolgt wurde und um welche es sich handelt, um die angegebene Implikatur zu erhalten.
	
	\ea Wir haben einige Personen entlassen.\\
	$+>$ Es wurden nicht alle entlassen.
	\z
	
	\ea A: Wie war das Bewerbungsgespräch?\\
	B: Das Wetter ist ja super heute!\\
	$+>$ Es war furchtbar!
	\z
		
\end{itemize}

\end{frame}
