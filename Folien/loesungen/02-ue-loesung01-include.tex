%%%%%%%%%%%%%%%%%%%%%%%%%%%%%%%%%%
%% UE 1 - 02 Phonetik
%%%%%%%%%%%%%%%%%%%%%%%%%%%%%%%%%%

\begin{frame}
\frametitle{Übung -- Lösung}

Wie viele Laute haben die folgenden Wörter?

\begin{columns}
	\column{.30\textwidth}
	\begin{enumerate}
		\item \ab{Fische}
		\item \ab{Nixe}
		\item \ab{lang}
		\item \ab{Bearbeitung}
		\item[]
	\end{enumerate} 				
	\column{.35\textwidth}
	\begin{enumerate}
		\item<1-> \alertgreen{\textipa{[ f \textsci{} \textesh{} @ ]}}
		\item<3-> \alertgreen{\textipa{[ n \textsci{} k s @ ]}}
		\item<5-> \alertgreen{\textipa{[ l a N ]}}
		\item<7-> \alertgreen{\textipa{[ b @ P a \textscr\  b \t{aI} t U N ]}}
		\item<9->[] \alertgreen{\textipa{[ b @ P a:  b \t{aI} t U N ]}}
	\end{enumerate} 
	\column{.15\textwidth}
	\begin{enumerate}
		\item<2->[] \alertgreen{4}
		\item<4->[] \alertgreen{5}
		\item<6->[] \alertgreen{3}
		\item<8->[] \alertgreen{10--11} % bearbeitung ai als ein
		% Laut oder als zwei gezählt
		\item<10->[] \alertgreen{9--10} % beabeitung ai als ein
		% Laut oder als zwei gezählt
	\end{enumerate}
\end{columns}


\bigskip
\begin{itemize}
	\item[]<11-> \alertgreen{\textipa{[\t{aI}]} kann man als einen oder als zwei Laute zählen.}
\end{itemize}         

\end{frame}
