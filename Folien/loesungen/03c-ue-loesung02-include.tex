%%%%%%%%%%%%%%%%%%%%%%%%%%%%%%%%%%
%% UE 1 - 03c Phonologie
%%%%%%%%%%%%%%%%%%%%%%%%%%%%%%%%%%


\begin{frame}
\frametitle{Übung -- Lösung}

\begin{itemize}
	\item Was bedeutet die Annahme des Sonoritätsprinzips und der Onset-Maximierung für die folgenden Beispielwörter:
	
	\eal
	\settowidth\jamwidth{XXXXXXXXXXXXXXXXXXXXXXXXXXXXXXXXXXXX}
	\ex Fabrik\loesung{1}{\textipa{[fa:.b\textscr Ik]}}
	\ex Imker\loesung{2}{\textipa{[PIm.k5]}}
	\ex neblig\loesung{3}{\textipa{[ne:.blI\c{c}]}}
	\ex Falter\loesung{4}{\textipa{[fal.t5]}}
	\ex regnen\loesung{5}{\textipa{[\textscr e:.gn@n]}}
	
	\loesung{6}{Koda: *Obstruent vor Sonorant}
	\loesung{7}{Onset: *Sonorant vor Obstruent}
	
	\zl
	
	\item<8-> Onset-Maximierung ist nicht strikt. Alternativ ginge auch \textipa{[ne:p.lI\c{c}]}, \textipa{[\textscr e:k.n@n]}.
	
\end{itemize}

\end{frame}	
