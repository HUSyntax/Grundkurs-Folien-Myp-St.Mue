%%%%%%%%%%%%%%%%%%%%%%%%%%%%%%%%%%%%%%%%%%%%%%
%% Compile: XeLaTeX BibTeX XeLaTeX XeLaTeX
%% Loesung-Handout: Stefan Müller
%% Course: GK Linguistik
%%%%%%%%%%%%%%%%%%%%%%%%%%%%%%%%%%%%%%%%%%%%%%

%\documentclass[a4paper,10pt, bibtotoc]{beamer}
\documentclass[10pt,handout]{beamer}

%%%%%%%%%%%%%%%%%%%%%%%%
%%     PACKAGES      
%%%%%%%%%%%%%%%%%%%%%%%%

%%%%%%%%%%%%%%%%%%%%%%%%
%%     PACKAGES       %%
%%%%%%%%%%%%%%%%%%%%%%%%



%\usepackage[utf8]{inputenc}
%\usepackage[vietnamese, english,ngerman]{babel}   % seems incompatible with german.sty
%\usepackage[T3,T1]{fontenc} breaks xelatex

\usepackage{lmodern}

\usepackage{amsmath}
\usepackage{amsfonts}
\usepackage{amssymb}
%% MnSymbol: Mathematische Klammern und Symbole (Inkompatibel mit ams-Packages!)
%% Bedeutungs- und Graphemklammern: $\lsem$ Tisch $\rsem$ $\langle TEXT \rangle$ $\llangle$ TEXT $\rrangle$ 
\usepackage{MnSymbol}
%% ulem: Strike out
\usepackage[normalem]{ulem}  

%% Special Spaces (s. Commands)
\usepackage{xspace}				
\usepackage{setspace}
%	\onehalfspacing

%% mdwlist: Special lists
\usepackage{mdwlist}	


%%%%%%%%%%%%%%%%%%%%%%%%%%%%%%%%
%% TIPA & Phonetics

\usepackage[
%noenc,
safe]{tipa}

%% TIPA Problems/Solutions:
%% Problems with U, serif fonts and ligatures

%%Test 1
%\DeclareFontSubstitution{T3}{cmss}{m}{n}

%%Test 2
%\DeclareFontSubstitution{T3}{ptm}{m}{n}

%%Test 3
%\usepackage{tipx}


%\usepackage{vowel}


%%%%%%%%%%%%%%%%%%%%%%%%%%%%%%%%
%% Examples

\usepackage{jambox}



%\usepackage{forest-v105}
%\usepackage{langsci-forest-v105-setup}


%%%%%%%%%%%%%%%%%%%%%%%%%%%%%%%%
%% Fonts for Chinese, Vietnamese, etc. (s. Graphematik)

\usepackage{xeCJK}
\setCJKmainfont{SimSun}


%\usepackage{natbib}
%\setcitestyle{notesep={:~}}


% for toggles, is loaded in hu-beamer-includes-pdflatex
%\usepackage{etex}


%%%%%%%%%%%%%%%%%%%%%%%%%%%%%%%%
%% Fonts for Fraktur

\usepackage{yfonts}

\usepackage{url}

% für UDOP
\usepackage{adjustbox}


%% huberlin: Style sheet
%\usepackage{huberlin}
\usepackage{hu-beamer-includes-pdflatex}
\huberlinlogon{0.86cm}

% %% % use this definition, if you want to see the outlines in the handout
\renewcommand{\outline}[1]{%
%\beamertemplateemptyfootbar%
\huberlinjustbarfootline
\frame{\frametitle{\outlineheading}#1}%
%\beamertemplatecopyrightfootframenumber%
\huberlinnormalfootline 
\huberlinpagedec
}



%% Last Packages
%\usepackage{hyperref}	%URLs
%\usepackage{gb4e}		%Linguistic examples

% sorry this was incompatible with gb4e and had to go.
%\usepackage{linguex-cgloss}	%Linguistic examples (patched version that works with jambox

\usepackage{multirow}  %Mehrere Zeilen in einer Tabelle
\usepackage{adjustbox} %adjusting tables
%\usepackage{array}
\usepackage{marginnote}	%Notizen




%%%%%%%%%%%%%%%%%%%%%%%%%%%%%%%%%%%%%%%%%%%%%%%%%%%%
%%%            MyP-Commands                     
%%%%%%%%%%%%%%%%%%%%%%%%%%%%%%%%%%%%%%%%%%%%%%%%%%%%


%%%%%%%%%%%%%%%%%%%%%%%%%%%%%%%%
% Delete Caption from Figures and Tables
\setbeamertemplate{caption}{\centering\insertcaption\par }


%%%%%%%%%%%%%%%%%%%%%%%%%%%%%%%%
% German quotation marks:
\newcommand{\gqq}[1]{\glqq{}#1\grqq{}}		%double
\newcommand{\gq}[1]{\glq{}#1\grq{}}			%simple


%%%%%%%%%%%%%%%%%%%%%%%%%%%%%%%%
% Abbreviations in German
% package needed: xspace
% Short space in German abbreviations: \,	
\newcommand{\idR}{\mbox{i.\,d.\,R.}\xspace}
\newcommand{\su}{\mbox{s.\,u.}\xspace}
%\newcommand{\ua}{\mbox{u.\,a.}\xspace}       % in abbrev
%\newcommand{\zB}{\mbox{z.\,B.}\xspace}       % in abbrev
%\newcommand{\s}{s.~}
%not possibel: \dh --> d.\,h.

%rot unterstrichen
%\newcommand{\rotul}[1]{\textcolor{red}{\underline{#1}}}

%%%%%%%%%%%%%%%%%%%%%%%%%%%%%%%%
%Abbreviations in English
\newcommand{\ao}{a.o.\ }	% among others
%\newcommand{\cf}[1]{(cf.~#1)}	% confer = compare
\renewcommand{\ia}{i.a.}	% inter alia = among others
%\newcommand{\ie}{i.e.~}	% id est = that is
\newcommand{\fe}{e.g.~}	% exempli gratia = for example
%not possible: \eg --> e.g.~
\newcommand{\vs}{vs.\ }	% versus
\newcommand{\wrt}{w.r.t.\ }	% with respect to


%%%%%%%%%%%%%%%%%%%%%%%%%%%%%%%%
% Dash:
\newcommand{\gs}[1]{--\,#1\,--}


%%%%%%%%%%%%%%%%%%%%%%%%%%%%%%%%
% Rightarrow with and without space
\def\ra{\ensuremath\rightarrow}			%without space
\def\ras{\ensuremath\rightarrow\ }		%with space


%%%%%%%%%%%%%%%%%%%%%%%%%%%%%%%%
%% X-bar notation

%% Notation with primes (not emphasized): \xbar{X}
\newcommand{\MyPxbar}[1]{#1$^{\prime}$}
\newcommand{\xxbar}[1]{#1$^{\prime\prime}$}
\newcommand{\xxxbar}[1]{#1$^{\prime\prime\prime}$}

%% Notation with primes (emphasized): \exbar{X}
\newcommand{\exbar}[1]{\emph{#1}$^{\prime}$}
\newcommand{\exxbar}[1]{\emph{#1}$^{\prime\prime}$}
\newcommand{\exxxbar}[1]{\emph{#1}$^{\prime\prime\prime}$}

% Notation with zero and max (not emphasized): \xbar{X}
\newcommand{\zerobar}[1]{#1$^{0}$}
\newcommand{\maxbar}[1]{#1$^{\textsc{max}}$}

% Notation with zero and max (emphasized): \xbar{X}
\newcommand{\ezerobar}[1]{\emph{#1}$^{0}$}
\newcommand{\emaxbar}[1]{\emph{#1}$^{\textsc{max}}$}

%% Notation with bars (already implemented in gb4e):
% \obar{X}, \ibar{X}, \iibar{X}, \mbar{X} %Problems with \mbar!
%
%% Without gb4e:
\newcommand{\overbar}[1]{\mkern 1.5mu\overline{\mkern-1.5mu#1\mkern-1.5mu}\mkern 1.5mu}
%
%% OR:
\newcommand{\MyPibar}[1]{$\overline{\textrm{#1}}$}
\newcommand{\MyPiibar}[1]{$\overline{\overline{\textrm{#1}}}$}
%% (emphasized):
\newcommand{\eibar}[1]{$\overline{#1}$}
\newcommand{\eiibar}[1]{\overline{$\overline{#1}}$}

%%%%%%%%%%%%%%%%%%%%%%%%%%%%%%%%
%% Subscript & Superscript: no italics
\newcommand{\MyPdown}[1]{\textsubscript{#1}}
\newcommand{\MyPup}[1]{\textsuperscript{#1}}


%%%%%%%%%%%%%%%%%%%%%%%%%%%%%%%%
% Objekt language marking:
%\newcommand{\obj}[1]{\glqq{}#1\grqq{}}	%German double quotes
%\newcommand{\obj}[1]{``#1''}			%English double quotes
%\newcommand{\MyPobj}[1]{\emph{#1}}		%Emphasising
\newcommand{\MyPobj}[1]{\textit{#1}}		%Emphasising

%%%%%%%%%%%%%%%%%%%%%%%%%%%%%%%%
%% Semantic types (<e,t>), features, variables and graphemes in angled brackets 

%%% types and variables, in math mode: angled brackets + italics + no space
%\newcommand{\type}[1]{$<#1>$}

%%% OR more correctly: 
%%% types and variables, in math mode: chevrons! + italics + no space
\newcommand{\MyPtype}[1]{$\langle #1 \rangle$}

%%% features and graphemes, in math mode: chevrons! + italics + no space
\newcommand{\abe}[1]{$\langle #1 \rangle$}


%%% features and graphemes, in math mode: chevrons! + no italics + space
\newcommand{\ab}[1]{$\langle$#1$\rangle$}  %%same as \abu  
\newcommand{\abu}[1]{$\langle$#1$\rangle$} %%Umlaute


%%%%%%%%%%%%%%%%%%%%%%%%%%%%%%%%
% Marking text with colour:
% package needed: xcolor
% Command \alert{} in Beamer >> red
\newcommand{\alertred}[1]{\textcolor{red}{#1}}
\newcommand{\alertblue}[1]{\textcolor{blue}{#1}}
\newcommand{\alertgreen}[1]{\textcolor{green}{#1}}


%%%%%%%%%%%%%%%%%%%%%%%%%%%%%%%%
%% Outputbox
\newcommand{\outputbox}[1]{\noindent\fbox{\parbox[t][][t]{0.98\linewidth}{#1}}\vspace{0.5em}}


%%%%%%%%%%%%%%%%%%%%%%%%%%%%%%%%
%% (Syntactic) Trees
% package needed: forest
%
%% Setting for simple trees
\forestset{
	MyP edges/.style={for tree={parent anchor=south, child anchor=north}}
}

%% this is taken from langsci-setup file
%% Setting for complex trees
%% \forestset{
%% 	sm edges/.style={for tree={parent anchor=south, child anchor=north,align=center}}, 
%% background tree/.style={for tree={text opacity=0.2,draw opacity=0.2,edge={draw opacity=0.2}}}
%% }

\newcommand\HideWd[1]{%
	\makebox[0pt]{#1}%
}

%%%%%%%%%%%%%%%%%%%%%%%%%%%%%%%
%%solutions in red + w/ jambox
\newcommand{\loesung}[2]{\jambox{\only<#1->{\textcolor{red}{#2}}}}

%%%%%%%%%%%%%%%%%%%%%%%%%%%%%%%%
%% TIPA Lösungen           

%%Tipa serif font fixed (requires package 'Linux Libertine B')

%% Solution 1 (RF)
%% Tipa font:
%\renewcommand\textipa[1]{{\fontfamily{cmr}\tipaencoding #1}}

%% Solution 2 (RF): older code for texlive 2017?
%\newfontfamily{\tipacm}[Scale=MatchUppercase]{Linux Libertine B}
%\renewcommand\useTIPAfont{\tipacm}

%\NewEnviron{IPA}{\expandafter\textipa\expandafter{\BODY}} %% not needed anymore

%% Solution 3 (RF): this solution is working but with problems with ligatures
%%% works for texlive 2018
\newfontfamily{\ipafont}[Scale=MatchUppercase]{Linux Libertine B}
\def\useTIPAfont{\ipafont}

%% Solution 4 (Kopecky & MyP): Test package: tipx (s. localpackages) and comment "Solution 3" 


%%%%%%%%%%%%%%%%%%%%%%%%%%%%%%%%
%% Toggles                  


\newtoggle{uebung}
\newtoggle{loesung}\togglefalse{loesung}

\newtoggle{hausaufgabe}
%%%%%%%%%%%%werden alle durch \togglefalse{hausaufgabe} ersetzt?:
\newtoggle{ha-loesung}\togglefalse{ha-loesung}
\newtoggle{phonologie-loesung}
\newtoggle{graphematik-loesung}


%% Neue Toggle-Struktur
\newtoggle{toc}
\newtoggle{sectoc}
\newtoggle{gliederung}

\newtoggle{ue-loesung}


% The toc is not needed on Handouts. Save trees.
\mode<handout>{
\togglefalse{toc}
}

\newtoggle{hpsgvorlesung}\togglefalse{hpsgvorlesung}
\newtoggle{syntaxvorlesungen}\togglefalse{syntaxvorlesungen}

%\includecomment{psgbegriffe}
%\excludecomment{konstituentenprobleme}
%\includecomment{konstituentenprobleme-hinweis}

\newtoggle{konstituentenprobleme}\togglefalse{konstituentenprobleme}
\newtoggle{konstituentenprobleme-hinweis}\toggletrue{konstituentenprobleme-hinweis}

%\includecomment{einfsprachwiss-include}
%\excludecomment{einfsprachwiss-exclude}
\newtoggle{einfsprachwiss-include}\toggletrue{einfsprachwiss-include}
\newtoggle{einfsprachwiss-exclude}\togglefalse{einfsprachwiss-exclude}

\newtoggle{psgbegriffe}\toggletrue{psgbegriffe}

\newtoggle{gb-intro}\togglefalse{gb-intro}


%%%%%%%%%%%%%%%%%%%%%%%%%%%%%%%%
%% Useful commands                    

%%%%%%%%%%%%%%%%%%%%%
%% FOR ITEMS:
%\begin{itemize}
%  \item<2-> from point 2
%  \item<3-> from point 3 
%  \item<4-> from point 4 
%\end{itemize}
%
% or: \onslide<2->
% or \only<2->{Text}
% or: \pause

%%%%%%%%%%%%%%%%%%%%%
%% VERTICAL SPACE:
% \vspace{.5cm}
% \vfill

%%%%%%%%%%%%%%%%%%%%%
% RED MARKING OF TEXT:
%\alert{bis spätestens Mittwoch, 18 Uhr}
%\newcommand{\alertred}[1]{\textcolor{red}{#1}}

%%%%%%%%%%%%%%%%%%%%%
%% RESCALE BIG TABLES:
%\scalebox{0.8}{
%For Big Tables
%}

%%%%%%%%%%%%%%%%%%%%%
%% BLOCKS:
%\begin{alertblock}{Title}
%Text
%\end{alertblock}
%
%\begin{block}{Title}
%Text
%\end{block}
%
%\begin{exampleblock}{Title}
%Text
%\end{exampleblock}

%%%%%%%%%%%%%%%%%%%%%
%% JAMBOX FOR EXAMPLES:
%\ea 
%\settowidth\jamwidth{Test} 
%Die Studierenden, die weitgehend von Stipendien leben, erhalten einen Mietzuschuss. 
%\jambox{Test}
%\z 

%%%%%%%%%%%%%%%%%%%%%
%% TOGGLES:


%%%%%%%%%%%%%%%%%%%%%%%%%%%%%%%%%%
%%%%%%%%%%%%%%%%%%%%%%%%%%%%%%%%%%
%\subsection{Übung}
%
%%%%%%%%%%%%%%%%%%%%%%%%%%%%%%%%%%
%%%%%%%%%%%%%%%%%%%%%%%%%%%%%%%%%%
%\iftoggle{uebung}{
%%%%%%%%%%%%%%%%%%%%%%%%%%%%%%%%%%
%\begin{frame}
%\frametitle{Übung}
%
%\end{frame}
%
%} 
%%% END true = Q
%%% BEGIN false = Q + A
%{
%%%%%%%%%%%%%%%%%%%%%%%%%%%%%%%%%%
%\begin{frame}
%\frametitle{Übung}
%
%\end{frame}
%%%%%%%%%%%%%%%%%%%%%%%%%%%%%%%%%%
%
%\begin{frame}
%\frametitle{Lösung}
%
%\end{frame}
%
%}%% END LOESUNG	
%%%%%%%%%%%%%%%%%%%%%%%%%%%%%%%%%%


%%%%%%%%%%%%%%%%%%%%%%%%%%%%%%%%%%
%%%%%%%%%%%%%%%%%%%%%%%%%%%%%%%%%%
%\subsection{Hausaufgabe}
%
%%%%%%%%%%%%%%%%%%%%%%%%%%%%%%%%%%
%%%%%%%%%%%%%%%%%%%%%%%%%%%%%%%%%%
%\iftoggle{hausaufgabe}{
%%%%%%%%%%%%%%%%%%%%%%%%%%%%%%%%%%
%
%\begin{frame}
%\frametitle{Hausaufgabe}
%
%\end{frame}
%
%} 
%%% END true = Q
%%% BEGIN false = Q + A
%{
%%%%%%%%%%%%%%%%%%%%%%%%%%%%%%%%%%
%
%\begin{frame}
%\frametitle{Hausaufgabe}
%
%\end{frame}
%
%
%%%%%%%%%%%%%%%%%%%%%%%%%%%%%%%%%%
%%%%%%%%%%%%%%%%%%%%%%%%%%%%%%%%%%
%\subsection*{Lösung der Hausaufgabe}
%
%%%%%%%%%%%%%%%%%%%%%%%%%%%%%%%%%%
%
%\begin{frame}
%\frametitle{Lösung}
%
%\end{frame}
%
%}%% END LOESUNG	
%%%%%%%%%%%%%%%%%%%%%%%%%%%%%%%%%%



%%%%%%%%%%%%%%%%%%%%%%%%%%%%%%%%%%%%%%%%%%%%%%%%%%%%
%%%             Preamble's End                   
%%%%%%%%%%%%%%%%%%%%%%%%%%%%%%%%%%%%%%%%%%%%%%%%%%%% 

\begin{document}
	
	
%%%% ue-loesung
%%%% true: Übung & Lösungen (slides) / false: nur Übung (handout)
%	\toggletrue{ue-loesung}

%%%% ha-loesung
%%%% true: Hausaufgabe & Lösungen (slides) / false: nur Hausaufgabe (handout)
%	\toggletrue{ha-loesung}

%%%% toc
%%%% true: TOC am Anfang von Slides / false: keine TOC am Anfang von Slides
\toggletrue{toc}

%%%% sectoc
%%%% true: TOC für Sections / false: keine TOC für Sections (StM handout)
%	\toggletrue{sectoc}

%%%% gliederung
%%%% true: Gliederung für Sections / false: keine Gliederung für Sections
%	\toggletrue{gliederung}


%%%%%%%%%%%%%%%%%%%%%%%%%%%%%%%%%%%%%%%%%%%%%%%%%%%%
%%%             Metadata                         
%%%%%%%%%%%%%%%%%%%%%%%%%%%%%%%%%%%%%%%%%%%%%%%%%%%%      

\title{Grundkurs Linguistik}

\subtitle{Lösungen -- Graphematik}

\author[St. Mü.]{
	{\small Stefan Müller}
	\\
	{\footnotesize \url{https://hpsg.hu-berlin.de/~stefan/}}
	%	\\
	%	\href{mailto:mapriema@hu-berlin.de}{mapriema@hu-berlin.de}}
}

\institute{Institut für deutsche Sprache und Linguistik}


% bitte lassen, sonst kann man nicht sehen, von wann die PDF-Datei ist.
%\date{ }

%\publishers{\textbf{6. linguistischer Methodenworkshop \\ Humboldt-Universität zu Berlin}}

%\hyphenation{nobreak}


%%%%%%%%%%%%%%%%%%%%%%%%%%%%%%%%%%%%%%%%%%%%%%%%%%%%
%%%             Preamble's End                  
%%%%%%%%%%%%%%%%%%%%%%%%%%%%%%%%%%%%%%%%%%%%%%%%%%%%      


%%%%%%%%%%%%%%%%%%%%%%%%%      
\huberlintitlepage[22pt]
\iftoggle{toc}{
	\frame{
		\begin{multicols}{2}
			\frametitle{Inhaltsverzeichnis}
			\tableofcontents
			%[pausesections]
			\columnbreak
			\textcolor{white}{
				\ea \label{ex:04aphilo}
				\ex\label{ex:04amutter}
				\ex\label{ex:04azimmer}
				\ex\label{ex:04anacht}
				\ex\label{ex:04avasenstück}
				\ex\label{ex:04aHA3}
				\ex\label{ex:04aHA4}
				\ex\label{ex:04aHA5}
				\z
			}
		\end{multicols}
	}
}


%%%%%%%%%%%%%%%%%%%%%%%%%%%%%%%%%%%
%%%%%%%%%%%%%%%%%%%%%%%%%%%%%%%%%%%
\section{Übungen}

%%%%%%%%%%%%%%%%%%%%%%%%%%%%%%%%%%
%% UE 1 - 04a Graphematik
%%%%%%%%%%%%%%%%%%%%%%%%%%%%%%%%%%

\begin{frame}
\frametitle{Übung -- Lösung}

\begin{itemize}
	\item Geben Sie 10 Wörter an, die phonographisch geschrieben werden.
	
		\begin{description}
			\item[\alertred{\textbf{Beispiele:}}] \alertred{Beurteilung, schön, Gabel, suchen, Macht, Lager, kurz, niesen, Zopf, Gewerkschaft}
		\end{description}
	
	\item Wie würden Sie die folgenden Wörter phonographisch schreiben?
		
	\begin{exe}
		\exr{ex:04aphilo}
	\settowidth\jamwidth{XXXXXXXXXXXXXXXXXXXXXXXXXXXXX}
		\begin{xlist}
			\ex Philosophie \loesung{2}{\ab{filosofie}}
			\ex Handy \loesung{3}{\ab{hendie}}
			\ex Balkon \loesung{4}{\ab{balkong}}
			\ex Creme \loesung{5}{\ab{krem} oder \ab{kreme}}
			\ex Mutter \loesung{6}{\ab{muter}}
			\ex Streithahn \loesung{7}{\ab{schtreithan}}
		\end{xlist}
	\end{exe}
		

\end{itemize}

\end{frame}



%%%%%%%%%%%%%%%%%%%%%%%%%%%%%%%%%%
%% UE 2 - 04a Graphematik
%%%%%%%%%%%%%%%%%%%%%%%%%%%%%%%%%%

\begin{frame}
\frametitle{Übung -- Lösung}

\begin{itemize}	
	\item Versuchen Sie, graphematische Regularitäten und Prinzipien zu finden, die die Unterscheidung lang \vs kurz bei Vokalen anzeigen. Gibt es Ausnahmen?\\
	
%	\vspace{-.3cm}
	
	\begin{exe}
		\exr{ex:04amutter}
		\begin{xlist}
			\begin{multicols}{4}
				\ex Mutter
				\ex Mehl
				\ex See
				\ex Nase
				\ex dehnen
				\ex gehen
				\ex Bier
				\ex Moor
				\ex rot
				\ex zum
				\ex Mann
				\ex man
				\ex Herbst
				\ex Laub
				\ex sehr
				\ex Bohrer
			\end{multicols}
		\end{xlist}
	\end{exe}
	
\end{itemize}

%\vspace{-.3cm}


\begin{multicols}{2}
\begin{itemize}
\item[\alertgreen{--}] \alertgreen{Diphthonge: lang (n.)}
\item[\alertgreen{--}] \alertgreen{Doppelvokale: lang (c., h.), Sonderfall bei \ab{i}: \ab{ie} (g.)}
\item[\alertgreen{--}] \alertgreen{vor Dehnungs-h: lang (b., e., o., p.)}
\item[\alertgreen{--}] \alertgreen{vor Doppelkonsonant: kurz (a., k.)}
\item[\alertgreen{--}] \alertgreen{vor Konsonantencluster: kurz (m.)}
\item[\alertgreen{--}] \alertgreen{offene Schreibsilben: lang bei Hauptbetonung (d., f.),
\\sonst kurz (d.)}
\item[\alertgreen{--}] \alertgreen{einfach geschlossene Schreibsilben: uneindeutig bei Hauptbetonung\\(i., j., l.), sonst kurz (a., e., f., p.)}
\end{itemize}
\end{multicols}

\end{frame}



%%%%%%%%%%%%%%%%%%%%%%%%%%%%%%%%%%
%% UE 3 - 04a Graphematik
%%%%%%%%%%%%%%%%%%%%%%%%%%%%%%%%%%

\begin{frame}
\frametitle{Übung -- Lösung}

\begin{itemize}
	\item Warum schreibt man \ab{dehnen} mit \ab{h}, obwohl das erste \ab{e} in einer offenen Silbe steht und daher nach silbischen Prinzipien sowieso lang gesprochen werden müsste?
		\item[] \alertred{Morphemkonstanz, da bei Flexionsformen wie \ab{dehnst} geschlossene Silbe}
	\item Warum schreibt man \ab{mann} und \ab{ball}, obwohl nach silbischen Prinzipien die Geminate einen ambisyllabischen Konsonanten anzeigt?
		\item<2->[] \alertred{Morphemkonstanz, da Pluralform Silbengelenk hat}
	\end{itemize}


\end{frame}


%%%%%%%%%%%%%%%%%%%%%%%%%%
\begin{frame}
\frametitle{Übung -- Lösung}

\begin{itemize}
	\item Warum sind die Wörter \ab{(du) ziehst}, \ab{säubern} und \ab{(er) fällt} Beispiele für morphologisches Schreiben?
		\item<2->[] \alertred{\ab{h} im Infinitiv silbentrennend, Singular-Flexionsformen sind jedoch einsilbig;}
		\item<3->[] \alertred{\ab{ä} zeigt die Verwandschaft zu \ab{sauber}: laut PGK schriebe man \textipa{[O\texttoptiebar{}I]} \ab{eu};}
		\item<4->[] \alertred{Konsonantenverdopplung wegen Silbengelenk im Infinitiv, Singular-Flexionsformen sind jedoch einsilbig,\\ \ab{ä} wegen \ab{a} im Infinitiv: \textipa{[E]} wäre nach PGK \ab{e}}
	\item Wie hätte eine Person, die \ab{Rad} und \ab{König} als Beispiele für das morphologische Prinzip anführt, \gqq{phonographisches Schreiben} verstanden?
		\item<5->[] \alertred{Verschriftlichung der \emph{phonetischen} (richtig wäre: \emph{phonologische}) Repräsentation}
\end{itemize}


\end{frame}



%%%%%%%%%%%%%%%%%%%%%%%%%%%%%%%%%%
%% UE 4 - 04a Graphematik
%%%%%%%%%%%%%%%%%%%%%%%%%%%%%%%%%%

\begin{frame}
\frametitle{Übung -- Lösung}

\begin{itemize}
	\item Welche graphematischen Prinzipien (abgesehen von der phonographischen Schreibung) erklären die Schreibung der folgenden Wörter?
	
\begin{exe}
	\exr{ex:04azimmer}
	\settowidth\jamwidth{XXXXXXXXXXXXXXXXXXXXXXXXXXXXXXXXXXX}
	\begin{xlist}
		\ex \ab{Zi\rotul{mm}er} \loesung{1}{silbisches Prinzip: Silbengelenk}
		\ex \ab{W\rotul<2->{a}ise} \loesung{2}{Differenzierung homophoner Formen: zu \ab{weise}}
		\ex \ab{We\rotul<3->{h}en} \loesung{3}{silbisches Prinzip: silbentrennend}
		\ex \ab{Ru\rotul<4->{h}m} \loesung{4}{silbisches Prinzip: Dehnungs-h}
		\ex \ab{\rotul<5->{S}paß} \loesung{5}{ästhetische Schreibung: kein \ab{schp}}
		\ex \ab{A\rotul<6->{llee}} \loesung{6}{silbisches Prinzip: Silbengelenk \ab{ll}, Gespanntheit \ab{ee}}
		%          \ex \ab{Gras}
	\end{xlist}
\end{exe}
	
	\item Welche graphematische Funktion erfüllt das \ab{h} in den folgenden Wörtern?
	
\begin{exe}
	\exr{ex:04anacht}
	\settowidth\jamwidth{XXXXXXXXXXXXXXXXXXXXXXXXXXXXXXXXXXX}
	\begin{xlist}
		\ex \ab{Nacht} \loesung{7}{Teil von Digraph \ab{ch}}
		\ex \ab{Hilfe} \loesung{8}{Phonem-Graphem-Korrespondenz zu \textipa{[h]}}
		\ex \ab{sehen} \loesung{9}{silbentrennend}
		\ex \ab{Mehl} \loesung{10}{Dehnungs-h}
	\end{xlist}
\end{exe}
	
    \end{itemize}

\end{frame}


%%%%%%%%%%%%%%%%%%%%%%%%%%%%%%%%%%%%%%%
\begin{frame}{Übung -- Lösung}

	\begin{itemize}
	\item Wie würden die folgenden Wörter in phonographischer Schreibung aussehen? Geben Sie zunächst eine phonologische Transkription an (Notation mit / ~ /) und schreiben Sie anschließend phonographisch (Notation in \ab{ ~ }).

\begin{exe}
	\exr{ex:04avasenstück}
	\begin{multicols}{3}
	\begin{xlist}
		\ex \ab{Handy} 
		
		\ex \ab{Vasenstück}
		
		\ex \ab{Wannenbad}
		
%		\exi{} 
		\exi{} \only<2->{\alertred{\textipa{/hEn.di/}}}
		\exi{} \only<4->{\alertred{\textipa{/va:.z@n.StYk/}}}
		\exi{} \only<6->{\alertred{\textipa{/va\.n@n.ba:d/}}}

%		\exi{} 
		\exi{} \only<3->{\alertred{\ab{hendi}}}		
		\exi{} \only<5->{\alertred{\ab{wasenschtük}}}		
		\exi{} \only<7->{\alertred{\ab{wanenbad}}}		
	\end{xlist}
	\end{multicols}
\end{exe}
	
	\end{itemize}

\end{frame}




%%%%%%%%%%%%%%%%%%%%%%%%%%%%%%%%%%%
%%%%%%%%%%%%%%%%%%%%%%%%%%%%%%%%%%%
\section{Hausaufgaben}

%%%%%%%%%%%%%%%%%%%%%%%%%%%%%%%%%%
%% HA 1 - 04a Graphematik
%%%%%%%%%%%%%%%%%%%%%%%%%%%%%%%%%%

\begin{frame}%[allowframebreaks]
\frametitle{Hausaufgabe -- Lösung}

\begin{itemize}
	
	\item[1.] Kreuzen Sie die korrekten Aussagen an.
	
	\begin{itemize}
		\item[$\circ$] Die Orthographie ist eine linguistische Teildisziplin, die beschreibt wie man schreibt. Die Graphematik ist dagegen keine Teildisziplin der Linguistik, sondern eine \gqq{willkürliche} (normierende) Festlegung.
		
		\item[\alertgreen{$\checkmark$}] \alertgreen{Die Graphematik sollte intuitiv beherrschbar sein und das Lesen und Schreiben vereinfachen.}
		
		\item[$\circ$] Das Wort \ab{kalt} ist eine graphematisch \gqq{nackte} Silbe.
		
		\item[$\circ$] Es gibt im Deutschen eine eindeutige 1-zu-1-Korrespondenz zwischen Buchstaben und Lauten.
		
		\item[\alertgreen{$\checkmark$}] \alertgreen{Das Wort \ab{mächtig} wird aufgrund des morphologischen Prinzips (auch Prinzip der Schemakonstanz, Stammprinzip oder Verwandtschaftsprinzip) mit \ab{ä} geschrieben (vgl. \ab{Macht}).}
	\end{itemize}
\end{itemize}
\end{frame}


%%%%%%%%%%%%%%%%%%%%%%%%%%%%%%%%%%	
\begin{frame}
	\frametitle{Hausaufgabe -- Lösung}

\begin{itemize}
\item[2.] Ordnen Sie die graphematischen Prinzipien links den passenden Beispielen für die entsprechenden Prinzipien rechts zu.

NB: Beachten Sie bitte nicht die Großschreibung.

\vspace{.5cm}

\begin{minipage}{0.45\textwidth}
	\centering
	\begin{tabular}{|l|}
		\hline
		(A) Etymologische Schreibung\\
		\hline
		(B) Homonymievermeidung\\
		\hline
		(C) Morphologisches Prinzip\\
		\hline
		(D) Silbische Prinzip\\
		\hline
		(E) Phonographisches Prinzip\\
		\hline
	\end{tabular}
\end{minipage}
\hfill%
\begin{minipage}{0.45\textwidth}
	\centering
	\begin{tabular}{|p{0.075\textwidth}|l|}
		\hline
		\only<2->{\alertgreen{C}} & Bad, Bäder \\
		\hline
		\only<3->{\alertgreen{D}} & gehen \\
		\hline
		\only<4->{\alertgreen{A}} & Cello, *Tschello \\
		\hline
		\only<5->{\alertgreen{B}} & Wahl, Wal\\
		\hline
		\only<6->{\alertgreen{E}} & Flasche \\
		\hline
	\end{tabular}
\end{minipage}

\end{itemize}
\end{frame}


%%%%%%%%%%%%%%%%%%%%%%%%%%%%%%%%%%	
\begin{frame}%[allowframebreaks]
	\frametitle{Hausaufgabe -- Lösung}
	
\begin{itemize}
\item[3.] Betrachten Sie die unten angegebenen Kontexte. Diskutieren Sie kurz anhand dieser Beispiele, ob es sich bei der Groß- und Kleinschreibung des markierten Buchstabens um unterschiedliche Grapheme handeln kann oder nicht.

\begin{exe}
	\exr{ex:04aHA3}
	\begin{xlist}
	\ex Dieser \underline{W}eg ist sehr steil.
	\ex \underline{W}ege, die ich nicht bewandert habe, gibt es viele.
	\ex Meine Schlüssel sind \underline{w}eg.
	\ex \gqq{\underline{W}eg!}, schrie sie mich an und knallte mir die Tür vor der Nase zu.
	\ex Geh \underline{w}eg!
	\end{xlist}
\end{exe}
\end{itemize}
\end{frame}


%%%%%%%%%%%%%%%%%%%%%%%%%%%%%%%%%%	
\begin{frame}%[allowframebreaks]
	\frametitle{Hausaufgabe -- Lösung}

\begin{itemize}
%\pause
\item[\alertgreen{--}] \alertgreen{Graphem: Kleinste bedeutungsunterscheidende Einheit im schriftlichen System}

%\pause

\item[\alertgreen{--}] \alertgreen{\ab{Weg} und \ab{weg} kann als Minimalpaar angesehen werden, und \ab{W} und \ab{w} als unterschiedliche Grapheme, da sie bedeutungsunterscheidend sind (vgl.\ a und e). Es gibt darüber hinaus weitere Beispiele, die diese Tendenz zu belegen scheinen \ab{Reisen} \vs \ab{reisen}, \ab{Sie} \vs \ab{sie}, \ab{Gut} \vs \ab{gut}.}

%\pause

\item[\alertgreen{--}] \alertgreen{Andererseits kann die Großschreibung durch andere Prinzipien bedingt werden (\zB Satzanfang) und verliert somit den bedeutungsunterscheidenden Charakter (vgl.\ d und e).}

%\pause

\item[\alertgreen{--}] \alertgreen{Unter Berücksichtigung der gegebenen Beispiele könnte man zunächst vermuten, dass \ab{W} und \ab{w} unterschiedliche Grapheme  (vgl.\ Minimalpaare (a) und (c)). Die Groß- und  Kleinschreibung hat jedoch eine andere Funktion im Schriftsystem des Deutschen (\zB Markierung von Nomina und Satzanfängen) und wirkt sich somit nicht notwendigerweise bedeutungsunterscheidend aus.}
\end{itemize}

\end{frame}


%%%%%%%%%%%%%%%%%%%%%%%%%%%%%%%%%%		
\begin{frame}
	\frametitle{Hausaufgabe -- Lösung}

\begin{itemize}
\item[4.] Erläutern Sie stichpunktartig, welche (graphematische) Funktionen der Buchstabe \ab{h} in den folgenden Kontexten annimmt:

\begin{exe}
\exr{ex:04aHA4}
\settowidth\jamwidth{XXXXXXXXXXXXXXXXXXXXXXXXXXXXXXXXXX}
\begin{xlist}
	\ex Ha\underline{h}n: \loesung{2}{Dehnungs-h}
	
	\ex nä\underline{h}en: \loesung{3}{Silbentrennendes \ab{h}}
	
	\ex bein\underline{h}alten: \loesung{4}{Korrespondenz zu Phonem \textipa{/h/}}
	
	\ex Gesc\underline{h}ichte: \loesung{5}{Teil eines Trigraphen \ab{sch}} \loesung{5}{(Nicht Teil eines Lauts, sondern eines Graphems!)}
	
	\ex Geschic\underline{h}te: \loesung{6}{Teil eines Digraphen \ab{ch}} \loesung{6}{(Nicht Teil eines Lauts, sondern eines Graphems!)}
	
	\ex Dip\underline{h}thong: \loesung{7}{Teil eines Fremddigraphen \ab{ph}}
	
	\ex Dipht\underline{h}ong: \loesung{8}{Teil eines Fremddigraphen \ab{th}}
\end{xlist}
\end{exe}

\end{itemize}

\end{frame}


%%%%%%%%%%%%%%%%%%%%%%%%%%%%%%%%%%		
\begin{frame}
	\frametitle{Hausaufgabe -- Lösung}

\begin{itemize}
\item[5.] Geben Sie die \textbf{phonologische} Transkription, die \textbf{phonetische} Transkription und die \textbf{phonographische} Schreibung (nach der Phonem-Graphem-Korrespondenz) des folgenden Wortes an.

\begin{exe}
	\exr{ex:04aHA5} Abstellkammer
\end{exe}

\settowidth\jamwidth{XXXXXXXXXXXXXXXXXXXXXXXXXXXXXXXXX}
\item[] phonologisch: \loesung{2}{\textipa{/abStElkam@\textscr /}}

\item[] phonetisch: \loesung{3}{\textipa{[PapStElkam5]}}

\item[] phonographisch: \loesung{4}{\ab{abschtelkamer}}

\item[] \only<5->{\alertgreen{Hier erkennt man, dass es sich bei der phonographischen Trankskription um eine Phonem-Graphem-Korrespondenz (und nicht um eine Phon-Graphem-Korrespondenz) handelt.}}

\end{itemize}
\end{frame}

\input{loesungen-gk-literatur}


\end{document}