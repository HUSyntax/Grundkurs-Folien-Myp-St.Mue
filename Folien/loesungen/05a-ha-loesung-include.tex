%%%%%%%%%%%%%%%%%%%%%%%%%%%%%%%%%%
%% HA 1 - 05a Morphologie
%%%%%%%%%%%%%%%%%%%%%%%%%%%%%%%%%%

\begin{frame}
\frametitle{Hausaufgabe -- Lösung}

\begin{enumerate}
	\item[1.] Geben Sie an, \textbf{aus welchen Morphemen} die folgenden Wörter bestehen:
	
	\begin{exe}
		\exr{ex:05aHA1}
		\settowidth\jamwidth{XXXXXXXXXXXXXXXXXXXXXXXXXXXXX}
		\begin{xlist}
			\item verteilenden \loesung{2}{ver-, teil, -end, -en}
			\item Bereinigungsanlage \loesung{3}{be-, rein, -ig, -ung(-s), an-, lag, -e}
			\item Hausarbeit \loesung{4}{haus, arbeit}
		\end{xlist}
	\end{exe}	
	

	\item[2.] \textbf{Ordnen} Sie die \textbf{Beispiele} auf der linken Seite den \textbf{passenden Begriffen} auf der rechten Seite \textbf{zu}.

	\vspace{.5cm}

\begin{minipage}{0.4\textwidth}
	\centering
	\begin{tabular}{l}

		(A) \emph{be-} ist \ldots \\
		\hline
		(B) \emph{-ung} ist \ldots \\
		\hline
		(C) \emph{Druckerpatrone} ist \ldots \\
		\hline
		(D) \emph{schwarzer Freitag} ist \ldots \\
		\hline
		(E) \emph{kaufst} ist \ldots \\

	\end{tabular}
\end{minipage}
\hfill%
\begin{minipage}{0.56\textwidth}
	\centering
	\begin{tabular}{|l|l}

		\only<5->{\alertgreen{C}} & ein morphologisches Wort \\
		\hline
		\only<6->{\alertgreen{D}} & ein Mehrwortlexem \\
		\hline
		\only<7->{\alertgreen{A}} & ein phonetisch-phonologisches Wort \\
		\hline
		\only<8->{\alertgreen{B}} & ein Suffix\\
		\hline
		\only<9->{\alertgreen{E}} & ein flektivisches Wort \\

	\end{tabular}
\end{minipage}

\end{enumerate}

\end{frame}


%%%%%%%%%%%%%%%%%%%%%%%%%%%%%%%%%%
\begin{frame}
\frametitle{Hausaufgabe -- Lösung}

\begin{enumerate}
	\item[3.] Kreuzen Sie die richtigen Antworten an:

	\begin{itemize}
		\item[\alertgreen{$\checkmark$}] \alertgreen{\emph{grün} ist die Basis von \emph{begrün}(\emph{-en}).}
		
		\item[$\circ$] \emph{verarbeit-} ist die Wurzel von \emph{verarbeiten}.
		
		\item[\alertgreen{$\checkmark$}] \alertgreen{\emph{lier} in \emph{verlieren} ist ein unikales Morphem.}
		
		\item[\alertgreen{$\checkmark$}] \alertgreen{\emph{Autounfall} ist der Stamm von \emph{Autounfälle}}
	\end{itemize}	
	

	\item[4.] Bilden Sie \textbf{Minimalpaare}, um die Morphemhaftigkeit \textbf{der einzelnen Morpheme} in dem Wort \emph{verschreibt} zu ermitteln.


\settowidth\jamwidth{XXXXXXXXXXXXXXXXXXXXXXXXXXXXXXt}
	\begin{itemize}
		\item[\alertgreen{--}]<2-> \alertgreen{ver-\textbf{schreib}-t \vs ver-\textbf{läuf}-t} \loesung{2}{\ras \emph{schreib} ist ein Morphem}
		
		\item[\alertgreen{--}]<3-> \alertgreen{verschreib\textbf{-t} \vs verschreib\textbf{-st}} \loesung{3}{\ras \emph{-t} ist ein Morphem}
		
		\item[\alertgreen{--}]<4-> \alertgreen{\textbf{ver-}schreibt \vs \textbf{be-}schreibt} \loesung{4}{\ras \emph{ver-} ist ein Morphem}
	\end{itemize}

\end{enumerate}

\end{frame}


%%%%%%%%%%%%%%%%%%%%%%%%%%%%%%%%%%
\begin{frame}
\frametitle{Hausaufgabe -- Lösung}

\begin{enumerate}
	\item[5.] Gegeben sei der Satz in (\ref{ex:05aHA5}). Geben Sie \textbf{jeweils 3 Beispiele} für die unten angegebenen Morphemtypen an.
	
	\begin{exe}
		\exr{ex:05aHA5} Heute fragt Maria, ob wir den schönen Bären gekauft haben.
	\end{exe}
	
	\settowidth\jamwidth{XXXXXXXXXXXXXXXXXXXXXXXXXXXXXXXXXt}
	\begin{itemize}
		\item strukturalistische Idealfälle für Morpheme:
		\item[] \loesung{2}{heute, frag, Maria, ob, schön, Bär, kauf}
		
		\item Portmanteau-Morpheme:
		\item[] \loesung{3}{-t (Präsens, 3., Sg., Ind., Akt.), wir (1., Pl., Nom.),}
					\loesung{3}{-en (Mask., Sg., Akk. $+$ Präsens, 1., Pl., Ind. Akt.)}
		
		\item lexikalische Morpheme:
		\item[] \loesung{4}{heute, frag, Maria, schön, Bär, kauf} 
		
		\item grammatische Morpheme:
		\item[] \loesung{5}{-t, ob, den, -en, -en, ge- -t, hab, -en}
		
		\item freie Morpheme:
		\item[] \loesung{6}{heute, frag, Maria, ob, wir, den, schön, Bär, kauf, hab}
		
		\item gebundene Morpheme:
		\item[] \loesung{7}{-t, -en, -en, ge- -t, -en}
	\end{itemize}

\end{enumerate}

\end{frame}

