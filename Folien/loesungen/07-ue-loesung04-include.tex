%%%%%%%%%%%%%%%%%%%%%%%%
%Semantik 07-ue-loesung04
%%%%%%%%%%%%%%%%%%%%%%%%%


\begin{frame}
\frametitle{Übung -- Lösung}

\begin{itemize}
	\item Überprüfen Sie die Richtigkeit der folgenden Aussagen.
	
	\begin{itemize}
		\item Die komplexen Aussagen (\ref{ex:Tau1}) und (\ref{ex:Tau2}) sind \textbf{Tautologien}.
		
		\begin{exe}
		\exr{ex:Tau1} (p $\lor \lnot$ p)
		\exr{ex:Tau2} (p $\rightarrow$ p)
		\end{exe}
		
	\end{itemize}
\end{itemize}

\begin{table}
	\centering	
	%	\scalebox{0.8}{
	\begin{tabular}{c|c|c}
		\textbf{p}& \textbf{$\lnot$p} &\textbf{(p $\lor \lnot$p)} \\ 
		\hline 
		1 & 0 & \alertred{1}\\ 
		\hline 
		0 & 1 & \alertred{1} \\
	\end{tabular} 
	%	}
\end{table}
\begin{itemize}
	\item[]  \alertred{Der Satz ist eine Tautologie, weil die Aussage immer wahr ist (immer Wahrheitswert 1).}
\end{itemize}

\end{frame}


%%%%%%%%%%%%%%%%%%%%%%%%%%%%%%%%%%
\begin{frame}
\frametitle{Übung -- Lösung}

\begin{itemize}
	\item Überprüfen Sie die Richtigkeit der folgenden Aussagen.
	
	\begin{itemize}
		\item Die komplexe Aussagen (\ref{ex:Kon1}) ist eine \textbf{Kontradiktion}.
		
		\begin{exe}
		\exr{ex:Kon1} $\lnot$ (p $\lor \lnot$ p)
		\end{exe}
		
	\end{itemize}	
	
\end{itemize}

\begin{table}
	\centering	
	%	\scalebox{0.8}{
	\begin{tabular}{c|c|c|c}
		\textbf{p}& \textbf{$\lnot$p} &\textbf{(p $\lor \lnot$p)} & $\lnot$ \textbf{(p $\lor \lnot$p)}\\ 
		\hline 
		1 & 0 & 1& \alertred{0} \\ 
		\hline 
		0 & 1 & 1 & \alertred{0} \\
	\end{tabular} 
	%	}
\end{table}
\begin{itemize}
	\item[] \alertred{Der Satz ist ein Kontradiktion, weil der Wahrheitswert immer 0 ist. Die Aussage ist immer falsch.}
\end{itemize}

\end{frame}


%%%%%%%%%%%%%%%%%%%%%%%%%%%%%%%%%%
\begin{frame}
\frametitle{Übung -- Lösung}

\begin{itemize}
	\item Überprüfen Sie die Richtigkeit der folgenden Aussagen.
	
	\begin{itemize}
		\item Die komplexe Aussage (\ref{ex:Con1}) ist eine \textbf{Kontingenz}.
		
		\begin{exe}
		\exr{ex:Con1} ((p $\lor$ q) $\rightarrow$ q)
		\end{exe}
		
	\end{itemize}	
	
\end{itemize}

\begin{table}
	\centering	
	%	\scalebox{0.8}{
	\begin{tabular}{c|c|c|c}
		\textbf{p}& \textbf{q} & \textbf{(p $\lor$ q)} & \textbf{(p $\lor$ q) \ras q}\\ 
		\hline 
		1 & 0 & 1& \alertred{0} \\ 
		\hline 
		1 & 1 & 1 & \alertred{1} \\
		\hline 
		0 & 0 & 0 & \alertred{1}\\
		\hline 
		0 & 1 &  1 &  \alertred{1}\\
	\end{tabular} 
	%	}
\end{table}

\begin{itemize}
\item \alertred{Der Satz ist kontingent, d.\,h., dass sich unterschiedliche Wahrheitwerte ergeben können, die von der Welt abhängig sind. }
\end{itemize}

\end{frame}