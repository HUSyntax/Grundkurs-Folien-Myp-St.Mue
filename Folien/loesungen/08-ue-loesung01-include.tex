%%%%%%%%%%%%%%%%%%%%%%%%%%%%%%%%%%
%% UE 1 - 08 Pragmatik
%%%%%%%%%%%%%%%%%%%%%%%%%%%%%%%%%%

\begin{frame}
\frametitle{Übung -- Lösung}

Markieren und bestimmen Sie die deiktischen und anaphorischen Ausdrücke.
	
	\begin{enumerate}
		\item \alertgreen{Morgen} \alertgreen{(Temporaldeixis)} werde \alertgreen{ich} \alertgreen{(Personaldeixis)} \alertgreen{sie} \alertgreen{(Personaldeixis)} besuchen, obwohl es \alertgreen{mir} \alertgreen{(Personaldeixis)} zeitlich nicht passt.
		\item \alertgreen{Gestern} \alertgreen{(Temporaldeixis)} regnete es \alertgreen{vor dem Supermarkt} \alertgreen{(Lokaldeixis)}.
		\item Am 10.02.2014 hat Peter einen Sack Kartoffeln gekauft.
		\item \alertgreen{Ich} \alertgreen{(Personaldeixis)} treffe \alertgreen{Sie} (Sozialdeixis) in \alertgreen{Ihrem} \alertgreen{(Sozialdeixis)} Büro.
		\item Der Dozent wei\ss{}, dass es gut für \alertgreen{Sie} \alertgreen{(Sozialdeixis)} ist, \alertgreen{das} \alertgreen{(Objektdeixis)} zu lernen.
		\item Karl hat nicht hingeschaut und dann hat \alertgreen{er} \alertgreen{(Anapher)} \alertgreen{mich} \alertgreen{(Personaldeixis)} angefahren.
		\item Mario hat \alertgreen{ihn} \alertgreen{(Katapher)} rasiert und Peter hat \alertgreen{sich} \alertgreen{(Anapher)} gewaschen.
	\end{enumerate}     

\end{frame}