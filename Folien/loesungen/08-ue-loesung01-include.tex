%%%%%%%%%%%%%%%%%%%%%%%%%%%%%%%%%%
%% UE 1 - 08 Pragmatik
%%%%%%%%%%%%%%%%%%%%%%%%%%%%%%%%%%

\begin{frame}
\frametitle{Übung -- Lösung}

Markieren und bestimmen Sie die deiktischen und anaphorischen Ausdrücke.
	
	\begin{enumerate}
		\item \alertred{Morgen} \alertred{(Temporaldeixis)} werde \alertred{ich} \alertred{(Personaldeixis)} \alertred{sie} \alertred{(Personaldeixis)} besuchen, obwohl es \alertred{mir} \alertred{(Personaldeixis)} zeitlich nicht passt.
		\item \alertred{Gestern} \alertred{(Temporaldeixis)} regnete es \alertred{vor dem Supermarkt} \alertred{(Lokaldeixis)}.
		\item Am 10.02.2014 hat Peter einen Sack Kartoffeln gekauft.
		\item \alertred{Ich} \alertred{(Personaldeixis)} treffe \alertred{Sie} (Sozialdeixis) in \alertred{Ihrem} \alertred{(Sozialdeixis)} Büro.
		\item Der Dozent wei\ss{}, dass es gut für \alertred{Sie} \alertred{(Sozialdeixis)} ist, \alertred{das} \alertred{(Objektdeixis)} zu lernen.
		\item Karl hat nicht hingeschaut und dann hat \alertred{er} \alertred{(Anapher)} \alertred{mich} \alertred{(Personaldeixis)} angefahren.
		\item Mario hat \alertred{ihn} \alertred{(Katapher)} rasiert und Peter hat \alertred{sich} \alertred{(Anapher)} gewaschen.
	\end{enumerate}     

\end{frame}