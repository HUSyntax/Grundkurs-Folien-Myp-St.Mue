%%%%%%%%%%%%%%%%%%%%%%%%%%%%%%%%%%
%% UE 1 - 08 Pragmatik
%%%%%%%%%%%%%%%%%%%%%%%%%%%%%%%%%%

\begin{frame}
\frametitle{Übung -- Lösung}
		
		\begin{enumerate}
			\item Gegeben sei der Satz unter (\mex{1}):
			\ea
			Einige der US-amerikanischen Beamten wissen, wer Richard erdrosselt hat.
			\z
			\item [] Geben Sie bei jedem der Sätze unter (\mex{1})--(\mex{4}) an, ob es sich um eine Implikatur, oder ob es sich um eine Präsupposition zu (1) handelt. Schreiben Sie die richtige Antwort hinter den jeweiligen Satz. Wenn es sich um eine Präsupposition handelt, testen Sie dies anhand eines der Präsuppositionstests. \\
			\textbf{NB:} Vorsicht, zuweilen wird keine der Relationen wiedergegeben!
			\ea\label{l:ex:Prag43l} Es existieren US-amerikanische Beamte. \textcolor{red}{\ras Präsupposition}
			\ex\label{l:ex:Prag44l} Richard war ein Semantiker. \textcolor{red}{\ras weder noch}
			\ex\label{l:ex:Prag45l} Nicht alle US"=amerikanischen Beamten wissen, wer den Mord begangen hat. \textcolor{red}{\ras Implikatur}
			\ex\label{l:ex:Prag46l} Richard wurde erdrosselt. \textcolor{red}{\ras Präsupposition}
			\z	
		\end{enumerate}
		
	\end{frame}
	
	%%%%%%%%%%%%%%%%%%%%%%%%%%%%%%%%%%%%%%%%%%%
	
	\begin{frame}
		\frametitle{Übung -- Lösung}
		
		\begin{enumerate}
			\item[2.] Bestimmen und kennzeichnen Sie zwei deiktische Ausdrücke im Satz (\mex{1}). Geben Sie zudem eine Anapher mit ihrem Antezendens an.
			\ea Angelika hat gestern erwähnt, dass Irene sich dort mit den Formeln amüsiert hat.
			\z 
			
			
			\begin{itemize}
				\item[] \textcolor{red}{Ausdruck: \emph{gestern}, Art: Temporaldeixis}
				\item[] \textcolor{red}{Ausdruck: \emph{dort}, Art: Lokaldeixis}
				\item[] \textcolor{red}{Anapher: \emph{sich}, Antezedens: \emph{Irene}}
				\item[] \textcolor{red}{Auch möglich: \emph{den Formeln} \ras Objektdeixis}
			\end{itemize}
			
		\end{enumerate}
		
	\end{frame}
	
	%%%%%%%%%%%%%%%%%%%%%%%%%%%%%%%%%%%%%%%%%%%%
	
	\begin{frame}
		\frametitle{Übung -- Lösung}
		
		\begin{enumerate}
			\item[3.] Kreuzen Sie für Satz (\mex{1}) alle Sätze in der unten stehenden Liste an, die (konversationelle) Implikaturen dieses Satzes darstellen.
			\ea
			Gottfried hat einige Nachbarn beleidigt.
			\z 
			\begin{itemize}
				\item[$\circ$] Gottfried hat einen Nachbarn.
				\item[\textcolor{red}{$\checkmark$}] \textcolor{red}{Gottfried hat nicht alle Nachbarn beleidigt.}
				\item[$\circ$] Gottfried ist ein unbeliebter Mensch.
				\item[\textcolor{red}{$\checkmark$}] \textcolor{red}{Gottfried hat etwas Unhöfliches gesagt.}
				\item[\textcolor{red}{$\checkmark$}] \textcolor{red}{Gottfried hat einige Nachbarn nicht beleidigt.}
			\end{itemize}
			
		\end{enumerate}


\end{frame}