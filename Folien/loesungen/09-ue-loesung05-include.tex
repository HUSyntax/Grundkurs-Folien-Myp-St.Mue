%%%%%%%%%%%%%%%%%%%%%%%%%%%%%%%%%%
%% UE 5 - 09 Übungen
%%%%%%%%%%%%%%%%%%%%%%%%%%%%%%%%%%

\begin{frame}
\frametitle{Übung: Semantik/Pragmatik -- Lösung}

\begin{itemize}
	\item[15.] Geben Sie die Bedeutungsrelationen (so genau wie möglich) zwischen den folgenden Wörtern an.
	
	\begin{exe}
		\exr{ex:15}
	\settowidth \jamwidth{\alertgreen{kontradiktorische Antonymie}}
	
		\begin{xlist}
			\ex satt -- hungrig \only<2->{\jambox{\alertgreen{konträre Antonymie}}}
			\ex erwerben -- kaufen \only<3->{\jambox{\alertgreen{(partielle) Synonymie}}}
			\ex Haare -- Kopf \only<4->{\jambox{\alertgreen{Meronymie}}}
			\ex schuldig -- nicht schuldig \only<5->{\jambox{\alertgreen{kontradiktorische Antonymie}}}
			\ex heute -- Häute \only<6->{\jambox{\alertgreen{Homophonie}}}
			\ex Tiger -- Katze \only<7->{\jambox{\alertgreen{Hyponymie}}}
			\ex fruchtbar -- unfruchtbar \only<8->{\jambox{\alertgreen{kontradiktorische Antonymie}}}
		\end{xlist}
	
	\end{exe}
	
\end{itemize}

\end{frame}

%%%%%%%%%%%%%%%%%%%%%%%%%%%%%%%%%%

\begin{frame}
	
\begin{itemize}
	\item[16.] Illustrieren Sie die Begriffe Satzbedeutung, Äußerungsbedeutung und Sprecherbedeutung mithilfe des folgenden Satzes.
	
	\begin{exe}
		\exr{ex:16} Ich glaube, du gehst jetzt! \\
		\ras Peter zu Klaus am 25. August 2020 um 20:30 Uhr.
	\end{exe}
	
	\item Satzbedeutung: \\ \only<2->{\alertgreen{Der Sprecher des Satzes glaubt (zum Zeitpunkt der Äußerung), dass der Adressat der Äußerung geht.}}
	\item Äußerungsbedeutung: \\ \only<3->{\alertgreen{Klaus glaubt, dass Peter am 25. August 2020 um 20:30 Uhr geht.}}
	\item Sprecherbedeutung: \\ \only<4->{\alertgreen{Peter fordert Klaus bestimmt auf (\zB nach einer Auseinandersetzung) sofort zu gehen.}}
	
\end{itemize}

\end{frame}

%%%%%%%%%%%%%%%%%%%%%%%%%%%%%%%%%%

\begin{frame}

\begin{itemize}
	\item[17.] Geben Sie die Bedeutungsrelationen zwischen den folgenden Sätzen an.
	
	\begin{exe}
		\exr{ex:17}
	\settowidth \jamwidth{\only<2->{\alertgreen{\ras a impliziert b}}}
	
		\begin{xlist}
			\ex Hinter dem Baum steht ein Bär. \jambox{\only<2->{\alertgreen{Implikation}}}
			\ex Hinter dem Baum steht ein Tier. \jambox{\only<2->{\alertgreen{\ras a impliziert b}}}
		\end{xlist}
	
		\exr{ex:18}
		
		\begin{xlist}
			\ex Peter fängt an zu arbeiten. \jambox{\only<3->{\alertgreen{Paraphrase}}}
			\ex Peter nimmt die Arbeit auf.
		\end{xlist}
		
		\exr{ex:19}
		
		\begin{xlist}
			\ex Sandra ist groß. \jambox{\only<4->{\alertgreen{Kontradiktion}}}
			\ex Sandra ist nicht-groß.
		\end{xlist}
		
		\exr{ex:20}
		
		\begin{xlist}
			\ex Ich habe alle Studenten gesehen. \jambox{\only<5->{\alertgreen{Paraphrase}}}
			\ex Ich habe nicht einen Studenten nicht gesehen.
		\end{xlist}
	
		\exr{ex:21}
		
		\begin{xlist}
			\ex Maria geht wandern. \jambox{\only<6->{\alertgreen{Inkompatibilität}}}
			\ex Maria macht eine Kreuzfahrt.
		\end{xlist}
		
		\exr{ex:22}
		
		\begin{xlist}
			\ex Gert ist verletzt. \jambox{\only<7->{\alertgreen{Implikation}}}
			\ex Gert hat ein gebrochenes Bein. \jambox{\only<6->{\alertgreen{\ras b impliziert a}}}
		\end{xlist}
		
	\end{exe}

\end{itemize}
	
\end{frame}

%%%%%%%%%%%%%%%%%%%%%%%%%%%%%%%%%%

\begin{frame}
	
\begin{itemize}
		
	\item[18.] Geben Sie eine Wahrheitswerttabelle für den folgenden aussagenlogischen Ausdruck an und bestimmen Sie, ob es sich dabei um eine tautologische, eine kontradiktorische oder eine kontingente Aussage handelt.
	
	\begin{exe}
		\exr{ex:23} $((p \rightarrow q) \lor q)$
	\end{exe}
		
		\begin{table}
			\centering
			\scalebox{.95}{
				\only<2->{\alertgreen{
					\begin{tabular}{c|c|c|c}
					p & q & $(p \rightarrow q)$ & $((p \rightarrow q) \lor q)$ \\
					\hline
					1 & 1 & 1 & 1 \\
					\hline
					1 & 0 & 0 & 0 \\
					\hline
					0 & 1 & 1 & 1 \\
					\hline
					0 & 0 & 1 & 1 \\
					\end{tabular}
					}}}
		\end{table}

\medskip
	
	\only<2->{\item \alertgreen{Bei dem vorangehenden aussagenlogischen Ausdruck handelt es sich um eine kontingente Aussage.}}
				
\end{itemize}
	
\end{frame}

%%%%%%%%%%%%%%%%%%%%%%%%%%%%%%%%%%

\begin{frame}

\begin{itemize}
	\item[19.] Markieren Sie alle deiktischen und anaphorischen Elemente in den folgenden Sätzen und spezifizieren Sie diese.
		
	\begin{exe}
		\exr{ex:24}
		
		\begin{xlist}
			\ex Sie haben diese Tür nicht geschlossen.
			\ex Gestern war mir das Wetter echt zu kalt!
			\ex Peter wusste, dass er es sich dort gemütlich machen würde.
		\end{xlist}
	
	\end{exe}

\end{itemize}

\end{frame}

%%%%%%%%%%%%%%%%%%%%%%%%%%%%%%%%%%

\begin{frame}

\begin{itemize}
	\item[19.] Markieren Sie alle deiktischen und anaphorischen Elemente in den folgenden Sätzen und spezifizieren Sie diese.
	
	\begin{exe}
		\exr{ex:24}
		
		\begin{xlist}
			\ex \alertgreen{Sie} haben \alertgreen{diese} Tür nicht geschlossen.
			\ex \alertgreen{Gestern} war \alertgreen{mir} das Wetter echt zu kalt!
			\ex Peter wusste, dass \alertblue{er} es \alertblue{sich} \alertgreen{dort} gemütlich machen würde. 
			%	\ex \gqq{Ich bin sehr glücklich, mich wieder für das WTA-Finale qualifiziert zu haben. Ich freue mich darauf, dort anzutreten und gegen die Besten der Welt zu spielen}, sagte die 27-Jährige, die im vergangenen Jahr nur als Ersatzspielerin mitfahren durfte.
		\end{xlist}
	
	\end{exe}

\end{itemize}	
	
	\begin{minipage}[t]{0.35\textwidth}

	\begin{itemize}
		\item \alertgreen{Deiktische Elemente:}
		\only<2->{
		
			\begin{itemize}
				 \item \only<2->{\alertgreen{Sie: Sozialdeixis}}
				 \item \only<2->{\alertgreen{diese: Objektdeixis}}
				 \item \only<2->{\alertgreen{gestern: Temporaldeixis}}
				 \item \only<2->{\alertgreen{mir: Personaldeixis}}
				 \item \only<2->{\alertgreen{dort: Lokaldeixis}} 
			\end{itemize}		
			}

	\end{itemize}
	
	\end{minipage}
	\begin{minipage}[t]{0.60\textwidth}
	
	\begin{itemize}
		\item \alertblue{Anaphorische Elemente:}
		\only<2->{
		
			\begin{itemize}
				\item \only<2->{\alertblue{er: anaphorischer Ausdruck; Antezedens: \textit{Peter}}}
				\item \only<2->{\alertblue{sich: anaphorischer Ausdruck; Antezedens: \textit{er}}}
			\end{itemize}
			}
		
	\end{itemize}

	\end{minipage}


\end{frame}

%%%%%%%%%%%%%%%%%%%%%%%%%%%%%%%%%%

\begin{frame}
	
\begin{itemize}
	\item[20.] Bestimmen Sie die Art von Folgerung (Implikation, Präsupposition, Implikatur), die zwischen dem ersten und den folgenden Sätzen besteht:
	
	\begin{exe}
		\exr{ex:25} Sogar Peter hat zwei Kinder.
	\settowidth \jamwidth{\only<2->{\alertgreen{konversationalle Implikatur}}}
		
		\begin{xlist}
			\ex Peter hat nicht mehr als zwei Kinder. \jambox{\only<2->{\alertgreen{konversationalle Implikatur}}}
			\ex Es gibt ein Individuum namens Peter. \jambox{\only<3->{\alertgreen{Präsupposition}}}
			\ex Peter ist Vater. \jambox{\only<4->{\alertgreen{semantische Implikation}}}
			\ex Peter hat vier Kinder. \jambox{\only<5->{\alertgreen{keine Folgerung}}}
			\ex Überraschenderweise hat Peter Kinder. \jambox{\only<6->{\alertgreen{konventionelle Implikatur}}}
		\end{xlist} 
		
	\end{exe}

\end{itemize}
	
\end{frame}

%%%%%%%%%%%%%%%%%%%%%%%%%%%%%%%%%%

\begin{frame}

\begin{itemize}
	\item[21.] Bestimmen Sie jeweils eine semantische Implikation aus den folgenden Sätzen:
	
	\begin{exe}
		\exr{ex:26}
		
		\begin{xlist}
			\ex In einem Schuhkarton gibt es Platz für zwei Schuhe.
			\ex Saskia hat eine Schwedin geheiratet.
		\end{xlist}

	\end{exe}
	
\end{itemize}
	
	\only<2->{\alertgreen{(26a): $\vDash$ In einem Schuhkarton gibt es Platz für einen Schuh.}} ~\\
\medskip
	\only<3->{\alertgreen{(26b): $\vDash$ Saskia hat eine Nordeuropäerin geheiratet.}}
	

\end{frame}

%%%%%%%%%%%%%%%%%%%%%%%%%%%%%%%%%%

\begin{frame}
	
\begin{itemize}
		
	\item[22.] Bestimmen Sie jeweils eine Präsupposition aus den folgenden Sätzen:
	
	\begin{exe}
		\exr{ex:27}
		
		\begin{xlist}
			\ex Ich freue mich darüber, dass wir die Klausur bestanden haben.
			\ex Auch Maria ist schwanger.
			\ex Alle Geiseln wurden gerettet.
			\ex Sie mögen immer noch Syntax.
		\end{xlist}

	\end{exe}
		
\end{itemize}

	\only<2->{\alertgreen{(27a): \prspp Wir haben die Klausur bestanden.}} ~\\
\medskip
	\only<3->{\alertgreen{(27b): \prspp Mindestens eine weitere Entität neben Maria ist schwanger.}} ~\\
\medskip
	\only<4->{\alertgreen{(27c): \prspp Die Geiseln waren in Gefahr.}} ~\\
\medskip
	\only<5->{\alertgreen{(27d): \prspp Sie mochten bisher Syntax.}}
	
\end{frame}

%%%%%%%%%%%%%%%%%%%%%%%%%%%%%%%%%%

\begin{frame}
	
\begin{itemize}
	\item[23.] Geben Sie an, ob eine Maxime (scheinbar) verletzt oder befolgt wurde und um welche es sich handelt, um die angegebene Implikatur zu erhalten.
	
	\begin{exe}
		\exr{ex:28} Wir haben einige Personen entlassen.\\
		$+>$ Es wurden nicht alle entlassen.
	\end{exe}
	
	\begin{exe}
		\exr{ex:29} A: Wie war das Bewerbungsgespräch?\\
		B: Das Wetter ist ja super heute!\\
		$+>$ Es war furchtbar!
	\end{exe}
	
\end{itemize}

	\only<2->{\alertgreen{(\ref{ex:28}): Befolgung der Quantitätsmaxime}} ~\\
\medskip
	\only<3->{\alertgreen{(\ref{ex:29}): (scheinbare) Verletzung der Relevanzmaxime}}

\end{frame}

%%%%%%%%%%%%%%%%%%%%%%%%%%%%%%%%%%